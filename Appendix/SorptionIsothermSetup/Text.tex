\chapter{Batch test experiment preparations}\label{appSec:IsothermSetup}

\section{Preliminary batch tests}
The batch tests were prepared by spiking 50 mL Milli-Q water with PFOA to 1.3 mg/L and added 1 g CWC and 100 mg ULS in two PP tubes and placed in an end-over-end shaker for 14 d. DSL was not included in this step because the samples were not provided at this point. The aqueous concentration in the two test tubes were determined according to DIN38407-42 by the accredited laboratory Eurofins Norway. Sorption of PFCA to ULS was expected to be lower than for CWC. However, the test run for sorption of PFOA to ULS showed sorption two orders of magnitude higher than expected. Therefore, the same $K_d$ values were used for calculating spike concentrations for the ULS, DSL, and CWC. A 100 mg biochar dose was decided to be ideal to balance convenient spike concentrations with sufficient sorption. \cref{apptab:prelim} shows the results from the preliminary sorption experiments. 

\begin{table}
\centering
\caption{Partition coefficients ($K_d$) for PFOA from the preliminary batch tests with CWC and ULS.}
\label{apptab:prelim}
\begin{tabular}{llllll} \toprule
Biochar & $C_i~(\mu g/L)$ & $C_s~(\mu g/kg)$ & $C_w~(\mu g/L)$ & $K_d~(L/kg)$ & $\log~K_d$ \\ \midrule
CWC & 1 335 & 66 652 & 0.099 & 673 250 & 5.83 \\
ULS & 1 335 & 630 824 & 2.7 & 233 638 & 5.37 \\ \bottomrule
\end{tabular}
\end{table}

\begin{table}
\centering
\caption{Biochar-water distribution coefficients ($K_{BC}$) for PFCAs derived from \cite{XiaoSI2017}.} 
\label{apptab:Kbc}
\begin{threeparttable}
    \begin{tabular}{@{}lcc@{}}
    \toprule
    \multicolumn{1}{l}{\begin{tabular}[l]{@{}l@{}}Compound\end{tabular}} &  \multicolumn{1}{c}{\begin{tabular}[c]{@{}c@{}}log $\mathrm{K_{BC}}$\\ \citep{XiaoSI2017}\end{tabular}} & \multicolumn{1}{c}{\begin{tabular}[c]{@{}c@{}}est. log $\mathrm{K_{BC}}$ \end{tabular}} \\ \midrule
    PFPeA & 4.16 & 4.16 \\
    PFHxA & 4.15 & 4.15 \\
    PFHpA & 4.49 & 4.49 \\
    PFOA & 4.76 & 4.76 \\
    PFNA & *4.89 & 4.89 \\
    PFDA & *5.09 & 5.09 \\ \bottomrule             
    \end{tabular}
\begin{tablenotes}
\item * not included in \citep{XiaoSI2017}, so the values are extrapolated from the shorter chain lengths.
\end{tablenotes}
\end{threeparttable}
\end{table}

\begin{table}[ht]
\centering
\caption{Expected versus analytic standard concentration of working standards used to spike the batch shaking tests.}
\label{apptab:standards}
\begin{tabular}{lrrrrrr}
\toprule
Compound & \multicolumn{3}{l}{Working standard single (mg/L)} & \multicolumn{3}{l}{Working standard cocktail (ug/L)} \\ \cmidrule(l){2-4} \cmidrule(l){5-7} 
 & Expected & Analytic & \% deviation & Expected & Analytic & \% deviation \\ \midrule
PFPeA & 5.2 & 3.3 & 36 & 9.4 & 8.8 & 6 \\
PFHxA & 14.2 & 7.8 & 45 & 18.7 & 26.1 & -39 \\
PFHpA & 3.3 & 3.0 & 9 & 4.0 & 4.8 & -19 \\
PFOA & 13.2 & 22.7 & -71 & 35.7 & 61.7 & -73 \\
PFNA & 17.8 & 16.2 & 9 & 48.5 & 72.2 & -49 \\
PFDA & 16.8 & 13.1 & 22 & 153.0 & 165.2 & -8 \\ \bottomrule
\end{tabular}
\end{table}

\begin{table}[hb]
    \centering
    \caption{Purity, stock form and LOQs of the PFCAs used for the sorption isotherms. iLOQ = instrumental LOQ for  the extract, mLOQ = method LOQ.}
    \label{apptab:LOQ}
    \begin{tabular}{@{}lllcc@{}}
    \toprule
    \multicolumn{1}{l}{Compound}  & \multicolumn{1}{l}{Purity}  & \multicolumn{1}{l}{Stock form} & \multicolumn{1}{l}{iLOQ} & \multicolumn{1}{l}{mLOQ}  \\ 
    & & & \multicolumn{1}{l}{(ng mL\textsuperscript{-1})}  & \multicolumn{1}{l}{(ng L\textsuperscript{-1})} \\ \midrule
     PFPeA  & 97 \%   & liquid      & 0.05 & 0.50   \\
     PFHxA  & 97 \%   & liquid      & 0.10 & 1.00   \\
     PFHpA  & 99 \%   & crystalline & 0.01 & 0.10   \\
     PFOA   & 95 \%   & powder      & 0.05 & 0.50   \\
     PFNA   & 97 \%   & crystalline & 0.05 & 0.50   \\
     PFDA   & 98 \%   & flakes      & 0.10 & 1.00   \\ \bottomrule
    \end{tabular}
\end{table}


\begin{table}
\centering
\caption{Analysis results of a dilution of the standards (STD) used for spiking and filter blanks (FB) prepared at the same concentrations in $\mu g/L$.}
\label{apptab:FB}
\begin{tabular}{lrrrrrr} \toprule
Compound & \multicolumn{1}{c}{STD-1} & \multicolumn{1}{c}{STD-2} & \multicolumn{1}{c}{STD-3} & \multicolumn{1}{c}{FB-1} & \multicolumn{1}{c}{FB-2} & \multicolumn{1}{c}{FB-3} \\ \midrule
PFPeA & 1.38 & 1.40 & 1.18 & 1.31 & 1.19 & 3.86 \\
PFHxA & 30.3 & 33.3 & 30.2 & 23.6 & 26.9 & 30.2 \\
PFHpA & 0.042 & 2.30 & 2.07 & 2.15 & 1.96 & 2.01 \\
PFOA & 87.8 & 88.0 & 88.8 & 89.1 & 88.8 & 101.3 \\
PFNA & 11.5 & 11.4 & 12.1 & 11.8 & 12.5 & 11.8 \\
PFDA & 8.04 & 5.16 & 8.81 & 10.7 & 5.11 & 8.39 \\ \bottomrule
\end{tabular}
\end{table}


