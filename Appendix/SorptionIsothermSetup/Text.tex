\chapter{Sorption isotherm setup}\label{appSec:IsothermSetup}
In the following section, raw data for PFCA standard concentration calculations, mass of biochar weighed for each batch test, and pipetting volume are presented.

\begin{table}
\centering
\caption{Expected versus analytic standard concentration of working standards used to spike the batch shaking tests.}
\label{apptab:standards}
\begin{tabular}{lrrrrrr}
\toprule
Compound & \multicolumn{3}{l}{Working standard single (mg/L)} & \multicolumn{3}{l}{Working standard cocktail (ug/L)} \\ \cmidrule(l){2-4} \cmidrule(l){5-7} 
 & Expected & Analytic & \% deviation & Expected & Analytic & \% deviation \\ \midrule
PFPeA & 5.2 & 3.3 & 36 & 9.4 & 8.8 & 6 \\
PFHxA & 14.2 & 7.8 & 45 & 18.7 & 26.1 & -39 \\
PFHpA & 3.3 & 3.0 & 9 & 4.0 & 4.8 & -19 \\
PFOA & 13.2 & 22.7 & -71 & 35.7 & 61.7 & -73 \\
PFNA & 17.8 & 16.2 & 9 & 48.5 & 72.2 & -49 \\
PFDA & 16.8 & 13.1 & 22 & 153.0 & 165.2 & -8 \\ \bottomrule
\end{tabular}
\end{table}

%Consider removing stock form, purity, only leave LOQ 
\begin{table}
    \centering
    \caption{Purity, stock form and LOQs of the PFCAs used for the sorption isotherms. iLOQ = instrumental LOQ for  the extract, mLOQ = method LOQ.}
    \label{apptab:LOQ}
    \begin{tabular}{@{}lllcc@{}}
    \toprule
    \multicolumn{1}{l}{Compound}  & \multicolumn{1}{l}{Purity}  & \multicolumn{1}{l}{Stock form} & \multicolumn{1}{l}{iLOQ} & \multicolumn{1}{l}{mLOQ}  \\ 
    & & & \multicolumn{1}{l}{(ng mL\textsuperscript{-1})}  & \multicolumn{1}{l}{(ng L\textsuperscript{-1})} \\ \midrule
     PFPeA  & 97 \%   & liquid      & 0.05 & 0.50   \\
     PFHxA  & 97 \%   & liquid      & 0.10 & 1.00   \\
     PFHpA  & 99 \%   & crystalline & 0.01 & 0.10   \\
     PFOA   & 95 \%   & powder      & 0.05 & 0.50   \\
     PFNA   & 97 \%   & crystalline & 0.05 & 0.50   \\
     PFDA   & 98 \%   & flakes      & 0.10 & 1.00   \\ \bottomrule
    \end{tabular}
\end{table}

\section{Preliminary batch tests}
The batch tests were prepared by spiking 50 mL Milli-Q water with PFOA to 1.3 mg/L and added 1 g CWC and 100 mg ULS in two PP tubes and placed in an end-over-end shaker for 14 d. The aqueous concentration in the two test tubes were determined according to DIN38407-42 by the accredited laboratory Eurofins Norway. Sorption of PFCA to ULS was expected to be lower than for CWC. However, the test run for sorption of PFOA to ULS showed sorption two orders of magnitude higher than expected. Therefore, the same $K_d$ values were used for calculating spike concentrations for the three biochars. A 100 mg biochar dose was decided to be ideal to balance convenient spike concentrations with sufficient sorption. In summary, the same analyte spike concentrations and biochar doses were used to make the isotherms for CWC, ULS and DSL.


