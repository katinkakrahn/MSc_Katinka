\newpage
\addcontentsline{toc}{chapter}{Abstract}
\section*{Abstract}

Due to considerable persistence, bioaccumulation, and toxicity of per- and polyfluoroalkyl substances (PFASs), remediation of soil and water contaminated with PFAS is urgently needed. The potential for using various organic waste materials in the production of biochar sorbents for PFAS has sparked great interest among researchers.

In a recent study by S\o rmo et al. (2021), eight activated waste timber biochars applied with a 5\% amendment dose to heavily PFAS-contaminated, low-TOC (1.6\%) soil, reduced leachate concentration with 98-100\%. Since strong sorption was seen for these waste timber biochars, the present work tested the sorption behavior of two biochars from sewage sludge pyrolyzed at 700 ºC to C4-9 perfluorinated carboxylic acids (PFCAs). The sludge feedstocks were 1) Ullensaker sludge (ULS) containing PFAS-enriched wastewater from Oslo Airport, and 2) Biorest Lindum (BRL), a sludge digestate. In laboratory batch experiments, we prepared ten-point biochar-water sorption isotherms with ULS and BRL and six-point soil-biochar-water isotherms using a clean, sandy soil (1.3\% TOC) and biochar at 2\% w/w spiked with both a PFCA cocktail and C4-9 individually. 

Previous studies of pyrolyzed sewage sludge conclude with poor sorption to PFAS. However, a preliminary analysis of a selection of ULS batch samples (n=22) indicate sorption almost as strong to ULS as clean wood chips, the method control in our experiments. The initial spike concentrations were reduced with >98\% for all PFCAs tested, where K$_d$ ranged from 4.29-5.85 by increasing chain length in accordance with the literature. 

Even though elemental composition analysis of ULS is yet to be conducted, sludge char is expected to have low carbon content, porosity, and internal surface area—properties associated with poor sorption—and instead be enriched with minerals, especially iron oxides. Since ULS sorbs stronger than expected, we speculate that the sorption mechanisms between this biochar and PFAS is not dominated by hydrophobic interactions that were observed for the waste timber in S\o rmo et al., but by electrostatic forces between the mineral phases in the sludge matrix and the carboxylate groups of PFCAs.

The high sorption capacity of ULS is promising for the implementation of a sustainable and cost-effective waste management method of problematic wastes such as contaminated sewage sludge, and will be an important contributor to a circular economy. 


%aim / bakgrunn og hensikt


%Methods


%Results


%conclusion


