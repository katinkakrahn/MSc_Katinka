\newpage
\addcontentsline{toc}{chapter}{Abstract}
\section*{Abstract}
Due to the considerable persistence, mobility, and toxicity of per- and polyfluoroalkyl substances (PFAS), remediation of soil and water contaminated with PFAS is urgently needed. The potential for using various organic waste materials in the production of biochar sorbents for PFAS has sparked great interest among researchers. Biochar has potentials for the removal of contamination in substrates by strong sorption to the material and is also one of the most promising technologies for carbon sequestration. The present work tested the sorption strength of two sewage sludge biochars pyrolyzed at 700 \textdegree C to six perfluorinated carboxylic acids (PFCA). The sewage sludge feedstocks were Ullensaker sludge (ULS) containing PFAS-enriched wastewater from Oslo Airport, and digested sludge Lindum (DSL), the remaining fraction after anaerobic digestion of sludge biomass into bio-gas. Biochar from clean wood chips (CWC), a highly carbonaceous feedstock, was used to compare the relative sorption abilities between a clean substrate and that from sewage sludge feedstocks. Batch tests were prepared with ULS, DSL, and CWC biochars and different concentrations of spiked PFCAs. Both single-compound experiments and a cocktail of different PFCAs in both soil and with biochar only were tested to measure sorption attenuation, the sorption suppression effect that occurs in more complex systems. Ten-point biochar-water sorption isotherms and six-point soil-biochar-water isotherms using a sandy soil (1.3\% TOC) were generated. Pore size distribution was conducted for surface area and pore volume by research partners for the sewage sludge biochars that were determined by $\mathrm{CO_2}$ sorptometry for small pores (0.3-1.5 $nm$) and $\mathrm{N_2}$ sorptometry for large pores ($>1.5 nm$). 

The main hypothesis this thesis sought to test was if biochar from sewage sludge can serve as an equally efficient sorbent as biochar from clean wood chips. In all batch tests, the sewage sludge biochars were stronger sorbents than CWC biochar. Freundlich partition coefficients ($\log~K_F$) in $(\mu g/kg)/(\mu g/L)^n$ were highest for the experiments with biochar and singly spiked PFCAs. Sorption was positively correlated with increasing perfluorinated chain length. For PFDA, $\log~K_F$s were 6.00 $\pm$ 0.04, 5.61 $\pm$ 0.02, and 5.22 $\pm$ 0.07 for ULS, DSL, and CWC biochars respectively. Sorption was most attenuated in the presence of a PFCA cocktail (by 89-99\%) between biochar samples and PFOA, PFNA, and PFDA, and biochar mixed with soil reduced sorption of PFOA by 65-90\%. Attenuation in both soil and cocktail was between 90-99 \% between biochar samples and PFOA, PFNA, and PFDA. Furthermore, the chain length dependency on sorption indicates that hydrophobic interaction is likely the dominant sorption mechanism over electrostatic interactions between biochar and PFAS. Differences in sorption between the CWC, DSL, and ULS biochars were explained by three main findings: 1) a too large fraction of pores for CWC were too small pores to fit the PFCAs, whose maximum molecular dimensions are 0.96-1.54 $nm$, 2) CWC has a much lower pore volume for pore sizes above 1.5 $nm$ than DSL and ULS which influences the pore filling mechanism, and 3) ULS has a higher pore volume, surface area and carbon-content than DSL which explains the difference in sorption between the sludge chars. However, since these conclusions were drawn based on no more than three biochar samples, further research is needed to test a larger sample size in order to corroborate the findings in this study.

The high sorption strength of the sewage sludge biochars found in this research are promising for the implementation of a sustainable and cost-effective waste management method. The possibility of converting contaminated wastes that are challenging and costly to handle into commercial sorbents provide important contributions to a circular economy that simultaneously provides one solution to today's urgent need for carbon sequestration.