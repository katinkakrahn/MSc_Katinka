\newpage
\addcontentsline{toc}{chapter}{Abstract}
\section*{Abstract}
Due to the considerable persistence, mobility, and toxicity of per- and polyfluoroalkyl substances (PFAS), remediation of soil and water contaminated with PFAS is urgently needed. The potential for using various organic waste materials in the production of biochar sorbents for PFAS has sparked great interest among researchers. The present work tested the sorption strength of two sewage sludge biochars pyrolyzed at 700 \textdegree C to six perfluorinated carboxylic acids (PFCA). The sewage sludge feedstocks were Ullensaker sludge (ULS) containing PFAS-enriched wastewater from Oslo Airport, and digested sludge Lindum (DSL), the remaining fraction after anaerobic digestion of sludge biomass into bio-gas. Biochar from clean wood chips (CWC), a highly carbonaceous feedstock, was used to compare the relative sorption abilities between a clean substrate and heterogeneous sewage sludge feedstocks. Batch tests were prepared with ULS, DSL, and CWC and different concentrations of spiked PFCAs. Both single-compound experiments and a cocktail of different PFCAs were tested to measure sorption attenuation. Ten-point biochar-water sorption isotherms and six-point soil-biochar-water isotherms using a sandy soil (1.3\% TOC) were generated. 

In all batch tests, the sewage sludge biochars were stronger sorbents than CWC. Freundlich partition coefficients ($\log~K_F$) were highest for the experiments with biochar and singly spiked PFCAs. Sorption was positively correlated with increasing perfluorinated chain length. For PFDA, $\log~K_F$s were 6.00 $\pm$ 0.04, 5.61 $\pm$ 0.02, and 5.22 $\pm$ 0.07 for ULS, DSL, and CWC respectively. Sorption was most attenuated in the presence of a PFCA cocktail , and biochar mixed with soil, reduced sorption by  aqueous PFCA concentration reductions were  Differences in sorption between CWC, DSL, and ULS were explained by two main factors. 1) The difference in micropore and mesopore structure. A higher number of large micropores represented by low SA/PV ratio results in higher sorption of long-chain PFASs (PFOA, PFNA, and PFDA). 2) A higher proportion of carbon in the pore wall matrix further enhances sorption by creating more hydrophobic sorption sites. Since the conclusions drawn for the sorption mechanisms contributing to PFCA sorption are based on no more than three biochar samples, further research is needed to test a larger sample size in order to corroborate the findings in this study.

The high sorption capacity of ULS is promising for the implementation of a sustainable and cost-effective waste management method of problematic wastes such as contaminated sewage sludge, and will be an important contributor to a circular economy. Biochar is the most promising technology for carbon sequestration, and the possibility of converting wastes that are challenging to handle into commercial sorbents provide important contributions to a circular economy.