\newpage
\addcontentsline{toc}{chapter}{Preface}
\section*{Preface}
This thesis concludes my master's degree in Environment and Natural Resources at the Norwegian University of Life Sciences (NMBU). The work presented here was carried out in the fall and spring of 2021/22 in collaboration with the Norwegian Geotechnical Institute (\acrshort{NGI}), and was part of the Valorization of Organic Waste into Sustainable Products for Clean-up of Contaminated Water, Soil, and Air (\acrshort{VOW}) project funded by The Research Council of Norway (NFR 299070). It is a continuation of MSc work done by Nora Bjerkli (NMBU, May 2021), and a part of the ongoing research within the VOW project being done by Erlend S\o rmo, my main thesis advisor. 

S\o rmo is working on generating a larger database on the overall life cycle for biochars produced from waste materials. S\o rmo's work is examining a range of feedstocks, including wood chips, garden waste, digested sludge, raw sewage sludge, cardboard, food waste, reject, and residual wood waste. The overarching goal of the VOW project and S\o rmo's PhD is to develop "designer" biochar from pyrolysis of organic waste materials that have optimized binding properties as sorbents for various contaminants present in soil and water. The two sewage sludge biochars selected for the current study are part of S\o rmo's set of waste substrates. One contaminant class of great interest is per- and polyfluoroalkyl substances (PFAS) whose sorption affinity to sewage sludge biochar I have tested. The VOW-project is scheduled to be completed in 2023.

The VOW-project is a joint industry sustainability (BIA-X) project led by Gerard Cornelissen, my co-supervisor, from NGI, in collaboration with Lindum, Vesar and VEAS (waste/sewage-related management), Scanship (technology supplier), Lindum, Mivanor and Clairs (respectively stakeholders in product application to soil, water and air), and SINTEF (pyrolysis optimization developer). My work has also received important contributions from Hans Peter Arp, the project coordinator for the SLUDGEFFECT project (NFR 302371), who is a leading expert in the field of PFAS sorption.

My interest in writing this thesis has not only been to fulfill the requirements for an MSc degree, but to be able to publish the results of my research together with my thesis advisor, Erlend S\o rmo. My work has been one part of the work of a larger team of research partners whose aim has been to publish a scientific paper. With support from my advisors and NGI, I was given the opportunity to spend the month of January 2022 at the Norwegian University of Science and Technology (NTNU) in Trondheim to conduct my own sample analysis. Although challenging, being introduced to advanced analytical chemistry has been rewarding, and work that I am very proud to have been able to complete successfully. The process of writing my Master's thesis has sparked an interest to continue in academia as an organic chemistry researcher. For this reason, I expressed an interest in presenting my findings at a scientific conference. With funding from NGI and the VOW project, my work was accepted as a poster which will be presented at SETAC Europe's 32\textsuperscript{nd} annual meeting in Copenhagen, Denmark in May 2022, the same day this Master's thesis will be submitted. The poster is added to \cref{appSec:poster}.  

~\\
\centerline{Ås, May 2022} \\
\centerline{\textit{Katinka Muri Krahn}}



