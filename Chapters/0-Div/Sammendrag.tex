\newpage
\addcontentsline{toc}{chapter}{Sammendrag}
\section*{Sammendrag}
Grunnet per- og polyfluoralkylsubstansers (PFAS) betydelige persistens, mobilitet og toksisitet, er teknikker for remediering av jord og vann forurenset med PFAS høyst nødvendig. Potensialet for å bruke ulike lettforurensede organiske avfallsmaterialer til produksjon av biokullsorbenter for PFAS har vakt stor interesse blant forskere. På grunn av den sterke sorpsjonskapasiteten til ulike kulltyper, har biokull et betydelig potensial for å remediere PFAS-forurensning i jord og vann. Det representerer også en av de mest lovende teknologiene for karbonbinding.

Denne studien forsøkte å teste om biokull produsert fra kloakkslam kan fungere like effektivt til sorpsjon av PFAS som biokull produsert fra et rent, trebasert materiale (CWC), og aktivert karbon som har blitt testet i tidligere studier. Avløpsslammene som ble testet var Ullensaker-slam (ULS) som inneholder avløpsvann anriket med PFAS, og slam fra biorest (DSL), fraksjonen som gjenstår fra anaerob nedbryting av organisk avfall. CWC ble brukt til å sammenligne den relative sorpsjonsstyrken til et rent tresubstrat og den til kloakkslam. En serie med batch-tester ble utført for hvert biokull. Disse ble tilsatt seks perfluorerte karboksylsyrer (PFCA) med 5 til 10 perfluorerte karboner ved økende konsentrasjoner. "Attenuering" er den sorpsjonsdempende effekten som oppstår i mer komplekse systemer der naturlig organisk materiale som finnes i jord kan blokkere porene til biokull, og hvor andre forbindelser konkurrerer om et begrenset antall sorpsjonsseter. For å måle attenuering ble både forsøk med enkeltforbindelser og en cocktail av forskjellige PFCA tilsatt sandig jord (1,3\% TOC) med og uten biokull utført. Biokull-vann og jord-biokull-vann-isotermer ble generert fra batch-testene. Ved å bruke porestørrelsesfordelingene bestemt av $\mathrm{CO_2}$- og $\mathrm{N_2}$-sorptometri for henholdsvis porer mellom 0,4-1,5 nm og porer større enn 1,5 nm, ble sammenhengen mellom størrelsen på overflatearealene og porevolumene, og biokullenes evne til sorpsjon av PFCA vurdert mot hverandre. Effektene av karboninnhold, kalsium og jern for PFCA-sorpsjon ble også evaluert.

I alle batch-forsøkene viste avløpsslambiokullene seg å være sterkere sorbenter for PFCA enn CWC-biokull. Forskjellene i sorpsjon mellom biokull fra CWC, DSL og ULS kunne forklares basert på tre hovedfunn: 1) flertallet av porene i CWC var for små ($<$0,6 nm) til å romme PFCA, hvorav de maksimale molekylære dimensjonene varierer fra 0,96-1,54 nm, 2) CWC-biokull hadde et mye lavere porevolum for porestørrelser over 1,5 nm enn DSL- og ULS-biokull, noe som begrenser porenes fyllingskapasitet, og 3) ULS-biokull hadde høyere porevolum, overflateareal og karboninnhold enn DSL-biokull. Freundlich partisjonskoeffisienter ($\log~K_F$) i $\mathrm{(\mu g/kg)/(\mu g/L)^n}$ var høyest for batchtestene med biokull og enkelt-spikede PFCAer. For PFDA var $\log~K_F$s 6,00 $\pm$ 0,04, 5,61 $\pm$ 0,02 og 5,22 $\pm$ 0,07 for henholdsvis ULS-, DSL- og CWC-biokull. Sterkere sorpsjon var positivt korrelert med økende perfluorert kjedelengde. Kjedelendeavhengigheten ga indikasjoner på at hydrofobe interaksjoner sannsynligvis dominerender sorpsjonsmekanismene over elektrostatiske interaksjoner mellom biokull og PFAS. Attenueringsfaktorer var mellom 6-140 i en PFCA-cocktail, 8-138 i jord og en PFCA-cocktail for PFOA, PFNA og PFDA, og 3-10 for PFOA i jord.

En av hovedbegrensningene for denne studien var at konklusjonene som ble trukket er basert på kun tre biokullprøver. For å bekrefte funnene i denne studien vil det derfor være nødvendig med ytterligere forskning basert på testing av et høyere antall biokullprøver. 

Den høye sorpsjonsstyrken til biokull fra kloakkslam funnet i denne studien lover svært godt for videre utvikling av nye, bærekraftige og kostnadseffektive metoder for håndtering av giftig avfall. Å konvertere forurenset avfall som er kostbart å deponere på riktig måte, til kommersielle sorbenter er et viktig bidrag til en sirkulær økonomi, et bidrag som samtidig imøtekommer nåtidens tidskritiske behov for økt karbonbinding.


