The present work evaluated the sorption strength of two sewage sludge biochars pyrolyzed at 700 \textdegree C to six perfluorinated carboxylic acids (PFCA). The sewage sludge feedstocks were Ullensaker sludge (ULS) containing PFAS-enriched wastewater from Oslo Airport, and digested sludge Lindum (DSL), the remaining fraction after anaerobic digestion of sludge biomass into bio-gas. Biochar from clean wood chips (CWC), a highly carbonaceous feedstock, was used to compare the relative sorption abilities between a clean substrate and that from sewage sludge feedstocks. Batch tests were prepared with ULS, DSL, and CWC biochars and different concentrations of spiked PFCAs. Both single-compound experiments and a cocktail of different PFCAs in both soil and with biochar only were tested to measure sorption attenuation, the sorption suppression effect that occurs in more complex systems. Ten-point biochar-water sorption isotherms and six-point soil-biochar-water isotherms using a sandy soil (1.3\% TOC) were generated. Pore size distribution was conducted for surface area and pore volume by research partners for the sewage sludge biochars that were determined by $\mathrm{CO_2}$ sorptometry for small pores (0.3-1.5 $nm$) and $\mathrm{N_2}$ sorptometry for large pores ($>1.5 nm$).