The present work evaluated the sorption strength of two sewage sludge biochars pyrolyzed at 700 \textdegree C to six perfluorinated carboxylic acids (PFCA). The sewage sludge feedstocks were Ullensaker sludge (ULS) containing PFAS-enriched wastewater from Oslo Airport, and digested sludge Lindum (DSL), the remaining fraction after anaerobic digestion of sludge biomass into bio-gas. Biochar from clean wood chips (CWC), a highly carbonaceous feedstock, was used to compare the relative sorption abilities between a clean substrate and that from sewage sludge feedstocks. Batch tests were prepared with ULS, DSL, and CWC biochars and different concentrations of spiked PFCAs. Both single-compound experiments and a cocktail of different PFCAs in both soil and with biochar only were tested to measure sorption attenuation, the sorption suppression effect that occurs in more complex systems. Ten-point biochar-water sorption isotherms and six-point soil-biochar-water isotherms using a sandy soil (1.3\% TOC) were generated. Pore size distribution was conducted for surface area and pore volume by research partners for the sewage sludge biochars that were determined by $\mathrm{CO_2}$ sorptometry for small pores (0.3-1.5 $nm$) and $\mathrm{N_2}$ sorptometry for large pores ($>1.5 nm$).

The overarching goal of the VOW project and S\o rmo's PhD is to develop "designer" biochar from pyrolysis of organic waste materials that have optimized binding properties as sorbents for various contaminants present in soil and water.

My work has been one part of the work of a larger team of research partners whose aim has been to publish a scientific paper.

The main hypothesis this thesis sought to test was whether biochar from sewage sludge could serve as equally efficient sorbents as biochar from clean wood chips. 

CWC was used to compare the relative sorption strength of a clean substrate and that of sewage sludge feedstocks. 

In all batch tests, the sewage sludge biochars turned out to be stronger sorbents for PFCA than CWC biochar. Differences in sorption between the CWC, DSL, and ULS biochars were explained by three main findings: 1) the majority of pores in CWC were too small ($<$0.6 nm) to accommodate PFCAs, whose maximum molecular dimensions range from 0.96-1.54 nm, 2) CWC had a much lower pore volume for pore sizes above 1.5 nm than DSL and ULS, limiting the pores' filling capacity, and 3) ULS biochar had higher pore volume, surface area, and carbon-content than DSL biochar. Freundlich partition coefficients ($\log~K_F$) in $\mathrm{(\mu g/kg)/(\mu g/L)^n}$ were highest for the batch tests with biochar and singly spiked PFCAs. For PFDA, $\log~K_F$s were 6.00 $\pm$ 0.04, 5.61 $\pm$ 0.02, and 5.22 $\pm$ 0.07 for ULS, DSL, and CWC biochars respectively. Sorption was positively correlated with increasing perfluorinated chain length. Chain length dependency on sorption indicated that hydrophobic interactions are likely the dominant sorption mechanisms over electrostatic interactions between biochar and PFAS. Sorption was attenuated by factors of 6-140 in the presence of a PFCA cocktail, 8-138 in the presence of soil and a PFCA cocktail for PFOA, PFNA, and PFDA, and 3-10 for PFOA in the presence of soil. 

Due to the considerable persistence, mobility, and toxicity of per- and polyfluoroalkyl substances
(PFAS), techniques for the remediation of soil and water contaminated with PFAS are urgently
needed. The potential for using various lightly contaminated organic waste materials in the
production of biochar sorbents for PFAS has sparked great interest among researchers. Due to
its strong sorption capacity, biochar has significant potential to remediate PFAS-contamination
in soil and water. It also represents one of the most promising technologies for carbon seques-
tration