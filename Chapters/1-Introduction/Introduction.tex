%\chapter{Introduction}
\chapter{Introduction}\label{chap:intro}

Transport of contaminants.

Sewage sludge contains environmental pollutants and there is much concern recently related to spread of environmental pollutants through sewage sludge. If the findings in this study can be used to may provide promising outlooks for limiting spread of pollutants to the environment and rather valorizing the waste to sequester carbon and remediate contaminated sites.

Per- and polyfluoroalkyl substances (PFAS) used in aqueous film forming foam (AFFF) contaminates soil, groundwater, surface water and drinking water. Firefighting training site at Gardermoen, Norway used these substances since 1989. Been banned with new guidelines and regulations but there is still much left in the unsaturated zone heavily contaminated with PFAS (what are the concentrations of the various PFASs?) NGI document? 

POPS, PAHs, PBT properties, Stockholm Convention (2001)

PFAS uses

Human exposure: coating on clothing, drinking water, kitchen equipment, food packaging, herbicides, accumulation in food products, 

Environmental Quality Standards (EQS) - Miljødirektoratet \citep{EC2020PFAS}. \citep{MD2016workshop} \citep{MD2020EQS}

What is new: sludge char is being sold as soil amendment but cannot sell for as much money as sludge char as sorbent! Sorbents are more expensive. The results from this study may provide important economic advantages for Scanship and Lindum that can sell sludge char as sorbents for organic pollutants. The aim of the laboratory investigations was to characterize the PFCA sorption ability of biochars produced from waste materials to determine if such biochars can introduced to the market as sorbents. 

Important for wastewater treatment because. The use of biochar as sorbent for organic and inorganic contaminants has received increased attention in recent years. Biochar for soil remediation is becoming a particularly attractive concept due to a combined effect of serving as a similar effective sorbent as activated carbon combined with the major potential for carbon sequestration and eliminates the need for energy-intensive sewage sludge treatment. Biochar has received increased scientific and public attention in recent years for its sorptive abilities of various contaminants (both heavy metals and organic compounds) which are similar to activated carbon combined with the major potential for carbon sequestration. Commercial production is growing internationally and is now widely used for soil remediation \citep{Ahmad2014}. Low cost

%%%%%%%%%%%%%%%%%%%%%%%%%%%%%%%%%%%%%%%%%%%%%%%%%%%%%%%%%%%%%
\section{Background}\label{sec:Background}
 

%%%%%%%%%%%%%%%%%%%%%%%%%%%%%%%%%%%%%%%%%%%%%%%%%%%%%%%%%%%%%
\section{Objectives}
The overall goal of the study is to test if sewage sludge biochar can be used as an effective sorbent for PFAS in contaminated water and soils. Specifically, the objectives of this thesis are:
\begin{enumerate}
    \item {Provide a detailed mechanistic understanding of the sorption mechanisms of perfluorinated carboxylic acids of increasing chain lengths to waste biochar}
    \item{Provide a detailed mechanistic understanding of the sorption mechanisms of perfluorinated carboxylic acids of increasing chain lengths to a sandy, low-TOC soil amended with waste biochar}
    \item{Evaluate how a cocktail of PFCAs sorb to waste biochar compared with single PFCA compounds}
    \item{Evaluate if the waste biochars used in this study can be used to reduce naturally PFAS-contaminated to environmental quality standards set by the Norwegian government.}
\end{enumerate}

\subsection{Hypothesis}
The main hypothesis of this thesis was:
\textit{Sewage sludge biochar can serve as an effective sorbent for PFAS in contaminated water and soils.}

Three sub-hypotheses were tested:
\begin{itemize}
    \item PFAS sorption to sewage sludge biochar increases with chain length.
    \item When competition between PFASs in a cocktail is present, competition for sorption sites on the biochar occurs.
    \item When soil is amended with sewage sludge biochar, the sorption coefficient, K$_d$, will decrease somewhat compared to K$_d$ for sludge char alone due to attenuation by organic matter in the system. 
\end{itemize}
%%%%%%%%%%%%%%%%%%%%%%%%%%%%%%%%%%%%%%%%%%%%%%%%%%%%%%%%%%%%%
\section{Scope} 
What research has already been done on this subject? How does the work of this thesis stand out?
%%%%%%%%%%%%%%%%%%%%%%%%%%%%%%%%%%%%%%%%%%%%%%%%%%%%%%%%%%%%%
\section{Approach}
A literature study is aimed at...
Laboratory investigations are to be conducted to find... 

%%%%%%%%%%%%%%%%%%%%%%%%%%%%%%%%%%%%%%%%%%%%%%%%%%%%%%%%%%%%%
%Provides a list of the chapters and a short, one line description
\section{Structure of the document}
In \cref{chap:LitStudy} literature study, theory

In \cref{chap:MatlsMethds} the materials and methods are described and outlined.

In \cref{chap:Results&Disc} the results from laboratory testing are reported and discussed.

In \cref{chap:Conclusion} a conclusion of the work done in both lab and field is made.

In \cref{chap:furtherwork} recommendations for further work are proposed.