\begin{figure}
    \centering
    \begin{tabular}{c}
    \chemname{\chemfig[atom style={scale=0.8}]{O=[:90](-[:30,,,1]OH)-[:150](-[:67.5]F)(-[:112.5]F)-[:210](-[:247.5]F)(-[:292.5]F)-[:150](-[:67.5]F)(-[:112.5]F)-[:210](-[:247.5]F)(-[:292.5]F)-[:150](-[:67.5]F)(-[:112.5]F)-[:210](-[:247.5]F)(-[:292.5]F)-[:150](-[:90]F)(-[:150]F)-[:210]F}}{perfluorooctanoic acid (PFOA)} \\
\\
    \chemname{\chemfig[atom style={scale=0.8}]{F-[:292.5](-[:67.5]F)(-[:330]S(=[:60]O)(-[:330,,,1]OH)=[:240]O)-[:210](-[:292.5]F)(-[:247.5]F)-[:150](-[:112.5]F)(-[:67.5]F)-[:210](-[:292.5]F)(-[:247.5]F)-[:150](-[:112.5]F)(-[:67.5]F)-[:210](-[:292.5]F)(-[:247.5]F)-[:150](-[:112.5]F)(-[:67.5]F)-[:210](-[:270]F)(-[:150]F)-[:210]F}}{perfluorooctanesufonic acid (PFOS)} \\
\\
    \chemname{\chemfig[atom style={scale=0.8}]{O=[:60]S(=[:60]O)(-[:330,,,1]OH)-[:150](-[:67.5]H)(-[:112.5]H)-[:210](-[:247.5]H)(-[:292.5]H)-[:150](-[:67.5]F)(-[:112.5]F)-[:210](-[:247.5]F)(-[:292.5]F)-[:150](-[:67.5]F)(-[:112.5]F)-[:210](-[:247.5]F)(-[:292.5]F)-[:150](-[:67.5]F)(-[:112.5]F)-[:210](-[:270]F)(-[:210]F)-[:150]F}}{6:2 fluorotelomer sulfonic acid (6:2 FTSA)} \\
\\
   %\chemname{
            %\chemleft
              %  \chemfig{
                   % -C
                      %  ( -[2]F)
                       % ( -[6]F)
                   % -C
                       % ( -[2]F)
                       % ( -[6]F)
                   % -[0]
               % }
           % \chemright]$_n$
       % }{polytetrafluoroethylene (PTFE)} \\
    \end{tabular}
    \caption{The most commonly studied and used PFAS compounds, PFOA, PFOS monomers and the polymer polytetrafluoroethylene used in non-stick, oil- and water-repellent products like Teflon and Gore-Tex.}
    \label{fig:PFASstruct}
\end{figure}

\begin{table}
\centering
\caption{Physicochemical properties of relevant perfluorinated carboxylic acids modeled using COSMOterm by \cite{wang2011SI}.}
\label{tab:COSMOtherm}
\begin{threeparttable}
\begin{tabular}{lccccc}
\toprule
\multicolumn{1}{c}{Acronym} & log $K_{AW}$* & log $K_{OW}$\textsuperscript{\dag} & log $K_{OA}$\textsuperscript{\ddag} & log $P_L$\textsuperscript{\S} & log $S_W$\textsuperscript{\P} \\ \midrule
PFPeA & -2.90 & 3.43 & 6.33 & 3.13 & -0.37 \\
PFHxA & -2.58 & 4.06 & 6.63 & 2.66 & -1.16 \\
PFHpA & -2.25 & 4.67 & 6.92 & 2.20 & -1.94 \\
PFOA & -1.93 & 5.30 & 7.23 & 1.73 & -2.73 \\
PFNA & -1.58 & 5.92 & 7.50 & 1.27 & -3.55 \\
PFDA & -1.27 & 6.50 & 7.77 & 0.82 & -4.31 \\ \bottomrule
\end{tabular}
\begin{tablenotes}
\item * air-water partition coefficient
\item \textsuperscript{\dag} octanol-water partition coefficient
\item \textsuperscript{\ddag} octanol-air partition coefficient
\item \textsuperscript{\S} liquid vapor pressure in Pa
\item \textsuperscript{\P} solubility in water in mol L\textsuperscript{-1}
\end{tablenotes}
\end{threeparttable}
\end{table}

\cref{tab:COSMOtherm} gives an overview of different physicochemical properties of the PFCAs used for the research of this thesis modeled using COSMOterm by \cite{wang2011physchem}

By 2023, the use of C9-C14 perfluorinated carboxylic acids has been ruled to be phased out \citep{ECHA2020}. The Norwegian Environmental Protection Agency (Milj\o direktoratet) has set a goal to eliminate all release of PFAS to the environment within 2027 . that have been used for decades which that have been found to

convention." \citep{Schlabach2017}

"Today, PFOA is selected as a candidate for the “substances of very high concern” by The European Chemicals Agency ("Candidate List of substances of very high concern for Authorization," 2018). 
During manufacture of use of these products, some PFAS

PFOA can get into the soil, water, and air. It can stay in the environment and in your body for a long time" Products using PFAS were originally manufactured by emulsion polymerization of PFOA into polytetrafluoroethylene (PTFE), a polymer commonly known under the trademark, Teflon\textsuperscript{\texttrademark} \citep{Lehmler2005}. PTFE and its monomers like PFOA and PFOS make highly chemically inert compounds. However, traces of monomer residues from the polymer manufacture are carried with the products and as effluent from industrial sites. The PFOA and PFOS conjugates are more mobile than the monomer product and ends up in the environment and humans. 

Emerging contaminants definition \citep{Li2019} and by \citep{EPA2014}: an "emerging contaminant" is "a chemical or material that is characterized by a perceived, potential, or real threat to human health or the environment or by a lack of published health standards. A contaminant may also be "emerging" because a new source or a new pathway to humans has been discovered or a new detection method or treatment technology has been developed"

"Reductive defluorination in anaerobic environments has been proposed as a pathway for environmental degradation; however, conclusive proof of this occurring in the environment has yet to be documented" \citep{ArpNGI}. "Chemical inert:nonflammable, not readily degraded by strong acids, alkalis, or oxidizing agents, not subject to photolysis, hydrolysis or biodegrade  = practically non-biodegradable and persistent in the environment" \citep{Lau2007,EPA2014}. 

anaerobic digestion: AD transforms sludge organic solids to biogas (methane, ammonia, H$_2$S, CO$_2$). Biogas comprises of 60-70\% methane, and 30-40\% carbon dioxide, trace amounts of other gases.

At high enough concentrations, the molecules arrange in micelles, and this concentration is referred to the critical micelle concentration (CMC) at which the physicochemical behavior changes dramatically \citep{bhhatarai2011,Goss2009comment}.  

The large long-range transport potential (LRTP) owes to the chemical inertness and charge. The formal charge of the PFAS polar heads make these compounds much more mobile that a C-F chain alone because the molecule as a whole becomes more polar, which would have comparable properties to long-chain oils which would stick to soil and particles due to high hydrophobicity.

 In contrast to water solubility, sorption to soil and sediment plays an important role for retention and accumulation of PFAS in the environment \citep{li2018,LehmannAndJoseph2015,Cornelissen2005}.
 
   \citep{MD2016workshop} \citep{MD2020EQS}
especially related to persistency and long-range transport potential (LRTP) \citep{MD2020EQS} because a wide range of PFAS substances have been detected in Arctic biota such as fish, polar bears and mink \citep{Schlabach2017}.

Partitioning behavior \citep{wang2011physchem}: log $K_{OW}$, log $S_W$, log $S_O$ These parameters are again determined by compound polarity and its electronic extreme, ionic charge which make compounds more soluble than non-polar compounds \citep{Reemtsma2016}.

PFAS pollution of groundwater and soil is particularly prominent around several Norwegian airports where old firefighting training facilities have used fire-fighting foam containing PFAS for decades .

Today, the USEPA advised drinking water concentration limit of PFOA is set to 0.07 \textmu g L\textsuperscript{-1} \citep{us2016drinking}. Environmental Quality Standards (EQS) - Miljødirektoratet \citep{EC2020PFAS}.

Char: PCM residue from natural fires
Charcoal: "PCM produced from pyrolysis of animal or vegetable matter in kilns for use in cooking or heating, including industrial applications such as smelting"

Ash content: gravimetrically by heating under air at 750 \textdegree C until a constant weight was obtained
Surface area and pore size distribution: measured by both BET-N$_2$ adsorption (used for pores >15 Å) and CO2 sorption (used for pores 2.5-15 Å) using a Quantachrome Autosorb I
Pore volume and pore size distributions: estimated from the adsorption isotherms by Density Functional Theory (DFT).

lists in the introduction many ways biochar functions as a soil amendment): soil improvement, waste management, climate change mitigation, and energy production.
 
Sewage is nutrient rich (P, K, N) and can be used as fertilizer
 
Biochar greatly benefits from being activated because it greatly increases microporosity and surface area
  
The ideal situation would be to use biochar from sewage sludge so that agriculture can benefit from their nutrient, soil conditioning, water retention properties.

Different sorption models are used to express how environmental contaminants are removed from water by sorption to solid phases

Hydrophobicity increases with chain length, so the short-chain PFCAs are typically the most mobile. Most research in the field has been conducted on PFOS and PFOA with a chain length of 8 carbons. But as more and more of the long-chain PFAS get restricted and are substituted with short-chain alternatives, the alternative effects and transport routes must be investigated. 


Hydrophobic effect = "cavity formation energy, results from the sum of forces that limit the solubility of molecules in water. Its underlying cause is the disruption of the cohesive energy of water due to the greater ordering of water molecules and the lower number of water-water H-bonds in the hydration shell of the non-polar moiety compared to the bulk water phase" \citep{sigmund2022sorption}.

several interactions: 
\begin{itemize}
    \item \textpi-\textpi electron-donor-acceptor (EDA): environmental contaminants are \textpi-electron acceptors and biochar is \textpi-electron donor. Is particularly strong for sorption of planar aromatic compounds to the planar graphene surface of biochar. 
    \item electrostatic
    \item pore-filling
    \item hydrophobic
    \item hydrogen bonding
    \item functional groups
    \item cation exchange and bridging interactions
\end{itemize} 

\citep{sigmund2022sorption}, figure key drivers and interactions for sorption of charged organic compounds
\begin{itemize}
    \item hydrophobic effect
    \item pi-pi electron donor-acceptor interaction
    \item H-bond
    \item charge assisted H-bond
    \item electrostatic repulsion
    \item cation bridging
    \item electrostatic attraction
    \item anion-pi bond
    \item cation-pi bond
\end{itemize}
