\chapter{Introduction}\label{chap:intro}

\section{PFAS}
\textit{Per}- and polyfluorinated alkyl substances (PFAS) are a large group of synthetic compounds that are used in numerous industrial and consumer products. As a result of decades of use, PFAS are ubiquitous in soils, groundwater, and surface water \citep{rankin2016north}. PFAS are both oil and water repellent, which makes them ideal as foaming agents and flame retardants, as a coating for the famous waterproof Gore-Tex\textsuperscript{\textregistered} textiles and non-stick, frictionless Teflon\textsuperscript{\texttrademark} cookware \citep{du2014adsorption}. Despite having many desirable properties, widespread production and distribution of PFAS with waterways has led to accumulation in soil, crops, wildlife, and higher trophic levels, including humans \citep{bhhatarai2011,Lau2007}. The synthetic structure of PFAS makes these chemicals resistant to natural degradation \citep{krafft2015per}, and although the mechanisms of PFAS toxicity are not well understood, they are also suspected to be bioaccumulative and toxic \citep{ding2013physicochemical,Lau2007}. Today, PFAS are recognized as emerging persistent organic pollutants (POPs) \citep{ECHA2020}, and legislation has been introduced to regulate the manufacture, sale, and use of several PFASs \citep{EPA2014,EC2020PFAS,goldenman2017study}. Two of the most widely used and distributed compounds, PFOA and PFOS, are listed under the Stockholm Convention as POPs. Considerations to add other compounds to this convention are ongoing \citep{EC2020PFAS}. An important example is efforts by the German Environment Agency to identify a new class of compounds called PMT (persistent, mobile, and toxic) and vPvM (very persistent, very mobile) under REACH, the EU chemicals regulation (Registration, Evaluation, Authorization and Restriction of Chemicals) and ECHA (the European Chemicals Agency) \citep{hale2020persistent}. PMT/vPvM substances have properties that are not covered by the Stockholm Convention, but is an equal great threat to the environment. 

A great deal of research has been done to develop ways of effectively remediating PFAS-contaminated soil and water. One of these is the overarching focus of this thesis. Research conducted for this thesis has been done in collaboration with the Norwegian Geotechnical Institute as part of the joint-industry sustainability (BIA-X) project "Valorization of Organic Waste" (VOW) (NFR 299070) funded by the Norwegian Research council. The aim of this thesis has been to study whether biochar from various waste materials can be used as sorbents in remediation of PFAS-contaminated soil and water.

\subsection{Physicochemical properties}\label{sec:physchem} 
PFAS are synthetic organic compounds that consist of a polar head, most commonly a carboxyl or sulfate functional group, and a non-polar chain of alkyl moieties that are either fully substituted (per-), or partly substituted (poly-), with fluorine \citep{wang2011physchem}. PFASs are divided into compound classes based on functional group and degree of fluorination, of which perfluorinated carboxylic acids (PFCAs) and perfluorinated sulfonic acids (PFSAs) are the most common. Since fluorine is the most electronegative atom identified, the C-F covalent bonds make one of the strongest known bonds in organic chemistry (BDE=485 kJ mol\textsuperscript{-1}) \citep{Lau2007}. The nature of these bonds become significant for two main reasons: 1) C-F bonds are not found naturally in the environment, and this means that no natural enzymes can degrade them. This accounts for why PFAS is persistent in the environment \citep{hale2020persistent,krafft2015per}. 2) The chain of repeating CF\textsubscript{2} units consists of highly polar bonds with a symmetric structure. This makes the  entire surface area of the PFAS tail carry a net negative charge. This in turn minimizes the ability of PFAS to undergo neither van der Waals interactions with other molecules nor hydrogen bonds due to the lack of a positive dipole \citep{Arp2006}. The combined properties of low van der Waals and non-polarity make the tail neither hydrophobic nor lipophobic. The result is a new compound class that is "everything-phobic". This everythingphobic-ness causes PFAS to behave uniquely in the environment by being more mobile than other organic pollutants \citep{hale2020persistent}. A more detailed discussion on the mobility of PFAS is given in the following section.  

The behavior of PFAS is often discussed interchangeably with hydrophobic compounds that have strong affinities to hydrophobic surfaces. More correctly, sorption of PFAS should be discussed in terms of a shear pushing of PFAS by water molecules onto hydrophobic surfaces \citep{Arp2006}. The cavity formation energy barrier rises for every $CF_2$ moiety present, making dissolution increasingly difficult at longer chain lengths \citep{bhhatarai2011,Arp2006}. This is the primary mechanism that reduces the mobility of PFAS leading to accumulation in soils and biota. The effect of chain length on partitioning between dissolved and solid phases is one of the main topics that will be discussed in further detail throughout this thesis.

Sorption is a collective term used to describe removal of molecules from the water phase by \textit{ab}sorption (bonding within a material) and \textit{ad}sorption (bonding on the surface of a solid). The role van der Waals forces play, as well as other mechanisms of attraction and repulsion for PFAS interaction with sorbents, will be discussed in a later section on sorption mechanisms \cref{sec:mechanisms}. 

In contrast to the everything-phobic chain, the hydrophilic head of PFAS can hydrogen bond to other polar compounds such as water, and interact electrostatically with positively charged species \citep{sigmund2022sorption}. Since the head groups of PFAS are acids, knowledge of its protonation state is key to further understand solubility and volatility, which in turn affect the mobility and long range transport potential of these chemicals. Determining the acid dissociation constants (\(pK_a\)) of PFAS has been the subject of much debate in the scientific community \citep{Goss2009comment}. This debate is linked to the complexity of how PFAS behaves at different concentrations, and on the water surface. However, most researchers agree that \(pK_a\) values are below 1 and decrease with increasing chain length \citep{wang2011physchem,Reemtsma2016}. Regardless, PFAS are expected to be negatively charged at environmentally relevant pH levels, enhancing its ability to dissolve in water by charge-assisted H-bonds \citep{Reemtsma2016}. For example, PFOA has a water solubility of 3.4 g L\textsuperscript{-1} \citep{PFOA}, whereas its structural analog, perfluorooctane (PFO) \citep{PFO}, which has no polar functional group, is practically insoluble (1 mg/L). 

\subsection{Mobility and accumulation of PFAS in the environment}
The combination of an everything-phobic tail and an ionizable hydrophilic head makes mobility of PFAS in the environment more complex than legacy POPs \citep{cabrerizo2018legacy,Arp2006}. Since PFASs are not very volatile---low vapor pressure and air-water partition coefficient ($K_{AW}$)---aqueous solubility and uptake in migrating biota play the primary roles for the mobility of PFAS in the environment \citep{Arp2006}. \cite{Schlabach2017} reports that a diverse suite of PFASs are present in Arctic biota such as fish, polar bears and mink. Sorption mechanisms are also slightly different for PFAS compared to hydrophobic and lipophilic legacy POPs like polychlorinated biphenyls (PCBs), hexacholorocyclohexanes (HCHs), and polycyclic aromatic hydrocarbons (PAHs), which have higher volatility, lower solubility and higher lipohilicity \citep{cabrerizo2018legacy,Cornelissen2005,li2018}. The latter property is often expressed as the octanol water partition coefficient (\(K_{OW} = C_{o}/C_w\)). This coefficient is used to represent partitioning between a hydrophobic phase represented by octanol (numerator) and the aqueous phase (denominator) \citep{Reemtsma2016}. The lower \(K_{OW}\) of PFAS than hydrocarbon surfactants can be used to explain why PFAS leaches from soils to groundwater more readily than, for example, PAHs \citep{Cornelissen2005,du2014adsorption}). A more environmentally relevant parameter than \(K_{OW}\) is soil organic carbon (SOC), represented by the SOC-water partition coefficient (\(K_{OC}\)). \(K_{OC}\) is an important driver of the sorption of PFAS in soils and sediments \citep{Sormo2021}. This coefficient also represents the higher tendency of PFAS to leach from soils to groundwater, as compared to compounds with higher $K_{OC}$ like PAHs and PCBs that sorb stronger to organic matter \citep{Cornelissen2005}. 

\subsection{PFAS in industrial runoff and wastewater}
The sources of PFAS contamination are typically local industrial point sources such as paper mills \citep{lee2020legacy,langberg2021paper}, leaching from landfills \citep{masoner2020landfill}, firefighting training facilities \citep{filipovic2015historical}, and discharge from fluorochemical plants \citep{gebbink2017presence} into the wastewater. This puts wastewater treatment plants (WWTPs) under additional pressure  to adequately treat wastewater so as to avoid further release of PFAS chemicals into the environment  \citep{Morin2017flameWaste}. By restricting many long-chain PFASs, new challenges related to controlling the spread and effects of increased use of short-chain replacements have emerged \citep{knutsen2019leachate}. Short-chain PFAS have higher mobility and tend to slip through existing water treatment processes, thereby contaminating food and drinking water \citep{hale2020persistent,brendel2018short}. Although short-chain PFAS are less bioaccumulative, persistence and toxicity are expected to be equivalent to the substances they replace \citep{EC2020PFAS}. 

%%%%%%%%%%%%%%%%%%%%%%%%%%%%%%%%%%%%%%%%%%%%%%%%%%%%%%%%%%%%%%%%%%%%%%%%%%%%%%%%%%%%%%%%%%%%%%%%%%%%%%%%%%%%%%%%%%%%%%%%%%%%%%%%%%%%%%%%%%%%%%%%%%%%%%%%%%%%%%%%%%%%%%%%%%%%%%%%%%%%%%%%%%%%%%%%%%%%%%

\section{Biochar---from traditional soil amendment to sorbents for emerging contaminants}
Biochar is a common term for the carbon-enriched product produced from pyrolysis of biomass \citep{LehmannAndJoseph2015}. Common feedstocks used to produce biochar are various crop residue, poultry litter, wood shavings, and grain straws \citep{Ahmad2014}. Pyrolysis is a thermal treatment method that burns organic matter in the absence of oxygen, a process which forms fused aromatic ring structures that have high porosity and surface area and residual functional groups \citep{LehmannAndJoseph2015}. Biochar improves soil health by increasing water retention and carbon content, and securing a more steady nutrient release \citep{das2020application}. In addition, the ash co-products enrich biochar with macro- and micro nutrients that are also beneficial in agriculture. The use of biochar as a soil amendment dates back 2,500 years to the pre-Columbian Amazonian peoples \citep{Tindall2017}. They used slash-and-char techniques to produce biochar which they spread over their fields. The result was \textit{terra preta}, the most fertile soil known \citep{Ahmad2014}. Today, biochar has numerous multidisciplinary applications, such as soil fertility improvement, waste recycling, carbon sequestration, and sorbents for remediation of contaminated sites \citep{beesley2011review}.

\subsection{Biochar as sorbents}
The application of biochar as a sorbent for soil and water remediation, is relatively new \citep{beesley2011review}. Historically, activated carbon (AC), has been used for soil remediation exclusively \citep{hagemann2018activated}. The term "activation" is used to describe the process of thermal and chemical treatment of a carbonaceous material by steam ($\mathrm{H_2O}$) or carbon dioxide ($\mathrm{CO_2}$) that expands its surface area by creating new micropores ($<$2nm), and simultaneously oxidizing surface functional groups to create a smoother, aromatic surface \citep{sajjadi2019comprehensive}. This process improves the sorptive capacity of the biochar, making it an ideal sorbent for organic pollutants \citep{Ahmad2014}. Current challenges connected to production of AC include high operating costs and energy inputs, the use of non-renewable fossil fuel coal as raw material \citep{Li2019}, and the fact that the activation process is associated with higher carbon losses and lower yields, resulting in increased greenhouse gas emissions \citep{Sormo2021}. Biochar is often termed a universal sorbent because its surface will both have regions of hydrophilic or charged functional groups that are ideal for electrostatically binding cationic organic and inorganic species such as heavy metals \citep{silvani2019can,zhang2013sorption}, and other regions of more condensed, aromatic surfaces that are better suited to sorb hydrophobic molecules \citep{Cornelissen2005}.

Researchers have had several successful attempts at using activation technology on biochar instead of fossil coal as sorbents for organic contaminants \citep{Sormo2021}. Commercial production is growing internationally and biochar is now widely used for soil remediation \citep{hagemann2018activated}. In recent years, there has been much interest in studying whether also non-activated biochars can be effective sorbents for use in contaminant remediation and water purification \citep{hagemann2018activated}. The main advantage of using biochar instead of AC is the potential for carbon sequestration and that no chemicals are added during production \citep{hagemann2018activated}. If engineered correctly, biochar is expected to be at least as effective as AC in sorption of organic pollutants \citep{Sormo2021}. Important aspects have been to study how feedstock, pyrolysis temperature and residence time affect biochar properties \citep{Hale2016}. Due to more heterogeneous feedstocks, biochar is not fully carbonized and need to be pyrolyzed at higher temperatures to have a sufficient surface area and porosity for sorption \citep{beesley2011review}. By adjusting pyrolysis temperature, biochar can be tailored to match the physicochemical properties of the contaminants of interest \citep{Hale2016}. The oxygen to carbon (O/C) ratio is used as proxy for polarity and hydrophobicity of a sorbent's surface, where a high ratio is indicative of a more reduced surface with oxygen-containing functional groups. Likewise, the hydrogen/carbon (H/C) ratio is used as proxy for aromaticity, where a lower ratio is indicative of a higher degree of fused aromatic ring structures that form a more porous biochar material \citep{Ahmad2014}. The O/C and H/C ratios have been found to decrease with increasing pyrolysis temperature \citep{Hale2016}. Therefore, biochar produced at high temperatures (700-900 \textdegree C) are the most suitable for sorbing organic contaminants \citep{Figueiredo2018}. 

One of the main challenges proposed by the application of non-activated biochar in natural systems is its lower porosity than AC \citep{leng2021overview}. High porosity becomes particularly important when carbonaceous sorbents are applied to organic matter (OM)-rich soils because small pores are vulnerable to pore blockage by large organic molecules \citep{Sorengard2019}. However, application of biochar to soil with low OM content has proved to be equally effective as the use of non-activated biochar \citep{Alhashimi2017}. This way, biochar provides a contribution to reduce the demand for AC in remediation, which in the future may only be limited to OM-rich soils and ultra-fine water cleansing. The latest research on biochar has been to study the potential for producing biochar sorbents from various waste materials \citep{Cornelissen2011Capping, Kupryianchyk2016b, Sormo2021, zhou2010sorption}, with an end goal to end up with a net negative GWP and energy use and a better waste handling alternative to landfilling or incineration \citep{Alhashimi2017}. Ongoing research, including the present work, investigates the effectiveness of non-activated biochar as sorbents for PFAS. 

\subsection{Sorption mechanisms}\label{sec:mechanisms}
Sorption is a function of several factors such as biochar morphology, contaminant concentration, competing contaminants, sorbate physicochemical properties, and molecular size \citep{Li2019,du2014adsorption}. Sorption to porous carbonaceous materials occur via chemisorption (molecular interactions of attraction and repulsion) and physisorption (encapsulation of the adsorbent in the maze of pores) \citep{Li2019}. It has been postulated that sorption of PFAS occurs primarily via direct (specific) polar interactions, hydrophobic (non-specific) interactions and ion exchange mechanisms \citep{Hale2017fire,yu2009sorption}. Polar interactions include H-bond and charge-assisted H-bond interactions, non-specific interactions refer mainly to the hydrophobic effect, also referred to as cavity formation energy, which results from the sum of forces that limit solubility of large, non-polar molecules in water \citep{Arp2006,sigmund2022sorption}. Electrostatic interactions occur both as attraction and repulsion, and cation bridging is one specific mechanism of electrostatic attraction. Determination of sorption isotherms is a central part to understanding the interactions between biochar and PFAS and sorption capacity \citep{yu2009sorption,Li2019}.

%%%%%%%%%%%%%%%%%%%%%%%%%%%%%%%%%%%%%%%%%%%%%%%%%%%%%%%%%%%%%%%%%%%%%%%%%%%%%%%%%%%%%%%%%%%%%%%%%%%%%%%%%%%%%%%%%%%%%%%%%%%%%%%%%%%%%%%%%%%%%%%%%%%%%%%%%%%%%%%%%%%%%%%%%%%%%%%%%%%%%%%%%%%%%%%%%%%%%%%%%%%%%%%%%%%%%%%%%%%%%%%%%%%%%%%%%%%%%%%%%%%%%%%%%%%%%%%%%%%%%%%%%%%%%%%%%%%%%%%%%%%%%%%%%%%%%%%%%%%%%%%%%%%%%%%%%%%%%%%%%%%%%%%%%%%%%%%%%%%%%%%%%%%%%%%%%%%%%%%%

\section{Valorization of sewage sludge---no longer a waste?}
Recycling and valorization of waste have become key areas of interest in research and development aimed at achieving a transition to a more circular economy \citep{Ahmad2014}. In recent years, the European Union has invested substantial funding in projects that work on developing remedial techniques for the treatment of soil and water contaminated with PFAS (see \cref{sec:SDGs}) \citep{EC2020PFAS,ECHA2020}. One of the most promising techniques so far is treatment \textit{in situ} by applying sorbents that immobilize PFAS from the bioavailable aqueous phase, by strong sorption to biochar \citep{Ahmad2014,Sormo2021,Kupryianchyk2016b}. 

\subsection{Potential for sewage sludge biochar as sorbents for PFAS}
Biosolids, the residual semi-solid material waste left over from wastewater treatment, are produced in large quantities and are expensive to dispose of because they are heavily polluted \citep{Raheem2018}. The high moisture and ash contents, heavy metals and cocktail of organic contaminants contained in sewage sludge make treatment and disposal difficult \citep{Li2019}. Therefore, incineration or landfilling are often resorted to, which release significant greenhouse gases (GHG), fly ash and PAHs \citep{huang2022comparative}, as well as heightened risks for leaching into soils and groundwater \citep{propp2021organic}. 

WWTPs have started to look at possible ways to handle biosolids and raw sewage sludge in a more sustainable and cost-effective way \citep{Raheem2018}. One is to generate energy from waste. Anaerobic digestion (AD) (\cref{eq:AD}) of sludge with the help of microorganisms yields digestate---the liquid residual fraction from the anaerobic treatment of sewage sludge and other organic wastes---is now being used for the commercial production of bio gas. In some cases, biosolids are applied to agricultural fields as fertilizers \citep{moodie2021legacy}. However, high contents of micro and macro plastics and heavy metals may limit the use of biosolids in agriculture \citep{mohajerani2020microplastics}. The final product that is rich in nutrients (P, K, N) can be further utilized as fertilizer and/or compost. 

Research into the production of biochar that uses digestate and raw sewage sludge as feedstocks, represents yet another attempt to find sustainable ways to valorize organic waste. This concept is particularly attractive because it is considered to be one of the most promising carbon sequestration technologies, simultaneously giving economic value to a problematic waste material \citep{arvaniti2014sorption}. Additionally, increased yields are expected for sewage sludge compared to cleaner, wood-based feedstocks because of the higher content of inorganic ions, observed by higher ash contents \citep{Cantrell2012}. Lower losses of volatile carbon is identified as a benefit resulting from more ash present because the ions raise the bond dissociation energy of organic and inorganic carbon \citep{Cantrell2012,Enders2012}. 

Production of sorbents from sewage sludge is today still in a pioneering phase, from which there are several proposed challenges: 1) Sewage sludge has a lower carbon content and is therefore expected to have low surface area and porosity, and hence, poor sorption strength. Due to the lower porosity, it is expected that uptake may be slower than by AC so that either a higher biochar dose or more frequent biochar filter exchange is needed to achieve desired quality of treated water or soil. 2) Non-activated biochar consists of more oxygen and nitrogen-containing functional groups which makes it more polar and thus attracts charged and polar contaminants to a higher extent. Non-activated biochar contains a higher non-carbonized fraction that interacts with contaminants in a different way than fully condensed aromatic structures. Discussing the relevance/benefits of this more heterogeneous matrix for sorption of PFAS is one of the main focuses of this thesis. 3) Due to the heterogeneity of sewage sludge, biochar will vary in composition and thereby have inconsistent sorption capacities from time to time. 4) Bio oils and syn-gas produced as by-products is expected to contain a concentrated cocktail of pollutants and is a possible source of greenhouse gases, particulate matter and heavy metals. Pollution control measurements was conducted simultaneously by research partners and will provide important information that goes into the overall life cycle assessment (LCA) which is discussed further in \cref{sec:LCA}). 
