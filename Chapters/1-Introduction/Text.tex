\chapter{Introduction}\label{chap:intro}

\section{PFAS}
Per- and polyfluorinated alkyl substances---PFAS---are synthetic, oil- and water-repellent compounds that serve numerous applications in industrial and consumer products \citep{Nicole2013}. These include paint, coatings, firefighting foam, electronics, cosmetics, cookware, and textiles. Despite many desirable properties, the widespread production and use have distributed PFAS with waterways which ultimately accumulates in soil, crops, wildlife, and higher trophic levels including humans \citep{bhhatarai2011}. The persistency, bioaccumulation and toxicity has become a major environmental concern that has led to legislative action on the manufacture, sale, and use of several PFAS compounds \citep{EPA2014,ECHA2020,EC2020PFAS}. Today, PFOA and PFOS are listed under the Stockholm Convention on Persistent Organic Pollutants (POPs), which means that their use are restricted, while assessments of prohibiting more compounds is ongoing \citep{EC2020PFAS}. By 2023, the use of C9-C14 perfluorinated carboxylic acids must be phased out \citep{ECHA2020}.

Sewage sludge contains environmental pollutants and there is much concern recently related to spread of environmental pollutants through sewage sludge. If the findings in this study can be used to may provide promising outlooks for limiting spread of pollutants to the environment and rather valorizing the waste to sequester carbon and remediate contaminated sites.

Per- and polyfluoroalkyl substances (PFAS) used in aqueous film forming foam (AFFF) contaminates soil, groundwater, surface water and drinking water. Firefighting training site at Gardermoen, Norway used these substances since 1989. Been banned with new guidelines and regulations but there is still much left in the unsaturated zone heavily contaminated with PFAS (what are the concentrations of the various PFASs?) NGI document? 

Endocrine disrupting compounds (EDCs)
. Today, the USEPA advised drinking water concentration limit of PFOA is set to 0.07 \textmu g L\textsuperscript{-1} \citep{us2016drinking}. 

Products using PFAS were originally manufactured by emulsion polymerization of PFOA into polytetrafluoroethylene (PTFE), a polymer commonly known under the trademark, Teflon\textsuperscript{\texttrademark} (\cref{fig:PFASstruct}) \citep{Lehmler2005}. PTFE and its monomers like PFOA and PFOS make highly chemically inert compounds. However, traces of monomer residues from the polymer manufacture are carried with the products and as effluent from industrial sites. The PFOA and PFOS conjugates are more mobile than the monomer product and ends up in the environment and humans. PFOS and PFAS are unintended breakdown products during manufacture of AFFF. Production and use of perfluorooctanoate (PFOA) and PFOS and its homologues have been phased out in the western countries following agreements with manufacturers by 2014. "Today, PFOA is selected as a candidate for the “substances of very high concern” by The European Chemicals Agency ("Candidate List of substances of very high concern for Authorization," 2018). There are also planned restrictions under REACH, and global restriction are prepared under the Stockholm.
that have been used for decades which that have been found to have serious environmental and health effects \citep{Lau2007}. During manufacture of use of these products, some PFAS
PFOA a precursor for Teflon, PFOA used in the manufacturing of PTFE, or Teflon since the 1940s \citep{lindstrom2011}. "during production, PFOA can get into the soil, water, and air. It can stay in the environment and in your body for a long time"

Environmental Quality Standards (EQS) - Miljødirektoratet \citep{EC2020PFAS}. \citep{MD2016workshop} \citep{MD2020EQS}

What is new: sorption of PFAS to sludge biochars is not yet studied. sludge char is being sold as soil amendment but cannot sell for as much money as sludge char as sorbent! Sorbents are more expensive. The results from this study may provide important economic advantages for Scanship and Lindum that can sell sludge char as sorbents for organic pollutants. The aim of the laboratory investigations was to characterize the PFCA sorption ability of biochars produced from waste materials to determine if such biochars can introduced to the market as sorbents. 

Important for wastewater treatment because. The use of biochar as sorbent for organic and inorganic contaminants has received increased attention in recent years. Biochar for soil remediation is becoming a particularly attractive concept due to a combined effect of serving as a similar effective sorbent as activated carbon combined with the major potential for carbon sequestration and eliminates the need for energy-intensive sewage sludge treatment. Biochar has received increased scientific and public attention in recent years for its sorptive abilities of various contaminants (both heavy metals and organic compounds) which are similar to activated carbon combined with the major potential for carbon sequestration. Commercial production is growing internationally and is now widely used for soil remediation \citep{Ahmad2014}. Low cost

This study is one of the first to investigate the potential for sewage sludge biochars to work as sorbents for PFAS.

Objectives (from poster):
Compare the relative abilities of sewage sludge biochars and clean wood chips to sorb perfluorinated carboxylic acids (PFCAs)
Identify possible sorption mechanisms of PFCAs for the different biochar feedstocks
Study the effects of increasing perfluorinated carbon chain-length, competing sorbates, and the presence of soil on sorption

\subsubsection{LRTP}
Common for all POPs are that they do not only affect locally but are transported to remote areas of the globe and contaminates the most pristine water and soils left on Earth. The large long-range transport potential (LRTP) owes to the chemical inertness and charge. The formal charge of the PFAS polar heads make these compounds much more mobile that a C-F chain alone because the molecule as a whole becomes more polar, which would have comparable properties to long-chain oils which would stick to soil and particles due to high hydrophobicity. Hydrophobicity increases with chain length, so the short-chain PFCAs are typically the most mobile. Table 11 AMAP report shows that PFPeA to PFDA are present in Arctic biota such as fish, polar bears and mink. PFOS has the highest concentration in Arctic of all PFASs. "PFNA was the most prominent compound of the carboxylic acid group" \citep{Schlabach2017}.

On the Norwegian priority list that has set a goal to eliminate all release of PFAS to the environment within 2027 \citep{MD2016workshop}.
%%%%%%%%%%%%%%%%%%%%%%%%%%%%%%%%%%%%%%%%%%%%%%%%%%%%%%%%%%%%%
\section{Background}\label{sec:Background}
 

%%%%%%%%%%%%%%%%%%%%%%%%%%%%%%%%%%%%%%%%%%%%%%%%%%%%%%%%%%%%%
\section{Objectives}
The overall goal of the study is to test if sewage sludge biochar can be used as effective sorbents for PFAS for remediation of contaminated water and soil. Specifically, the objectives of this thesis are:

\begin{enumerate}
    \item{Compare the relative abilities of sewage sludge biochars and clean wood chips to sorb perfluorinated carboxylic acids (PFCAs)}
    \item{Identify possible sorption mechanisms of PFCAs for the different biochar feedstocks}
    \item{Study the effects of perfluorinated carbon chain-length on sorption }
    \item{Study the attenuation effect of competing sorbates and the presence of soil on sorption to evaluate the effectiveness of sewage sludge biochars as sorbents applied to real-world application settings}
\end{enumerate}

\subsection{Hypothesis}
The main hypothesis of this thesis was:
\textit{Biochar from clean feedstock and waste based feedstock reduce PFAS concentration in water.}

Three sub-hypotheses were tested:
\begin{itemize}
    \item Biochar from clean wood chips is a more effective sorbent than sewage sludge biochars in PFAS contaminated water and soil due to higher carbon fraction.
    \item PFAS sorption increases with chain length.
    \item Sorption of PFAS to biochar becomes non-linear at increasing concentrations.
    \item Sorption of PFAS to biochar is attenuated by the presence of soil and competing congeners. 
\end{itemize}
%%%%%%%%%%%%%%%%%%%%%%%%%%%%%%%%%%%%%%%%%%%%%%%%%%%%%%%%%%%%%
\section{Scope} 
What research has already been done on this subject? How does the work of this thesis stand out?
%%%%%%%%%%%%%%%%%%%%%%%%%%%%%%%%%%%%%%%%%%%%%%%%%%%%%%%%%%%%%
\section{Approach}
A literature study is aimed at...
Laboratory investigations are to be conducted to find... 
