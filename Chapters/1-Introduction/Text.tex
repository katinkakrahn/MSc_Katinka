\chapter{Introduction}\label{chap:intro}

\section{PFAS}
\textit{Per}- and polyfluorinated alkyl substances (PFAS) are a large group of synthetic compounds that are used in numerous industrial and consumer products. As a result of decades of use, PFAS is ubiquitous in soils, groundwater, and surface water \citep{rankin2016north}. PFAS are both oil and water repellent, which makes them ideal as foaming agents and flame retardants, as a coating for the famous waterproof Gore-Tex\textsuperscript{\textregistered} textiles and non-stick, frictionless Teflon\textsuperscript{\texttrademark} cookware \citep{du2014adsorption}. Despite having many desirable properties, widespread production and distribution of PFAS with waterways has led to accumulation in soil, crops, wildlife, and higher trophic levels, including humans \citep{bhhatarai2011,Lau2007}. The synthetic structure of PFAS makes these chemicals resistant to natural degradation, bioaccumulative and toxic. Today, PFAS are recognized as emerging persistent organic pollutants. Legislation has been introduced to regulate the manufacture, sale, and use of several PFASs \citep{EPA2014,ECHA2020,EC2020PFAS,MD2016workshop}. Two of the most widely used and distributed compounds, PFOA and PFOS, are listed under the Stockholm Convention as Persistent Organic Pollutants (POPs). Considerations to add other compounds to this convention are ongoing \citep{EC2020PFAS}. A great deal of research has been done to develop ways of effectively remediating PFAS-contaminated soil and water. One of these is the overarching focus of this thesis. Research conducted for this thesis has been done in collaboration with the Norwegian Geotechnical Institute as part of the Valorization of Organic Waste (VOW) project funded by the Norwegian Research council. The aim of this thesis has been to study whether biochar from various waste materials can be used as sorbents in remediation of PFAS-contaminated soil and water.

\subsection{Physicochemical properties}\label{sec:physchem} 
PFAS are synthetic organic compounds that consist of a polar head, most commonly a carboxyl or sulfate functional group, and a non-polar chain of alkyl moieties that are either fully substituted (per-), or partly substituted (poly-), with fluorine \citep{wang2011physchem}. PFASs are divided into compound classes based on functional group and degree of fluorination, of which perfluorinated carboxylic acids (PFCAs) and perfluorinated sulfonic acids (PFSAs) are the most common. Since fluorine is the most electronegative atom identified, the C-F covalent bonds make one of the strongest known bonds in organic chemistry (BDE=485 kJ mol\textsuperscript{-1}) \citep{Lau2007}. The nature of these bonds become significant for two main reasons: 1) C-F bonds are not found naturally in the environment, and this means that no natural enzymes can degrade them. This accounts for why PFAS is persistent, bioaccumulative, and toxic in food chains \citep{hale2020persistent,krafft2015per}. 2) The chain of repeating CF\textsubscript{2} units consists of highly polar bonds with a symmetric structure. This makes the  entire surface area of the PFAS tail carry a net negative charge. This in turn minimizes the ability of PFAS to undergo neither van der Waals interactions with other molecules nor hydrogen bonds due to the lack of a positive dipole \citep{Arp2006}. The combined properties of low van der Waals and non-polarity make the tail neither hydrophobic nor lipophobic. The result is a new compound class that is "everything-phobic". In the environment, this everythingphobic-ness causes PFAS to behave uniquely in the environment. 

Meanwhile, the behavior of PFAS is often discussed interchangeably with hydrophobic compounds that have strong affinities to hydrophobic surfaces. More correctly, sorption of PFAS should be discussed in terms of a shear pushing of PFAS by water molecules onto hydrophobic surfaces \citep{Arp2006}. This is the primary mechanism that reduces the mobility of PFAS leading to accumulation in soils and biota. Dissolution becomes increasingly difficult at longer chain lengths. The cavity formation energy barrier rises for every $CF_2$ moiety present \citep{bhhatarai2011,Arp2006}. The effect of chain length on partitioning between dissolved and solid phases is one of the main themes that will be discussed in further detail throughout this thesis.

Sorption is a collective term used to describe removal of organic contaminants from the water phase by \textit{ab}sorption (bonding within a material) and \textit{ad}sorption (bonding on the surface of a solid). The role van der Waals forces play, as well as other mechanisms of attraction and repulsion for PFAS interaction with sorbents, will be discussed in a later section on sorption mechanisms \cref{sec:mechanisms}. 

In contrast to the everything-phobic chain, the hydrophilic head of PFAS can hydrogen bond to other polar compounds such as water, and interact electrostatically with positively charged species \citep{sigmund2022sorption}. Since the head groups of PFAS are acids, knowledge on its protonation state is key to further understand solubility and volatility, which in turn affect the mobility and long range transport potential of these chemicals. Determining the acid dissociation constants (\(pK_a\)) of PFAS has been the subject of much debate in the scientific community \citep{Goss2009comment}. This debate is linked to the complexity of how PFAS behaves at different concentrations, and on the water surface. However, most researchers agree that \(pK_a\) values are below 1 and decrease with increasing chain length \citep{wang2011physchem,Reemtsma2016}. Regardless, PFAS are expected to be negatively charged at environmentally relevant pH levels, enhancing its ability to dissolve in water by charge-assisted H- bonds \citep{Reemtsma2016}. For example, PFOA has a water solubility of 3.4 g L\textsuperscript{-1} \citep{PFOA}, whereas its structural analog, perfluorooctane (PFO) \citep{PFO}, which has no polar functional group, is practically insoluble (1 mg/L). 

\subsection{Mobility and accumulation of PFAS in the environment}
The combination of an everything-phobic tail and a ionizable hydrophilic head makes mobility of PFAS in the environment more complex than legacy POPs \citep{cabrerizo2018legacy,Arp2006}. Since PFASs are not very volatile---low vapor pressure and air-water partition coefficient ($K_{AW}$)---aqueous solubility and uptake in migrating biota play important roles for the mobility of PFAS in the environment \citep{Arp2006}. \cite{Schlabach2017} reports that PFCAs with $CF_2$ chain lengths 5-10 are present in Arctic biota such as fish, polar bears and mink. Sorption mechanisms are also slightly different for PFAS compared to hydrophobic and lipophilic legacy POPs like polychlorinated biphenyls (PCBs), hexacholorocyclohexanes (HCHs), and dioxins, which have higher volatility, lower solubility and higher lipohilicity \citep{cabrerizo2018legacy,Cornelissen2005,li2018}. The latter property is often expressed as the octanol water partition coefficient (\(K_{OW}\)). This coefficient is used to represent partitioning between a water phase and a hydrophobic phase \citep{Reemtsma2016} and can be used to explain why PFAS leaches more readily than, for example polycyclic aromatic hydrocarbons (PAHs). This is due to a lower partition coefficient ($K_{OC}$) than hydrocarbons and surfactants. And the polar head makes PFASs more mobile \citep{Cornelissen2005,du2014adsorption}). This means that PFAS has a higher tendency to leach from soils to groundwater, something that can be illustrated by the low partitioning coefficients between OC and water ($K_{OC}$) compared to compounds with higher $K_{OC}$ like PAHs and PCBs that sorb stronger to organic matter \citep{Cornelissen2005}. A more environmentally relevant parameter than \(K_{OW}\) is soil organic carbon (OC), which is an important driver of the sorption of PFASs in soil and sediments.

\subsection{PFAS in industrial runoff and wastewater}
Apart from accumulation of PFAS in the unsaturated zone and leaching into groundwater, PFAS is collected from local industrial point sources such as paper mills \citep{lee2020legacy,langberg2021paper}, leaching from landfills \citep{masoner2020landfill}, firefighting training facilities \citep{MD2016workshop} and fluorochemical plants \citep{gebbink2017presence} into the wastewater. This places high pressure on wastewater treatment plants (WWTPs) for proper handling to avoid further release of PFAS chemicals  \citep{Morin2017flameWaste}. Upon restriction of many long-chain PFASs, new challenges have emerged with controlling the spread and effects of the increased use of short-chain replacements \citep{knutsen2019leachate}. These have higher mobility and tend to slip through existing water treatment processes, thereby contaminating food and drinking water \citep{hale2020persistent,brendel2018short}. Although short-chain PFAS are less bioaccumulative, persistence and toxicity is expected to be equivalent to the substances they replace \citep{EC2020PFAS}. 

Biosolids, the residual semi-solid material which is left over after treatment of sewage sludge, is produced in large quantities and is expensive to dispose of because it is heavily polluted \citep{Raheem2018}. WWTPs have started to look at possible ways to handle sewage sludge and biosolids in a more sustainable and cost-effective way \citep{Raheem2018}. In some cases, biosolids are applied to agricultural fields as fertilizers \citep{moodie2021legacy}. However, high contents of micro- and macro plastics and heavy metals may limit its suitability for this use. Anaerobic digestion (AD) of sludge yields digestate as the final product rich in nutrients (P, K, N) which can be further utilized as fertilizer and/or compost. Digestate is the liquid residual fraction from anaerobic treatment of sewage sludge and other organic waste that produced commercial bio gas, and other uses for these waste residues are now being looked at  

%%%%%%%%%%%%%%%%%%%%%%%%%%%%%%%%%%%%%%%%%%%%%%%%%%%%%%%%%%%%%%%%%%%%%%%%%%%%%%%%%%%%%%%%%%%%%%%%%%%%%%%%%%%%%%%%%%%%%%%%%%%%%%%%%%%%%%%%%%%%%%%%%%%%%%%%%%%%%%%%%%%%%%%%%%%%%%%%%%%%%%%%%%%%%%%%%%%%%%

\section{Valorization of sewage sludge---no longer a waste?}
Recycling and valorization of waste has become one of the key areas of interest in research and development in efforts to transition society to a circular economy \citep{Ahmad2014}. In recent years, large research funds have been devoted to develop remediation techniques, replacement and regulations of the use of PFAS European Union and nationally funded projects (see \cref{sec:greendeal}). The need for wastewater treatment and soil PFAS stabilization options is urgently needed. One main approach is treatment \textit{in situ} by applying sorbents that immobilize PFAS from the bioavailable, aqueous phase by strong sorption to this carbonaceous, porous material \citep{Ahmad2014,Sormo2021,Kupryianchyk2016b}. Biochar production provides a hopeful route for valorization of this contaminated organic waste that simultaneously can contribute to increased revenue of WWTPs.

\subsection{Biochar as sorbents}
Biochar is a common term for the carbon-enriched product produced from pyrolysis of biomass \citep{LehmannAndJoseph2015}. Pyrolysis is a thermal treatment method that burns organic matter in the absence of oxygen, which forms fused aromatic ring structures that have high porosity and surface area and residual functional groups \citep{LehmannAndJoseph2015}. Additionally, the ash co-products make biochar enriched in macro- and micro nutrients that is particularly beneficial for use as soil amendment in agriculture. The integration of biochar as soil amendment dates back 2,500 years when the Pre-Columbian Amazonian peoples developed \textit{terra preta}--the most fertile soil known--by slash-and-char techniques \citep{Tindall2017,Ahmad2014}. Biochar improves soil health by increasing water retention, carbon content and a more steady nutrient release \citeauthor{Ahmad2014}. Systematic research on the subject matter began as early as in the 20\textsuperscript{th} century \citep{Retan1915}. 

\subsubsection{Activated carbon as sorbents}
In terms of soil and water remediation purposes, the application of biochar as a sorbent is relatively new. Historically, activated carbon (AC), i.e. black carbon (BC), is most common \citep{Cornelissen2005}. The term "activation" is used to describe the process of thermal and chemical treatment of BC with oxygen, steam $\mathrm{H_2O}$, $\mathrm{CO_2}$ or $\mathrm{N_2}$ that expands its surface area by the creation of new micropores ($<$2nm), simultaneously removing (oxidizing) surface functional groups to create a more smooth, aromatic surface \citep{sajjadi2019comprehensive}. This process improves the sorptive capacity of biochar making ideal sorbents for organic pollutants \citep{Ahmad2014}. However, high operating costs and energy inputs are required for production of AC, in addition to that it is produced from non-renewable fossil fuel coal \citep{Li2019}. Additionally, the activation process is associated with higher carbon losses and lower yields, resulting in increased greenhouse gas emissions.

\subsubsection{Activated biochar as sorbents}
Researchers on the potential for using activation technology on biochar has been ongoing for many years and has had promising results \citep{Sormo2021}. The main motivation for using biochar instead of AC is that the energy required to produce biochar is many times lower than AC. Biochar can serve numerous multidisciplinary applications. Applications include soil fertility improvement, waste recycling, carbon sequestration, and sorbents for remediation of contaminated sites \citep{Ahmad2014}. Today, it is especially the potential for biochar produced from various waste materials to be used as sorbents to reduce the leaching of contaminants that have caught researchers attention \citep{Cornelissen2011Capping, Kupryianchyk2016b, Sormo2021, zhou2010sorption}. The goal with producing biochar from sewage sludge is to have a net negative GWP and energy use compared to AC, but no definite numbers exist today. LCA is being conducted through collaborators with this project and will be discussed in \cref{sec:LCA}. 

\subsubsection{Biochar sorbents for PFAS}
In recent years, there has been much interest in studying whether non-activated biochars can be effective sorbents for contaminant remediation, and how degree of activation, activation temperature and activation agent type plays a role for PFAS sorption \citep{Sormo2021}. If engineered correctly, biochar is expected to be least as effective as AC in sorption of organic pollutants, in addition to being a more sustainable and cost-effective alternative to AC \citep{Alhashimi2017, Sormo2021}. However, previous literature report that activation showed to be an important factor for sorption of PFAS when applied to organic matter-rich soil due to larger degree of pore blocking by the organic matter \citep{Sormo2021}. Meanwhile, application of biochar to soil with low organic matter content has been shown to be equally efficient with non-activated biochar \citep{Alhashimi2017}. 

Biochar structure and properties differ according to feedstock, and pyrolysis residence time and temperature \citep{Ahmad2014}. Increasing the pyrolysis temperature increases surface area and porosity, properties postulated to optimize sorption \citep{Ahmad2014}. The influence of pyrolysis temperature on physicochemical properties of biochar from sewage sludge \citep{Figueiredo2018}Biochar is considered a universal sorbent because they can be tailored to match the target contaminants by knowledge of how to properly adjust these factors into desired sorbent properties \citep{Ahmad2014}. The biochar surface carries a net negative charge that gives biochar CEC properties ideal for steady nutrient release for agriculture \citep{das2020application}, and alternatively, cation bridging of anionic species \citep{silvani2019can}, regions of more hydrophilic or charged functional groups ideal electrostatically binding cationic organic and inorganic species such as heavy metals \citep{silvani2019can,zhang2013sorption}, and high porosity and aromatic surface area ideal for sorption of hydrophobic and larger molecules \citep{Cornelissen2005}. The oxygen to carbon (O/C) ratio is used as proxy for polarity and hydrophobicity of a sorbent's surface, where a high ratio is associated to a more reduced surface with oxygen-containing functional groups, whereas a hydrogen/carbon (H/C) ratio is an estimate of aromaticity \citep{Ahmad2014}. Biochar produced at lower pyrolysis temperature have a higher O/C and H/C ratio, making them more suitable for agronomic purposes, while biochar produced at higher temperature have lower O/C and H/C ratios, which are associated with increased porosity and aromatic surface area suitable for sorption of hydrophobic organic contaminants, proposed to apply to PFAS as well. Commercial production is growing internationally and is now widely used for soil remediation \citep{Ahmad2014}.

\subsection{Potential for sewage sludge biochar as sorbents for PFAS}
The incorporation of waste-based biochar in soil remediation has become a particularly attractive concept due to the circular concept of taking a problematic waste material and turning it into a commercial sorbent that additionally has a major potential for carbon sequestration and eliminates the need for energy-intensive sewage sludge treatment \citep{arvaniti2014sorption}. However, production of sorbents from sewage sludge is today still in a pioneering phase, from which there are several proposed challenges: 1) Sewage sludge has a lower carbon content and is therefore expected to have low surface area and porosity, and hence, poor sorption strength. 2) Non-activated biochar consists of more oxygen and nitrogen-containing functional groups which makes it more polar and thus attracts charged and polar contaminants at a higher extent. Non-activated biochar contains a higher non-carbonized fraction that interacts with contaminants in a different way than fully condensed aromatic structures. Discussing the relevance/benefits of this more heterogeneous matrix for sorption of PFAS is one of the main focuses of this thesis. 3) the heterogeneous feedstock might cause challenges with reliability of sewage sludge sorbents. For some contaminants, this biochar might be a better sorbent, and poor for other. It is expected that biochar from waste materials may take longer to adsorb contaminants than AC, so either need more biochar or longer time to reach same adsorption equilibrium. 4) bio oils and syn-gas produced as by-products is expected to contain a concentrated cocktail of pollutants and is a possible source of greenhouse gases, particulate matter and heavy metals. Pollution control measurements was conducted simultaneously by research supervisors will provide important information that goes into the overall life cycle impact assessment which is discussed further in \cref{sec:LCA}). Meanwhile, benefits from omitting the activation step will lead to higher biochar yield and lower ash content. Additionally, lower loss of volatile C for feedstocks with higher inorganic constituents like sewage sludge because minerals in ash raises bond dissociation energy of organic and inorganic carbon bonds \citep{Cantrell2012,Enders2012}. 

\subsection{Sorption mechanisms}\label{sec:mechanisms}
Sorption is a function of several factors such as biochar morphology, contaminant concentration, competing contaminants, sorbate physicochemical properties, and molecular size \citep{Li2019,du2014adsorption}. Sorption to porous carbonaceous materials occur via chemisorption (molecular interactions of attraction and repulsion) and physisorption (encapsulation of the adsorbent in the maze of pores) \citep{Li2019}. It has been postulated that sorption of PFAS occurs primarily via direct (specific) polar interactions, hydrophobic (non-specific) interactions and ion exchange mechanisms \citep{Hale2017fire,yu2009sorption}. Polar interactions include H-bond and charge-assisted H-bond interactions, non-specific interactions refer mainly to the hydrophobic effect, also referred to as cavity formation energy, which results from the sum of forces that limit solubility of large, non-polar molecules in water \citep{Arp2006,sigmund2022sorption}. Electrostatic interactions occur both as attraction and repulsion, and cation bridging is one specific mechanism of electrostatic attraction. Determination of sorption isotherms is a central part to understanding the interactions between biochar and PFAS and sorption capacity \citep{yu2009sorption,Li2019}.








