\chapter{Introduction}\label{chap:intro}

\section{PFAS}
\textit{Per}- and polyfluorinated alkyl substances (PFAS) are a large group of synthetic compounds that are used in numerous industrial and consumer products. After decades of use, PFAS are ubiquitous in soils, groundwater, and surface water \citep{rankin2016north}. PFAS are both oil and water repellent, making them ideal as foaming agents and flame retardants. They are also used as a coating for the famous waterproof Gore-Tex\textsuperscript{\textregistered} textiles, and non-stick, frictionless Teflon\textsuperscript{\texttrademark} used as a coating for cooking utensils \citep{du2014adsorption}. Despite having many desirable properties, the widespread production and distribution of PFAS into waterways has led to its beingaccumulated in soil, crops, wildlife, and higher trophic levels, including humans \citep{bhhatarai2011,Lau2007}. The synthetic structure of PFAS, and the strength of the C-F bonds, makes these chemicals resistant to natural degradation \citep{krafft2015per}. And although the mechanisms of PFAS-toxicity are not well understood, they are also suspected to be bioaccumulative and toxic \citep{ding2013physicochemical,Lau2007}. Today, PFAS are recognized as emerging persistent organic pollutants (POPs) \citep{ECHA2020}, and legislation that regulates the manufacture, sale, and use of several PFASs has been introduced  \citep{EPA2014,EC2020PFAS,goldenman2017study}. Two of the most widely used and distributed compounds, PFOA and PFOS, are listed under the Stockholm Convention as \acrshort{POPs}. Consideration is currently being given to adding other PFAS compounds to this convention \citep{EC2020PFAS}. An important example is efforts by the German Environment Agency to identify a new class of compounds as PMT (persistent, mobile, and toxic) and vPvM (very persistent, very mobile), and to include these under REACH, the EU chemicals regulation (Registration, Evaluation, Authorization and Restriction of Chemicals), and ECHA (the European Chemicals Agency) \citep{hale2020persistent}. \acrshort{PMT}/\acrshort{vPvM} substances have properties that are not covered by the Stockholm Convention, but represent an equally great threat to the environment, and to maintaining the purity of the world's drinking water.

\subsection{Overarching research project}
A great deal of research has been done to develop effective ways of remediating PFAS-contaminated soil and water. One of these is the overarching focus of this thesis. The aim of the research presented here is to study whether biochar from various lightly contaminated organic waste materials can be used as a sorbent in the remediation of PFAS-contaminated soil and water. A sorbent is a porous, carbonaceous material with the ability to remove contamination (adsorbate molecules) from water by strongly binding the contaminants to the sorbent's structure, both internally and externally's \citep{LehmannAndJoseph2015}. Remediation work by the use of a sorbent is done by mixing the sorbent with the contaminated water or soil \textit{in situ}, thereby eliminating transportation needs of the contaminated masses. Sorption is a collective term used to describe removal of molecules from the water phase by \textit{ab}sorption, dissolution-like partitioning into the sorbent, and \textit{ad}sorption, bonding on the surface of the sorbent. These processes immobilize the contaminants, making them unavailable for biological uptake \citep{crccare2017assessment}. The degree to which molecules sorb to solid surfaces depends both on the structure and surface properties of the sorbent, and the adsorbate of question as "like attracts like" \citep{ball2012like}. In In order to achieve the desired decontamination result, it is therefore of the utmost importance to "design" sorbents that match the exact chemical properties of the target adsorbate. The present study is based on information concerning the physicochemical properties of two biochars producerd from two lightly contaminated feedstocks, and one biochar produced from a clean, wood-based feedstock, and to evaluate the suitability of these biochars as sorbents for PFAS.

\subsection{PFAS in industrial runoff and wastewater}
The sources of PFAS contamination are typically local industrial point sources such as paper mills \citep{lee2020legacy,langberg2021paper}, leaching from landfill sites \citep{masoner2020landfill}, firefighting training facilities \citep{filipovic2015historical}, and discharge from fluorochemical plants \citep{gebbink2017presence} into the wastewater system. This puts wastewater treatment plants (\acrshort{WWTPs}) under additional pressure to adequately treat wastewater so as to avoid further release of PFAS chemicals into the environment \citep{Morin2017flameWaste}. By restricting many long-chain PFASs, new challenges have emerged. These are related to controlling the spread and effects of increased use of short-chain replacements \citep{knutsen2019leachate}. Short-chain PFAS have higher mobility and tend to slip through existing water treatment processes, thereby contaminating food and drinking water \citep{hale2020persistent,brendel2018short}. Although short-chain PFAS are less bioaccumulative, persistence and toxicity are expected to be equivalent to the substances they replace \citep{EC2020PFAS}. 

\subsection{Physicochemical and sorption properties of PFAS}\label{sec:physchem} 
PFAS are synthetic organic compounds that consist of a polar head, most commonly a carboxyl or sulfate functional group, and a non-polar chain of alkyl moieties that are either fully substituted (per-), or partly substituted (poly-), with fluorine \citep{wang2011physchem}. PFASs are divided into compound classes based on functional group and degree of fluorination, of which perfluorinated carboxylic acids (\acrshort{PFCA}s) and perfluorinated sulfonic acids (PFSAs) are the most common. Since fluorine is the most electronegative atom identified, the C-F covalent bonds make up one of the strongest known bonds in organic chemistry (BDE=485 kJ mol\textsuperscript{-1}) \citep{Lau2007}. The nature of these bonds is significant for two main reasons: 1) C-F bonds are not found naturally in the environment, which means that no natural enzymes can degrade them. This accounts for why PFAS is persistent in the environment \citep{hale2020persistent,krafft2015per}. 2) The chain of repeating CF\textsubscript{2} units consists of highly polar bonds with a symmetric structure, causing the entire surface area of the PFAS tail to carry a net negative charge. This in turn minimizes the ability of PFAS to undergo either van der Waals interactions with other molecules or hydrogen bonds due to the lack of a positive dipole \citep{Arp2006}. The combined properties of low van der Waals and non-polarity make the tail neither hydrophobic nor lipophobic. The result is a new compound class that is "everything-phobic". This everything-phobicness causes \acrshort{PFAS} to behave uniquely in the environment. They are more mobile than other organic pollutants that often are hydrophobic \citep{hale2020persistent}. A more detailed discussion on the mobility of PFAS is given in the following section.  

Sorption of PFAS is often discussed interchangeably with sorption of hydrophobic compounds that have strong affinities to hydrophobic surfaces. Sorption of PFAS should, more correctly, be discussed in terms of their capacity to directly push water molecules onto hydrophobic surfaces \citep{Arp2006}. The cavity formation energy barrier rises for every $CF_2$ moiety present, making dissolution increasingly difficult at longer chain lengths \citep{bhhatarai2011,Arp2006}. This is the primary mechanism that reduces the mobility of PFAS, leads to its accumulation in soils and biota. The role different mechanisms of attraction and repulsion play, as well as the effect of chain length on PFAS interaction with sorbents, are two of the main topics that will be discussed in further detail throughout this thesis. 

In contrast to the everything-phobic chain, the hydrophilic head of PFAS can hydrogen-bond to other polar compounds such as water, and interact electrostatically with positively charged species \citep{sigmund2022sorption}. Since the head groups of PFAS are acids, knowledge of its protonation state is key to further understand solubility and volatility, which in turn affect the mobility and long range transport potential of these chemicals. Determining the acid dissociation constants (\(pK_a\)) of PFAS has been the subject of much debate in the scientific community \citep{Goss2009comment}. This debate is linked to the complexity of how PFAS behaves at different concentrations, and on the water surface. However, most researchers agree that \(pK_a\) values are below 1 and decrease with increasing chain length \citep{wang2011physchem,Reemtsma2016}. PFAS are thus expected to be negatively charged at environmentally relevant pH levels, enhancing their ability to dissolve in water by charge-assisted H-bonds \citep{Reemtsma2016}. For example, PFOA has a water solubility of 3.4 g L\textsuperscript{-1} \citep{PFOA}, whereas its structural analog, perfluorooctane (PFO) \citep{PFO}, which has no polar functional group, exhibits a far lower solubility (1 mg/L). Furthermore, hydrophobic organic compounds (\acrshort{HOC}s) are much more insoluble. Hexachlorobenzene (HCB), for example, has a solubility of $0.005 mg/L$ \citep{mcphedran2013hydrophobic}. 

\subsection{Mobility and accumulation of PFAS in the environment}
The combination of an everything-phobic tail and an ionizable hydrophilic head makes understanding the mobility of PFAS in the environment more complex than that of legacy POPs \citep{cabrerizo2018legacy,Arp2006}. Since PFASs are not very volatile---having a low vapor pressure and air-water partition coefficient ($K_{aw}$)---aqueous solubility and uptake in migrating biota play primary roles in mobilizing PFAS in the environment \citep{Arp2006}. \cite{Schlabach2017} reported that a diverse array of PFASs are present in Arctic biota such as fish, polar bears and mink. Sorption mechanisms are also slightly different for PFAS compared to hydrophobic and lipophilic legacy POPs like polychlorinated biphenyls (PCBs), hexacholorocyclohexanes (HCHs), and polycyclic aromatic hydrocarbons (PAHs) which have higher volatility, lower solubility, and higher lipophilicity than???\citep{cabrerizo2018legacy, Cornelissen2005,li2018}. The latter property is often expressed as the octanol water partition coefficient (\(K_{ow} = C_{o}/C_w\)). This coefficient is used to represent partitioning of adsorbate molecules between particulate organic matter in the water phase (numerator) and water (denominator) \citep{Reemtsma2016}. Due to the unique properties of PFAS described above, PFAS has a lower \(K_{ow}\) value as compared to HOCs. Therefore, \(K_{OW}\) can be used to explain why PFAS more readily leaches from soils to groundwater \citep{Cornelissen2005,du2014adsorption}). A more environmentally relevant parameter than \(K_{ow}\) is \(K_{oc}\), the organic carbon-water partition coefficient. Determining \(K_{oc}\) is tedious, costly and time-consuming. Hence models have been found to estimate \(K_{oc}\) from water solubility and \(K_{ow}\) \citep{pandey2021qspr}. \(K_{oc}\) can be calculated by the empirical equation derived by \citep{karickhoff1981semi}:

\begin{equation}
    \log K_{oc} = 0.989 \times \log K_{ow} - 0.346
\end{equation}

This coefficient also represents the higher tendency of PFAS to leach from soils to groundwater, and is a more dynamic parameter because it accounts for the fraction of organic matter present in the soil. PFAS has a lower \(K_{oc}\) than hydrophobic organic contaminants like PAHs and PCBs that sorb more strongly to organic matter \citep{Cornelissen2005}. The influence of soil organic matter on sorption of PFAS to biochar will be discussed \cref{sec:attenuation} later in this chapter.  

%%%%%%%%%%%%%%%%%%%%%%%%%%%%%%%%%%%%%%%%%%%%%%%%%%%%%%%%%%%%%%%%%%%%%%%%%%%%%%%%%%%%%%%%%%%%%%%%%%%%%%%%%%%%%%%%%%%%%%%%%%%%%%%%%%%%%%%%%%%%%%%%%%%%%%%%%%%%%%%%%%%%%%%%%%%%%%%%%%%%%%%%%%%%%%%%%%%%%%

\section{Biochar---from traditional soil amendment to sorbents for emerging contaminants}
Biochar (BC) is a common term for the carbon-enriched product produced from pyrolysis of biomass \citep{LehmannAndJoseph2015}. Common feedstocks used to produce BC are various crop residues, poultry-production byproducts, wood shavings, and grain straw \citep{Ahmad2014}. Pyrolysis is a thermal treatment method that burns organic matter in the absence of oxygen, a process which forms fused aromatic ring structures that have high porosity and surface area, and low residual functional groups \citep{LehmannAndJoseph2015}. BC improves soil health by increasing water retention and carbon content. It also increases the pH of acidic soils, and secures a more steady nutrient release \citep{das2020application}. In addition, the ash co-products enrich biochar with macro- and micro nutrients that are also beneficial in agriculture. The use of biochar as a soil amendment dates back 2,500 years to the pre-Columbian Amazonian peoples \citep{Tindall2017}. They used slash-and-char techniques to produce biochar which they spread onto their fields. The result was \textit{terra preta}, the most fertile soil known \citep{Ahmad2014}. Today, biochar has numerous multidisciplinary applications, such as soil fertility improvement, waste recycling, carbon sequestration, and sorbents for remediation of contaminated sites \citep{beesley2011review}.

\subsection{Biochar as sorbents}
The application of biochar as a sorbent for soil and water remediation, is relatively new \citep{beesley2011review}. Historically, activated carbon (\acrshort{AC}) from fossil coal sources has been the only product used for soil remediation \citep{hagemann2018activated}. The term "activation" is used to describe the thermal treatment ($<$ 800\textdegree C) of a carbonaceous material by using steam ($\mathrm{H_2O}$), or by using carbon dioxide ($\mathrm{CO_2}$). These treatments expand the surface area of carbonaceious material by creating new nanopores ($<$2nm), thereby increasing sorption capacity \citep{LehmannAndJoseph2015}. Simultaneously, surface functional groups are oxidized to create a smoother, aromatic surface that improves sorption affinity to organic molecules \citep{sajjadi2019comprehensive}. This process improves the sorptive affinity of the biochar, making it an ideal sorbent for organic pollutants \citep{Ahmad2014}. Current challenges connected to production of AC include high operating costs and energy inputs, the use of a non-renewable fossil fuel (coal) as raw material \citep{Li2019}, and the fact that the activation process is associated with higher carbon losses and lower yields. All of these challenges result in reduced carbon sequestration \citep{Sormo2021}. Biochar is often termed a universal sorbent because its surface will have regions of hydrophilic or charged functional groups that make it ideal for electrostatically binding cationic organic and inorganic species such as heavy metals \citep{silvani2019can,zhang2013sorption}, as well as other regions of more condensed, aromatic surfaces which are better suited to sorb hydrophobic molecules \citep{Cornelissen2005}.

Researchers have been successful in using activation technology on biochar instead of fossil coal as a sorbent for organic contaminants \citep{Sormo2021}. Commercial production is increasing internationally, and activated biochar is now being widely used for soil remediation \citep{hagemann2018activated}. In recent years, there has been much interest in studying whether non-activated biochars can also be effective sorbents for use in contaminant remediation and water purification \citep{hagemann2018activated}. Biochar's  main advantage over AC is its potential for carbon sequestration \citep{smith2016soil} and a reduced reliance on Chinese coal mines \citep{zheng2019influence}. If engineered correctly, biochar is expected to be at least as effective as AC in sorption of organic pollutants \citep{Sormo2021}. 
An important area of investigation has been to study how feedstock pyrolysis temperature and residence time affect biochar properties \citep{Hale2016}. Since feedstocks used to produce biochars often have a heterogeneous composition of elements other than carbon, they need to be pyrolyzed at higher temperatures to gain a sufficient surface area and porosity for sorption \citep{beesley2011review}. By adjusting pyrolysis temperature, biochar can be tailored to match the physicochemical properties of the contaminants of interest \citep{Hale2016}. The oxygen to carbon (O/C) ratio is used as proxy for polarity and hydrophobicity of a sorbent's surface, where a high ratio is indicative of a more oxidized surface, one that is high in oxygen-containing functional groups. Likewise, the hydrogen/carbon (H/C) ratio is used as proxy for aromaticity, where a lower ratio is indicative of a higher degree of fused aromatic ring structures that form a more porous biochar material \citep{Ahmad2014}. O/C and H/C ratios have been found to decrease with increasing pyrolysis temperature \citep{Hale2016}. Therefore, biochars produced at high temperatures (700-900 \textdegree C) are the most suitable for sorbing organic contaminants \citep{Figueiredo2018}. 

One of the main challenges posed by the application of non-activated biochar in natural systems is its lower porosity than AC \citep{leng2021overview}. High porosity becomes particularly important when carbonaceous sorbents are applied to organic matter (\acrshort{OM})-rich soils. Small pores are vulnerable to pore blockage by large organic molecules \citep{Sorengard2019}. However, application of biochar to soil with low OM content has proved to be equally effective as the application of non-activated biochar \citep{Alhashimi2017}. In this way, biochar contributes to reducing the demand for AC in a range of remedial uses, and thereby limiting the future need for AC to the treatment of OM-rich soils, and ultra-fine water cleansing. 

The latest research on biochar has studied the potential for producing biochar sorbents from various lightly contaminated organic waste materials such as papermill waste \citep{van2010effects}, sewage sludge \citep{fathianpour2018lead}, biosolids \citep{wang2011}, and palm oil mill sludge \citep{lam2017adsorption}.  Using these waste materials represents a better waste-management alternative to the use of landfill sites an incineration. This will result in a net reduction in energy use, a reduction in the emission of GHGs,  and reduced global warming \citep{Alhashimi2017}. Ongoing research, including the present work, investigates the effectiveness of non-activated biochar as sorbents from waste feedstocks for PFAS. 

\subsection{Sorption mechanisms}\label{sec:mechanisms}
Sorption to \acrshort{BC} involves several factors such as biochar morphology, contaminant concentration, competing contaminants, sorbate physicochemical properties, and molecular size \citep{Li2019,du2014adsorption}. Sorption to porous carbonaceous materials occurs via chemisorption (covalent sorbent-sorbate bonding), and physisorption (encapsulation of the adsorbent in the biochar's maze of pores and electrostatic attraction) \citep{Li2019}. Researchers have postulated that sorption of PFAS occurs primarily via direct (specific) polar interactions, hydrophobic (non-specific) interactions, and ion exchange mechanisms \citep{Hale2017fire,yu2009sorption}. Polar interactions include H-bond and charge-assisted H-bond interactions. Non-specific hydrophobic interactions can more accurately be described as high cavity formation energy which results from the sum of forces that limit solubility of large, non-polar molecules in water \citep{Arp2006,sigmund2022sorption}. Electrostatic interactions involve both attraction and repulsion. Cation bridging is one form of electrostatic attraction.

\subsection{Modeling sorption with sorption isotherms}
Determination of sorption isotherms is a central part to understanding the interactions between biochar and PFAS and sorption capacity \citep{yu2009sorption,Li2019}.

%%%%%%%%%%%%%%%%%%%%%%%%%%%%%%%%%%%%%%%%%%%%%%%%%%%%%%%%%%%%%%%%%%%%%%%%%%%%%%%%%%%%%%%%%%%%%%%%%%%%%%%%%%%%%%%%%%%%%%%%%%%%%%%%%%%%%%%%%%%%%%%%%%%%%%%%%%%%%%%%%%%%%%%%%%%%%%%%%%%%%%%%%%%%%%%%%%%%%%%%%%%%%%%%%%%%%%%%%%%%%%%%%%%%%%%%%%%%%%%%%%%%%%%%%%%%%%%%%%%%%%%%%%%%%%%%%%%%%%%%%%%%%%%%%%%%%%%%%%%%%%%%%%%%%%%%%%%%%%%%%%%%%%%%%%%%%%%%%%%%%%%%%%%%%%%%%%%%%%%%

\section{Valorization of sewage sludge}
Recycling and valorization of waste have become key areas of interest in research and development aimed at achieving a transition to a more circular economy \citep{Ahmad2014}. In recent years, the European Union has invested substantial funding in projects that work on developing remedial techniques for the treatment of soil and water contaminated with PFAS (see \cref{sec:SDGs}) \citep{EC2020PFAS,ECHA2020}. One of the most promising techniques developed so far is an \textit{in situ} treatment that involves applying sorbents which immobilize PFAS from the bioavailable aqueous phase by strong sorption to biochar \citep{Ahmad2014,Sormo2021,Kupryianchyk2016b}.

\subsection{Sewage sludge biochar as sorbent for PFAS}
Biosolids are the residual semi-solid material waste left over from wastewater treatment. They are produced in large quantities and are expensive to dispose of because they are often polluted with heavy metals, micro plastics and organic pollutants \citep{Raheem2018}. High moisture and ash contents, as well as the presence of heavy metals and a cocktail of organic contaminants contained in sewage sludge, make treatment and disposal difficult \citep{Li2019}. Incineration or landfilling, disposal methods often resorted to, release significant greenhouse gases (\acrshort{GHG}), fly ash and \acrshort{PAHs} \citep{huang2022comparative}. These disposal methods are associated with increased risk that contaminants leach into soils and groundwater \citep{propp2021organic}. 

Wastewater treatment plants (WWTPs) have started to look at possible ways to handle biosolids and raw sewage sludge in a more sustainable and cost-effective manner \citep{Raheem2018}. One is to generate energy from waste. With the help of microorganisms, it is possible to produce digestate, the liquid residual fraction from the anaerobic treatment of sewage sludge and other organic wastes (\cref{eq:AD}). This digestate is now being used for the commercial production of bio gas. In some cases, biosolids are applied to agricultural fields as fertilizers \citep{moodie2021legacy}. However, high contents of micro and macro plastics, and heavy metals, may limit the use of biosolids in agriculture \citep{mohajerani2020microplastics}.

Research into the production of biochar using digestate and raw sewage sludge as feedstocks, represents a promising novel strategy to find sustainable ways to valorize organic waste. Production of biochar is particularly attractive because it is considered one of the most promising carbon sequestration technologies, simultaneously transforming a problematic waste material into an economically valuable resource \citep{arvaniti2014sorption}. Biochar from sewage sludge has a higher content of inorganic constituents yielding more ash after pyrolysis. One benefit that comes with higher ash content is a lower loss of volatile carbon attributed to the fact that inorganic ions raise the bond dissociation energy of organic and inorganic carbon \citep{Cantrell2012}. 

Still today, industrial production of sorbents by pyrolysis of sewage sludge is still in a pioneering phase. Some of the challenges include: 1) Sewage sludge biochar has a lower carbon content than more homogeneous, wood-based feedstocks. It likely has a low surface area and porosity, and hence, poor sorption strength. Due to its lower porosity, it is expected that uptake may be lower than by AC. In order to achieve an acceptable quality of treatment for contaminated soil and water, a higher dosage of biochar, or more frequent biochar filter exchange, may be needed. 2) Non-activated biochar possesses of more oxygen and nitrogen-containing functional groups, making it more polar. This in turn results in it being able to attract charged and polar contaminants to a greater extent. Non-activated biochar contains a higher non-carbonized fraction that interacts with contaminants in a different way than fully condensed, aromatic structures. One of the main focuses of this thesis is discussing the relevance and benefits of this more heterogeneous matrix for sorption of PFAS. 3) Due to the heterogeneity of sewage sludge, biochar will vary in composition. The result is that biochar, from time to time, will exhibit inconsistent sorption capacities. 4) Bio oils and syn-gas are by-products from the production of biochar. They are expected to contain organic pollutants, and constitute a possible source of greenhouse gases, particulate matter, and heavy metals. Pollution control measurements were conducted simultaneously by research partners. These will provide important information relevant for an overall life cycle assessment (\acrshort{LCA}) of biochar, a theme that will be further discussed in \cref{sec:LCA}).

%%%%%%%%%%%%%%%%%%%%%%%%%%%%%%%%%%%%%%%%%%%%%%%%%%%%%%%%%%%%%%%%%%%%%%%%%%%%%%%%%%%%%%%%%%%%%%%%%%%%%%%
\section{Application of sorbents to natural systems \label{sec:attenuation}}
For the discussion of remediation of PFAS-contaminated soil and water by the use of sorbents \textit{in situ}, it is important to understand some of the complexity of such natural systems and how the contaminants are distributed between the various phases already. Soils, sediments, sewage sludge and other solid materials that are relevant for wastewater treatment or soil remediation have an intrinsic affinity to PFAS, contributing to lowering the freely dissolved concentrations of PFAS to some degree \citep{arvaniti2014,zhang2013sorption}. The heterogeneity in composition of the various solid phases, size fractions and other contaminants found in such systems are therefore complex and difficult to explain. Some overarching trends from previous literature are attempted summarized here. 

Research agrees that organic matter (OM) is the most important soil and water parameter governing sorption of PFAS in natural systems \citep{zareitalabad2013perfluorooctanoic}. Soil organic matter (SOM) constitutes a diverse range of organic fractions present in soil, and are categorized based on the age of OM \citep{Cornelissen2005}. These include, from fresh, amorphous to hard, aromatic, condensed: biopolymers such as polysaccharides, proteins and lipids; humic substances; and soot and charcoal-like material, collectively termed black carbon \citep{cornelissen2004sorption}. The various OM-fractions have different affinities to PFAS. Hydrophobicity increases for the more condensed and aromatic carbon fractions. Therefore, PFAS have higher affinities to black carbon, which share similar characteristics with biochar, and thus, soils rich in highly condensed carbon have a natural buffer-system for organic pollutants. \cite{Cornelissen2005} provided an informative sketch of the difference in sorption strength or organic molecules between humic organic matter and black carbon (\cref{fig:cornelissen_sorption}). This figure shows that organic contaminants are more tightly bound to black carbon (100 times stronger) than to humic substances. Beneficially, black carbon is more recalcitrant, so once the organic contaminants have been sorbed, they are removed from the bioavailable phase "forever" \citep{Cornelissen2006}. There is only a minor fraction of black carbon contained in natural systems. Most of the organic matter is typically present as humic acids, which are large and complex organic molecules that consist of aromatic and aliphatic regions with a high number of functional groups. 

\begin{figure}[htb]
    \centering
    \includegraphics[width=0.7\textwidth]{Diagrams/Cornelissen_sorption.pdf}
    \caption{Sketch adapted from \cite{Cornelissen2005} of the differences in sorption strength of organic molecules (black dots) between humic organic matter and black carbon affecting bio-availability for microbes and organisms. Note: PFAS is ever rapidly desorbed nor degraded, so the green scheme is not relevant for the present discussion.}
    \label{fig:cornelissen_sorption}
\end{figure}

\subsection{Attenuation}
Attenuation is defined as the weakening of sorption strenght of a sorbent by the presence of soil and/or competing molecules.
The large-sized humic acids propose challenges when it comes to remediation of contaminated soil with biochar (BC) \citep{mahinroosta2020review}. The large molecules prevent efficient sorption of PFAS by clogging the pore throats of the BC, reducing the expected BC-water partitioning coefficient ($K_d$) derived from clean water systems by factors between 7-150 \citep{hale2009sorption, Teixido2013, cornelissen2004sorption}. Dissolved organic matter (DOM) reduces the mass transfer of the adsorbates into the BC pores by the deposition of the large and complex molecules on pore openings \citep{pignatello2006effect}. Furthermore, smaller humic molecules can compete with PFAS for sorption sites, thus reducing the sorption capacity of BC \citep{du2014adsorption}. This clogging effect by the presence of OM becomes a more significant challenge for sorbents with lower porosity. Therefore, activated carbon is commonly used for remediation of OM-rich soils \citep{Sormo2021}. This work will examine the BC properties of the present sewage sludge biochars to evaluate if these sorbents have sufficient porosity to function as sorbents for PFAS in soil. 

Sorbent-water partition coefficients are usually determined during equilibrium conditions. It has been shown that sorption equilibrium is reached between BC and clean water after 14 days \citep{Kupryianchyk2016b}. However, diffusion through clogged pore throats can take much longer time, up to three years in the presence of soil or sediment, where it was shown that AC sorption capacity for DDT in sediment was reduced from a factor 32 after one month to no attenuation after 2.5 years \citep{hale2009sorption}. This shows that pore blocking by SOM could be a kinetic effect more than a competitive effect. OM does not affect sorption capacity of BC, but influences the diffusion rate of adsorbate molecules into the pores. With this knowledge, it is possible to evaluate what rate of PFAS removal is needed for the intended use. The efficiency of AC in water purification is reduced by DOM or microbial biomass that deposits on the AC particles and block adsorption sites and/or pore entrances \citep{Teixido2013}. In this case, three years to reach equilibrium is not realistic because water is expected to flow through the carbon filter at a certain flow rate. Since sorbent amendment is commonly wanted as a fast contaminant removal method, this long time may most often not be acceptable. When it comes to application of sorbents to contaminated soil, preventing leaching of PFAS into the groundwater is urgent, but it may be possible that the native SOM retains PFAS to a sufficient degree that the biochar reaches the intended removal effect eventually. 

Apart from attenuation by OM, the presence of other contaminants has been shown to suppress sorption \citep{Cornelissen2006}.
The result is a "cocktail effect" which means that when a system is overwhelmed by different adsorbing contaminants, molecules experience attenuated sorption depending on their relative physicochemical properties and concentrations. There is an overall agreement that competition between sorbates and/or SOM are the most important parameters that lead to attenuated sorption of PFAS in natural systems \citep{zareitalabad2013perfluorooctanoic,higgins2006sorption,Teixido2013}. 