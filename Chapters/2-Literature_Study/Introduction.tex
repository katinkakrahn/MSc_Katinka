\chapter{Literature Study}\label{chap:LitStudy}
This chapter is a discussion of relevant literature for the work of this thesis with emphasis on the ecotoxicological risks of PFAS pollution in the environment and the application of biochar as sorbent for PFAS-contaminated soil and water.

\section{PFAS}
Per- and polyfluorinated alkyl substances---PFAS---are synthetic, oil- and water-repellent compounds that serve numerous applications in industrial and consumer products \citep{Nicole2013}. These include paint, coatings, firefighting foam, electronics, cosmetics, cookware, and textiles. Despite many desirable properties, the widespread production and use have distributed PFAS with waterways that end up in soil, crops, wildlife, and higher trophic levels including humans \citep{bhhatarai2011}. The persistency, bioaccumulation and toxicity in soil and water poses a major environmental concern that has led to legislative action on the manufacture, sale, and use of several PFAS compounds \citep{EPA2014,ECHA2020,EC2020PFAS}. Today, PFAS is recognized as an emerging contaminant \citep{Ryu2021EC}. By 2014, EPA listed PFOA and PFOS (and their salts) as emergent contaminants and began a phase-out DuPont "PFOA and PFOS have been listed under the Stockholm Convention on Persistent Organic Pollutants (POPs) and as a consequence, are now restricted under the EU POPs Regulation" and several other PFASs are being assessed for restriction under REACH and the Stockholm Convention \citep{EC2020PFAS}. Ongoing restriction of C9-C14 PFCAs under REACH (phased out before 2023) \citep{ECHA2020}. This section looks at the physicochemical properties, ecotoxicological risks, and transport and fate of PFAS in the environment.

\begin{figure}
    \centering
    \begin{tabular}{c}
    \chemname{\chemfig[atom style={scale=0.8}]{O=[:90](-[:30,,,1]OH)-[:150](-[:67.5]F)(-[:112.5]F)-[:210](-[:247.5]F)(-[:292.5]F)-[:150](-[:67.5]F)(-[:112.5]F)-[:210](-[:247.5]F)(-[:292.5]F)-[:150](-[:67.5]F)(-[:112.5]F)-[:210](-[:247.5]F)(-[:292.5]F)-[:150](-[:90]F)(-[:150]F)-[:210]F}}{perfluorooctanoic acid (PFOA)} \\
\\
    \chemname{\chemfig[atom style={scale=0.8}]{F-[:292.5](-[:67.5]F)(-[:330]S(=[:60]O)(-[:330,,,1]OH)=[:240]O)-[:210](-[:292.5]F)(-[:247.5]F)-[:150](-[:112.5]F)(-[:67.5]F)-[:210](-[:292.5]F)(-[:247.5]F)-[:150](-[:112.5]F)(-[:67.5]F)-[:210](-[:292.5]F)(-[:247.5]F)-[:150](-[:112.5]F)(-[:67.5]F)-[:210](-[:270]F)(-[:150]F)-[:210]F}}{perfluorooctanesufonic acid (PFOS)} \\
\\
    \chemname{\chemfig[atom style={scale=0.8}]{O=[:60]S(=[:60]O)(-[:330,,,1]OH)-[:150](-[:67.5]H)(-[:112.5]H)-[:210](-[:247.5]H)(-[:292.5]H)-[:150](-[:67.5]F)(-[:112.5]F)-[:210](-[:247.5]F)(-[:292.5]F)-[:150](-[:67.5]F)(-[:112.5]F)-[:210](-[:247.5]F)(-[:292.5]F)-[:150](-[:67.5]F)(-[:112.5]F)-[:210](-[:270]F)(-[:210]F)-[:150]F}}{6:2 fluorotelomer sulfonic acid (6:2 FTSA)} \\
\\
   %\chemname{
            %\chemleft
              %  \chemfig{
                   % -C
                      %  ( -[2]F)
                       % ( -[6]F)
                   % -C
                       % ( -[2]F)
                       % ( -[6]F)
                   % -[0]
               % }
           % \chemright]$_n$
       % }{polytetrafluoroethylene (PTFE)} \\
    \end{tabular}
    \caption{The most commonly studied and used PFAS compounds, PFOA, PFOS monomers and the polymer polytetrafluoroethylene used in non-stick, oil- and water-repellent products like Teflon and Gore-Tex.}
    \label{fig:PFASstruct}
\end{figure}
        
\subsection{Physicochemical properties}
PFAS are organic compounds that consist of a polar head, most commonly a carboxyl or sulfate
functional group, and a non-polar perfluoroalkyl chain of varying lengths of which either all
hydrogens (per-), or some (poly-) of the hydrogens bonded to carbon have been replaced with
fluorine atoms \citep{wang2011physchem}. Despite C-F bonds being highly polar (\textDelta \textchi=1.5), the symmetric structure of -CF$_2$- homologues makes the tail hydrophobic. PFCs are divided into compound classes based on the functional group attached to the chain, where the most common groups are perfluorinated carboxylic acids (PFCAs) and perfluorinated sulfonic acids (PFSAs) \cref{fig:PFASstruct}. The amphiphilic nature gives PFASs a set of unique and desirable properties as surfactants \citep{du2014adsorption}. These properties have made them ideal as foaming agents and flame retardants in firefighting foam (aqueous film forming foam (AFFF)) and as coating for the famous waterproof Gore-Tex\textsuperscript{\textregistered} textiles and non-stick, frictionless Teflon\textsuperscript{\texttrademark} cookware, but have also . 

\subsubsection{Intramolecular bonding}
PFASs are referred to as "forever chemicals" because they have an extremely high chemical and physical stability \citep{beans2021}. Since fluorine is a strong electrophile, the C-F covalent bonds make one of the strongest known bonds in organic chemistry (BDE=485 kJ mol\textsuperscript{-1}) \citep{Lau2007}. Since C-F bonds are not naturally found in the environment, there are few enzymes/microbes that can break them \citep{shahsavari2021}. Therefore, PFASs remain persistent molecules that stay in the environment "forever". 

\subsubsection{Intermolecular bonding}
The ability of PFCs to undergo van der Waals interactions can be modeled to say something about adsorption to hydrophobic surfaces in soils and sediment \citep{Arp2006}. Generally, these forces have shown to be insignificant for sorption of PFAS to solid phases for short-chain molecules (\textless C6) but is significant for long-chained PFCs (\textgreater C6) \citep{du2014adsorption}. In contrast to the lipophilic chain, the hydrophilic head can hydrogen bond to other polar compounds such as water and interact electrostatically with positively charged species. As surfactants, PFAS are soluble in water, but at high enough concentrations, the molecules arrange in micelles above the critical micelle concentration and the physicochemical behavior changes dramatically \citep{bhhatarai2011,Goss2009comment}.  

the speciation of the polar functional group on PFASs is affected by its surroundings---pH in particular. Generally, PFASs have low acid dissociation constants (\(K_a\)), and hence, appear as negatively charged in the environment \citep{Reemtsma2016}. Due to the hydrophobic tail, water solubility of PFASs are highly influenced by chain length where short-chain, PFCAs are have the highest solubility \citep{bhhatarai2011}.

\(pK_a\) values for PFCA homologues longer than C3 or C4 acid greatly decrease with increasing chain length. There has been some controversy related to the acid dissociation constants of PFAS, but it is now established that  \(pK_a\) values are greatly discussed/controversy in the scientific community \citep{Goss2009comment},  and is important because protonation affects the volatility of e.g. PFOA, which again affects mobility and spread of toxic contaminants \citep{Goss2009comment,wang2011physchem}. This issue is still not resolved, so p$K_a$ values range from 0-4 according to different researchers. 

\subsection{PFAS in water and soil}
In contrast to water solubility, sorption to soil and sediment plays an important role for retention and accumulation of PFAS in the environment \citep{li2018,LehmannAndJoseph2015,Cornelissen2005}. 
\(K_{OW}\) = octanol water partition coefficient used to say something about partitioning between a water phase and a hydrophobic phase (represented by octanol) \citep{Reemtsma2016}. For example, PFOA has a water solubility of 3.4 g L\textsuperscript{-1} \citep{PFOA} whereas its structural analog without the carboxylic end group is practically insoluble \citep{PFO}. The \(K_{OW}\) is therefore unique for PFAS compared to similar compounds   
Solubility and influences sorption mechanisms and mobility. PFCAs and other acidic PFASs appear in dissociated form due to low \(K_a\)'s at environmentally relevant pH's \citep{wang2011physchem}.
\cref{tab:COSMOtherm} gives an overview of different physicochemical properties of the PFCAs used for the research of this thesis modeled using COSMOterm by \cite{wang2011physchem}. 

Leaching to groundwater and surface water more than PAHs due to a lower partition coefficient ($K_{OC}$) than hydrocarbons, surfactants and the polar head makes them more mobile (more on this in \citep{du2014adsorption}). PFAS pollution of groundwater and soil is particularly prominent around several Norwegian airports where old firefighting training facilities uses fire-fighting foam containing PFAS \citep{MD2016workshop}. PFAS was the predominant compound used here and is now banned and replaced with other PFASs. 

Emerging contaminants definition \citep{Li2019} and by \citep{EPA2014}: an "emerging contaminant" is "a chemical or material that is characterized by a perceived, potential, or real threat to human health or the environment or by a lack of published health standards. A contaminant may also be "emerging" because a new source or a new pathway to humans has been discovered or a new detection method or treatment technology has been developed"

The charged head gives PFAS unique properties and influences the mobility of PFAS in the environment and sorption strength and mechanisms to carbonaceous material. Since PFASs are not very volatile (low vapor pressure and air-water partition coefficient ($K_{AW}$), aqueous solubility plays an important role for the mobility of PFAS in the environment \citep{Arp2006}.

Partitioning behavior \citep{wang2011physchem}: log $K_{OW}$, log $S_W$, log $S_O$ These parameters are again determined by compound polarity and its electronic extreme, ionic charge which make compounds more soluble than non-polar compounds \citep{Reemtsma2016}.

\begin{table}
\centering
\caption{Physicochemical properties of relevant perfluorinated carboxylic acids modeled using COSMOterm by \cite{wang2011SI}.}
\label{tab:COSMOtherm}
\begin{threeparttable}
\begin{tabular}{lccccc}
\toprule
\multicolumn{1}{c}{Acronym} & log $K_{AW}$* & log $K_{OW}$\textsuperscript{\dag} & log $K_{OA}$\textsuperscript{\ddag} & log $P_L$\textsuperscript{\S} & log $S_W$\textsuperscript{\P} \\ \midrule
PFPeA & -2.90 & 3.43 & 6.33 & 3.13 & -0.37 \\
PFHxA & -2.58 & 4.06 & 6.63 & 2.66 & -1.16 \\
PFHpA & -2.25 & 4.67 & 6.92 & 2.20 & -1.94 \\
PFOA & -1.93 & 5.30 & 7.23 & 1.73 & -2.73 \\
PFNA & -1.58 & 5.92 & 7.50 & 1.27 & -3.55 \\
PFDA & -1.27 & 6.50 & 7.77 & 0.82 & -4.31 \\ \bottomrule
\end{tabular}
\begin{tablenotes}
\item * air-water partition coefficient
\item \textsuperscript{\dag} octanol-water partition coefficient
\item \textsuperscript{\ddag} octanol-air partition coefficient
\item \textsuperscript{\S} liquid vapor pressure in Pa
\item \textsuperscript{\P} solubility in water in mol L\textsuperscript{-1}
\end{tablenotes}
\end{threeparttable}
\end{table}


\subsection{Ecotoxicological concerns of PFAS}
Products using PFAS were originally manufactured by emulsion polymerization of PFOA into polytetrafluoroethylene (PTFE), a polymer commonly known under the trademark, Teflon\textsuperscript{\texttrademark} (\cref{fig:PFASstruct}) \citep{Lehmler2005}. PTFE and its monomers like PFOA and PFOS make highly chemically inert compounds. However, traces of monomer residues from the polymer manufacture are carried with the products and as effluent from industrial sites. The PFOA and PFOS conjugates are more mobile than the monomer product and ends up in the environment and humans. PFOS and PFAS are unintended breakdown products during manufacture of AFFF. Production and use of perfluorooctanoate (PFOA) and PFOS and its homologues have been phased out in the western countries following agreements with manufacturers by 2014. "Today, PFOA is selected as a candidate for the “substances of very high concern” by The European Chemicals Agency ("Candidate List of substances of very high concern for Authorization," 2018). There are also planned restrictions under REACH, and global restriction are prepared under the Stockholm.
that have been used for decades which that have been found to have serious environmental and health effects \citep{Lau2007}. During manufacture of use of these products, some PFAS
PFOA a precursor for Teflon, PFOA used in the manufacturing of PTFE, or Teflon since the 1940s \citep{lindstrom2011}. "during production, PFOA can get into the soil, water, and air. It can stay in the environment and in your body for a long time"

\subsubsection{PBT and vPvB properties}
PFAS exhibit persistent, bioaccumulative and toxic (PBT) properties and can be transported long distances (long-range transport potential (LRTP) \citep{EC2020PFAS,MD2016workshop}. PFOA Production and use of PFAS has been used for decades but in  Therefore, recognized as POP.  They are listed in REACH and IUR as POPs (persistent organic pollutants) \citep{Schlabach2017}.   
convention." \citep{Schlabach2017} "Chemical inert:nonflammable, not readily degraded by strong acids, alkalis, or oxidizing agents, not subject to photolysis, hydrolysis or biodegrade  = practically non-biodegradable and persistent in the environment" \citep{Lau2007,EPA2014}. No natural enzymes that can break these bonds
but at the same properties that poses serious environmental and health concerns and behave differently in the environment than any other pollutant. In recent years, the ecotoxicological effects of PFAS in the environment has much research on remediation, replacement and regulations of the use of PFAS. especially related to persistency and long-range transport potential (LRTP) \citep{MD2020EQS} because a wide range of PFAS substances have been detected in Arctic biota such as fish, polar bears and mink \citep{Schlabach2017}. 
"Reductive defluorination in anaerobic environments has been proposed as a pathway for environmental degradation; however, conclusive proof of this occurring in the environment has yet to be documented" \citep{ArpNGI}.
"although the short-chain PFAS alternatives are less bioaccumulative, they are equally persistent in the environment as the substances they replace" \citep{ECHA2020}, so the short-chain PFAS "are starting to be regulated under the EU chemicals legislation (REACH). The short-chain can be particularly dangerous because they can "slip through" water treatment plants and contaminate food and drinking water \citep{Reemtsma2016}. PFAS are are very resistant to biologic degradation and thus accumulate in the environment and in the food chain---ultimately in humans \citep{Schlabach2017,Steenland2010,Lau2007}. 

\subsubsection{LRTP}
Common for all POPs are that they do not only affect locally but are transported to remote areas of the globe and contaminates the most pristine water and soils left on Earth. The large long-range transport potential (LRTP) owes to the chemical inertness and charge. The formal charge of the PFAS polar heads make these compounds much more mobile that a C-F chain alone because the molecule as a whole becomes more polar, which would have comparable properties to long-chain oils which would stick to soil and particles due to high hydrophobicity. Hydrophobicity increases with chain length, so the short-chain PFCAs are typically the most mobile. Table 11 AMAP report shows that PFPeA to PFDA are present in Arctic biota such as fish, polar bears and mink. PFOS has the highest concentration in Arctic of all PFASs. "PFNA was the most prominent compound of the carboxylic acid group" \citep{Schlabach2017}.

On the Norwegian priority list that has set a goal to eliminate all release of PFAS to the environment within 2027 \citep{MD2016workshop}.

\subsubsection{Health effects}
Endocrine disrupting compounds (EDCs)
Health effects of PFOA: \citep{Steenland2010} A review of health effects of PFOA comparing studies on rodents to humans. No convincing evidence of causal relationship except for heightened cholesterol levels. More on this... 
C8 Science Panel = three epidemiologists who carried out exposure and health studies in the communities residing near the DuPont factory, potentially affected by the releases of PFOA (or C8) emitted since the 1950s\footnote{\url{http://www.c8sciencepanel.org/}}
In a study of people living near the fluoropolymer manufacturer, DuPont Washington Works plant on the Ohio West Virginia border, sued the company for contaminating groundwater with PFOA C8 Science Panel  concluded that PFOA is probably linked to six outcomes : high cholesterol, kidney cancer, testicular cancer, ulcerative colitis, thyroid disease, and pregnancy-induced hypertension 
accumulate primarily in the serum, kidney ad liver \citep{EPA2014}. Today, the USEPA advised drinking water concentration limit of PFOA is set to 0.07 \textmu g L\textsuperscript{-1} \citep{us2016drinking}. 

%%%%%%%%%%%%%%%%%%%%%%%%%%%%%%%%%%%%%%%%%%%%%%%%%%%%%%%%%%%%%%%%%%%%%%%%%%%%%%%%%%%%%%%%%%%%%%%%%%%%%%%%%%%%%%%%%%%%%%%%%%%%%%%%%%%%%%%%%%%%%%%%%%%%%%%%%%%%%%%%%%%%%%%%%%%%%%%%%%%%%%%%%%%%%%%%%%%%%%

\section{Biochar}
Biochar definitions \citep{LehmannAndJoseph2015}:
"a carbon (C)-rich product when biomass such as wood, manure or leaves is heated in a closed container with little or unavailable air" \citep{LehmannAndJoseph2015} fused aromatic ring structures that form during pyrolysis (condensing) and play the defining role in sorption properties, enriched in C, P, N, micro nutrients. Macro structure resembles the starting material even though the micro morphology is quite different. Biochar structure and properties can be very adverse according to the feedstock, so the potential for designing/tailoring biochar product to its application is quite possible and has received great attention among researchers. 

The integration of biochar as soil amendment dates back 2,500 years the Pre-Columbian Amazonian peoples developed \textit{terra preta}--the most fertile soil known--by slash-and-char techniques \citep{Tindall2017,Ahmad2014}. Research on the subject matter began as early as in the 20\textsuperscript{th} century \citep{Retan1915}. The difference between activated carbon and biochar will be further discussed in \cref{sec:BCvsAC}. 

Universal sorbent, surface carries net negative charge giving biochar CEC properties, high surface area where some sites are fully aromatic rings/hydrocarbon regions that sorb hydrophobic molecules and some sites are less condensed, residual non-carbonaceous and functional groups like carboxyls on the surface, polar and charged molecules. Sorbing hydrophobic molecules, electrostatically binding polar molecules and attracting cationic organic and inorganic species \citep{Ahmad2014,vanloon2017Ch14}. 

Biochar characteristics: microporosity, surface area, residual functional groups
%Characteristics highly dependent on feedstock type, time and temperature of pyrolysis. Lower temperature pyrolysis leaves more oxygen-containing functional groups which are more efficient at removing polar molecules and positively-charged metal ions \citep{Ahmad2014} rephrase

\subsection{Pyrolysis}
Definition 
Pyrolysis is a thermal treatment method. 
Describe pyrolysis process for AC
"compared to the highly exothermic incineration, most of the pyrolytic reactions are endothermic consuming energy of around 100 kJ kg\textsuperscript{-1}

\subsection{Pyrogenic carbonaceous materials}
From Lehmann and Joseph 2015 \citep{LehmannAndJoseph2015}:
Pyrogenic carbonaceous materials (PCMs): the most general term, all carbon-containing residue from pyrolysis
Char: PCM residue from natural fires
Charcoal: "PCM produced from pyrolysis of animal or vegetable matter in kilns for use in cooking or heating, including industrial applications such as smelting"
Biochar: "carbonaceous material produced specifically for application to soil for agronomic or environmental management" "In 2012 the International Biochar Initiative (IBI) released the first Guidelines for ´Biochar that is used in Soil´to formally define this carbonaceous product and describe its desired characteristics. However, continuing research is required to understand what constitutes ´good´ biochar in agronomic and environmental management applications"
Activated carbon: charcoal that has been treated with oxygen, CO2, N2 to increase microporosity and surface area, which again increases adsorption sites for environmental pollutants. The term "activated" is commonly used to describe the enhanced surface area of charcoal upon thermal or chemical treatment.

\subsection{Biochar properties}
Surface area, chain length, porosity, composition
Literature review of previous findings, includes difference in porosity, surface area by pyrolysis and how this can be tailored to match the type of contaminants present and soil type
Sorbents: particle size in um, PT, RT
Sorbent characterization: total C, H, O, N, ash content, surface area, pore volume
Ash content: gravimetrically by heating under air at 750 \textdegree C until a constant weight was obtained
Surface area and pore size distribution: measured by both BET-N$_2$ adsorption (used for pores >15 Å) and CO2 sorption (used for pores 2.5-15 Å) using a Quantachrome Autosorb I
Pore volume and pore size distributions: estimated from the adsorption isotherms by Density Functional Theory (DFT).
O/C ratio: used as proxy for polarity and hydrophobicity of a sorbent's surface
H/C ratio: estimates aromaticity (AC have the highest degree of aromaticity which increases surface area and further, sorption)
Surface area: m$^2$/g, highest for AC. Samples with large total surface area exhibit high micropore volumes, whereas samples with low total SA, generally has larger fractions of open SA (macropores)
Pore volume: cm$^3$/g

Biochar properties are affected by pyrolysis temperature, residence time, and feedstock type \citep{Ahmad2014}. So, feedstock type is important for biochar properties and therefore it is exciting if waste (sewage sludge) exhibit some advantageous characteristics for use as sorbent for organic pollutants, which is the aim of this thesis. 
\citep{Li2019}
Production yield high inorganic content vs low inorganic content, ash content, lower loss of volatile C for feedstocks with higher inorganic constituents because minerals in ash raises bond dissociation energy of organic and inorganic C bonds \citep{Cantrell2012,Enders2012} as cited in \citep{Ahmad2014}

\subsubsection{Physicochemical properties}
The influence of pyrolysis temperature on physicochemical properties of biochar from sewage sludge \citep{Figueiredo2018}
Smoke/steam, oil, and charcoal
LCA (Life cycle assessment)

\subsection{Biochar applications}
Biochar as a soil amendment goes back centuries, write about how biochar improves soil health: water retention, steady nutrient release,  \citep{Ahmad2014} lists in the introduction many ways biochar functions as a soil amendment): soil improvement, waste management, climate change mitigation, and energy production

Sewage is nutrient rich (P, K, N) and can be used as fertilizer.

\citep{Ahmad2014}:
%\begin{list}
  % \item soil amendment
  %  \item waste management
  %  \item energy production
  %  \item carbon storage, stable over time
  %  \item sorbent 
%\end{list}

\subsection{Biochar versus activated carbon}\label{sec:BCvsAC}
Introduction to the main difference between AC and BC in terms of its history of production and application. 

Non-activated biochar contains a higher non-carbonized fraction that interacts with contaminants in a different way than fully condensed aromatic structures. Non-activated biochar consists of more O-containing functional groups which makes it more polar and thus attracts charged and polar contaminants at a higher extent. This is interesting in light of PFAS, due to the surfactant properties. 

Activation showed to be an important factor for sorption of PFAS when AC was applied to OM-rich soil due to larger degree of pore blocking by the organic matter \citep{Sormo2021}. Application of biochar to soil with low organic matter content proves equally efficient using non-activated biochar \citep{Alhashimi2017}. 

\subsubsection{Activation}
The purpose of biochar activation is to increase microporosity and surface area, has been used for a long time as organic and inorganic pollutants in soil and water \citep{Ahmad2014}. 
See \citep{Li2019}: change surface characteristics during pyrolysis to enhance hydrolysis into small fractions, the more fractions that are made increases surface area. 
Refer to \citep{Li2019} for AC vs BC research. 
\citep{Alhashimi2017}: Benefits of biochar instead of AC includes lower energy demand and global warming potential (GWP), and if engineered correctly, biochar can be least as effective as activated carbon for sorption of environmental contaminants, and the production comes at a lower cost. 

\begin{itemize}
\item Economic and environmental advantages
\item Technologies that are in use for removal of heavy metals from water: chemical precipitation, ion exchange, chemical oxidation and reduction, filtration, membrane technology, reverse osmosis, electrochemical treatment, electrodialysis, electroflotation, electrolytic recovery, and adsorption by activated carbon \citep{schroder2010,du2014adsorption}. 
\begin{itemize}
\item High operating energy and cost
\item Why does biochar can sequester carbon (negative GWP and energy use) compared to activated carbon? Energy required to produce activated carbon is many times larger than biochar. Spent activated carbon is not disposed of but regenerated for reuse
\end{itemize}
\item Why activated carbon may be better in some ways/barriers to overcome to use biochar:
\begin{itemize}
\item Biochar adsorption is not stable enough, can be better than AC for some contaminants, but poor for other
\item Biochar may take longer to adsorb contaminants than AC, so either need more biochar or longer time to reach same adsorption amount
\end{itemize}
\end{itemize}

%%%%%%%%%%%%%%%%%%%%%%%%%%%%%%%%%%%%%%%%%%%%%%%%%%%%%%%%%%%%%%%%%%%%%%%%%%%%%%%%%%%%%%%%%%%%%%%%%%%%%%%%%%%%%%%%%%%%%%%%%%%%%%%%%%%%%%%%%%%%%%%%%%%%%%%%%%%%%%%%%%%%%%%%%%%%%%%%%%%%%%%%%%%%%%%%%%%%%%

\section{Sorption}
Make figure showing different sorption mechanisms.\citep{Li2019} is a good place to start. Sorption of PFAS increases with pyrolysis temperature because then porosity and surface area increases.  \textit{in situ} bioremediation vs \textit{ex situ} bioremediation methods, pros and cons of each

Solution chemistry, sorbent properties, environmental factors affect sorption behavior \citep{du2014adsorption}
Different sorption models are used to express how environmental contaminants are removed from water by sorption to solid phases. Sorption is a function of several factors that include contaminant concentration concentration, competing sorbates, sorbate molecular structure, sorption mechanisms and the physicochemical properties of the biochar like porosity and surface area, particle diameter \citep{Li2019,du2014adsorption}. Make figure with factors influencing sorption (Qvale MSc: sorbent properties (surface chemistry, surface area), site-specific factors (inorganic ions, organic matter, pH), PFAS properties (chain length, functional group). Determination of sorption isotherms is a central part of understanding how PFAS interact with biochar \citep{Li2019}. A sorption isotherm is determined experimentally by preparing batch tests with a set water:soil ratio and spiking the system with increasing doses of the contaminant. A regression line is fitted by plotting the aqueous versus the sorbed concentration can be used to evaluate which sorption model that is most suitable for the system. Sorption of organic contaminants like PFAS by biochar is most best expressed using the Freundlich sorption isotherm model . Langmuir dual-mode are other relations used to describe adsorption. This study uses the Freundlich relation to describe the equilibrium distribution of PFAS between biochar, soil and water. "The Freundlich relation differs from that of Langmuir in that it does not consider all sites on the adsorbent surface to be equal but rather adsorption becomes progressively more difficult as more and more adsorbate accumulates. Furthermore, it is assumed that, once the surface is covered, additional adsorbed species can still be accommodated. In other words, multilayer adsorption is predicted by this relation" \citep{vanloon2017Ch14}
Langmuir assumes sorption sites are finite and adsorption is limited to monolayer coverage, a sorption maximum is reached, reversible process. Langmuir assumptions will not work in the presence of soil
Sorption quantified by determination of partition coefficient, $K_d$

\citep{Li2019} 

\subsection{Adsorption mechanisms}
Influences mobility of PFAS in the environment. 
Mechanisms
several interactions: chemisorption and physisorption \citep{Li2019}:
\begin{itemize}
    \item \textpi-\textpi electron-donor-acceptor (EDA): environmental contaminants are \textpi-electron acceptors and biochar is \textpi-electron donor. Is particularly strong for sorption of planar aromatic compounds to the planar graphene surface of biochar. Biochar consists of poly-condensed aromatic rings which make the surface \textpi-electron dense, called graphitized surface. Has the ability to bind electron-withdrawing molecules by being \textpi-electron donor. (ring condensation increases with pyrolysis temperature of biochar)
    \item electrostatic
    \item pore-filling
    \item hydrophobic
    \item hydrogen bonding
    \item functional groups
    \item cation exchange and bridging interactions
\end{itemize} 

\subsubsection{Dissolution chemistry}
Cavity formation energy
Hydrophobic, an overused word
\citep{sigmund2022sorption}, figure key drivers and interactions for sorption of charged organic compounds
\begin{itemize}
    \item hydrophobic effect
    \item pi-pi electron donor-acceptor interaction
    \item H-bond
    \item charge assisted H-bond
    \item electrostatic repulsion
    \item cation bridging
    \item electrostatic attraction
    \item anion-pi bond
    \item cation-pi bond
\end{itemize}

PFAS influenced by hydrophobic effect, pi-pi electron donor-acceptor interaction, charge-assisted H-bond, electrostatic repulsion
cation bridging, electrostatic attraction (pH-dependent), and anion-pi bond.

Hydrophobic effect = "cavity formation energy, results from the sum of forces that limit the solubility of molecules in water. Its underlying cause is the disrupton of the cohesive energy of water due to the greater ordering of water molecules and the lower number of water-water H-bonds in the hydration shell of the nonpolar moiety compared to the bulk water phase" \citep{sigmund2022sorption}.

"Only $~7\%$ of the numerous minerals in global soils have surfaces that are net positively charged at ambient pH, most importantly Fe-oxides and Al-oxides. 

\subsection{Adsorption models}
Linear: sorption increases proportionally with adsorbate concentration according to the following linear expression:
\begin{equation}\label{linear}
C_s = K_d \times C_w
\end{equation}

Langmuir: assumes constant sorption free energy (ideal monolayer coverage, identical sites, no interaction between sorbates)
Freundlich: assumes log distribution of sorption free energies, more empiric, often better fit and comparison with literature


\subsubsection{Freundlich sorption model}
Kd vs KF, linear vs non-linear sorption
n usually between 0.7 and 1.0. 

Limitation of the Freundlich model is that it does not predict an adsorption maximum. 
Freundlich relation is good for modeling the equilibrium distribution at low concentrations of the organic solute and small molecules \citep{vanloon2017Ch14}
Freundlich isotherm/relation: 
Expression:
\begin{equation} \label{eq:Freundlich}
    C_s = K_F \times (C_{w})^{n_F}
\end{equation}

Where $C_s$ is amount sorbed in \textmu g kg\textsuperscript{-1}, $K_F$ is the Freundlich partition coefficient in L/kg, $C_{w}$ is the equilibrium aqueous concentration in mg/L and n is the exponent of non-linearity. Taking the log of each side of \cref{eq:Freundlich} gives a linear expression:

\begin{equation} \label{eq:FreundlichLinear}
    \log C_s = \log K_F + n_F \times \log C_{w}
\end{equation}

Plotting the linear form in \cref{eq:FreundlichLinear}, the intercept is $log~K_F$ and the slope is n. 

Modeling sorption is complicated in the presence of soil since partition behavior between soil and water and biochar and water must be considered all together. Sorption to soil only is most often represented by the linear expression:

\begin{equation} \label{eq:KD}
    C_s = \frac{K_d}{C_{w}}
\end{equation}

Where $C_s$ = quantity sorbed per unit mass (\textmu g kg\textsuperscript{-1}), $C_{aq}$ =  equilibrium aqueous concentration (\textmu g L\textsuperscript{-1} and $K_d$ is the partition constant between the solid and aqueous phase. A new mass balance for the partitioning between water, soil and sorbent is obtained by building on the Freundlich equation (\cref{eq:Freundlich}) (mass PFAS, m ng\textsuperscript{-1} or \textmu g and mass soil or sorbent, M kg\textsuperscript{-1}). The mass balance between the three phases can be deduced:

\begin{equation} \label{eq:massBalance1}
    m_{tot} = m_{aq} + m_{s} + m_{bc}
\end{equation}

\begin{equation} \label{eq:massBalance2}
     m_{tot} = C_{aq}V_{aq} + C_sM_s + C_{bc}M_{bc}
\end{equation}

\begin{equation} \label{eq:massBalance3}
     m_{tot} = C_{aq}V_{aq} + K_dC_{aq}M_s + K_{F}C_{aq}^{n_F}M_{bc}
\end{equation}
 
Fitting \cref{eq:massBalance3} to the Freundlich equation yields:

\begin{equation} \label{eq:FreundFit}
    m_{tot} - C_{aq}V_{aq} - K_dC_{aq}M_s = K_{F}C_{aq}^{n_F}M_{bc}
\end{equation}

and get the partitioning of PFAS in the biochar (bc) expressed on the left hand side, and the aqueous PFAS concentration on the right hand side of \cref{eq:FreundFit}. To get the linear Freundlich expression log of each side is performed:

\begin{equation} \label{eq:FreundLinSoil1}
   \log (m_{tot} - C_{aq}V_{aq} - K_dC_{aq}M_s) = \log (K_{F}C_{aq}^{n_F}M_{bc})
\end{equation}

Further modifications are made to simplify to the Freundlich linear expression as y = b + ax:

\begin{equation} \label{eq:FreundLinSoil2}
    \log (m_{tot} - C_{aq}V_{aq} - K_dC_{aq}M_s) = \log K_{F} + \log C_{aq}^{n_F} + \log M_{bc}
\end{equation}

\begin{equation} \label{eq:FreundLinSoil3}
    \log (m_{tot} - C_{aq}V_{aq} - K_dC_{aq}M_s) = \log K_{F} + n_F \times \log C_{aq} + \log M_{BC}
\end{equation}

\begin{equation} \label{eq:FreundLinSoil4}
    \log (m_{tot} - C_{aq}V_{aq} - K_dC_{aq}M_s) - \log M_{BC} = \log K_{F} + n_F \times \log C_{aq}  
\end{equation}

\cref{eq:FreundLinSoil4} can be plotted using $\log C_{aq}$ on the x axis and $\log (m_{tot} - C_{aq}V_{aq} - K_dC_{aq}M_s) - \log M_{bc}$ on the y axis for each point on the isotherm. The slope derived from linear regression is $n_F$ and the intercept is $K_F$. Using the Freundlich model, sorption capacity is expressed in terms of \(n\) and affinity is expressed in terms of \(K_F\). 

\subsection{Sorbent addition to natural systems}
Pore filling/clogging see \citep{Li2019}
Important factors with sorbent addition:
Time: slow mass transfer under unmixed conditions for PCBs, transfer from sediment to biochar takes time \citep{Werner2006}
Fouling: weakening of sorbent sorption by natural organic matter, oil, competing contaminants
Interaction of fouling and time: given enough time, fouling is less
coal AC vs biomass-based AC. coal AC is more effective in sediment while biomass-based AC sorbs more strongly than coal-based AC in water \citep{amstaetter2012}. Explained by difference in pore size, coal has a wider spectra of pore widths, avoids pore filling by soil. 
Attenuation: sorption decreases with time
Coating by humic material in high TOC soil, pore blocking \citep{Hale2011}. Functional groups are typically located on the external part of biochar which can hydrogen bond with surface groups on humus preventing hydrophobic contaminants from entering the pores. 
Adsorption rate and intraparticle diffusion: \citep{du2014adsorption}
Describe importance of soil type, i.e., organic matter content in soil for effective sorption
Sorption generally decreases with increasing pH due to electrostatic repulsion between negatively charged surfaces and negatively charged PFAS \citep{du2014adsorption}. Cation bridging effect in high pH by presence of divalent ions such as Mg2+ and Ca2+, increases sorption even at high pH \citep{du2014adsorption}. Relevant for adsorption to sediment, black carbon. "Competition: inorganic anions can compete with anionic PFCs for adsorption sites, resulting in a decrease of PFCs sorption on adsorbents" \citep{du2014adsorption}. 
\citep{Sormo2021}: OM can weaken sorption to biochar due to pore blocking. One might think that soil high in organic matter will be strengthen sorption when amending the soil with biochar, however, this turns out to be the opposite. Refer to figure from \citep{Cornelissen2005}. Organic matter competes as sorption sites for PFAS but the sorption strenght is weaker so desorption will occur more easily than once sorbed to biochar. Pore clogging. 
Steric hindrance
The dispersive forces make the bonding so strong, not Van der Waals forces,  separation distance, condensed aromatic carbon surfaces \citep{Cornelissen2005}

Biochar charge morphology of functional groups are influenced by solution pH, deprotonation and protonation of functional groups, depends on environmental conditions \citep{Li2019}. Influences adsorption ability of environmental contaminants depending on surface morphology, especially hydrophobicity. In terms of PFCAs, biochar is expected to be a better sorbent at low pH due to limitation of electrostatic repulsion of biochar functional groups and the polar head of the perfluoroalkyl carboxylate. 

What is the pH of waste water? Around neutral for normal waste water, but acid mine drainage is a different story. 

%%%%%%%%%%%%%%%%%%%%%%%%%%%%%%%%%%%%%%%%%%%%%%%%%%%%%%%%%%%%%%%%%%%%%%%%%%%%%%%%%%%%%%%%%%%%%%%%%%%%%%%%%%%%%%%%%%%%%%%%%%%%%%%%%%%%%%%%%%%%%%%%%%%%%%%%%%%%%%%%%%%%%%%%%%%%%%%%%%%%%%%%%%%%%%%%%%%%%%

\section{Sustainability}
What matter will the use of biochar be in the green shift? LCA, circular economy

\subsection{Sewage sludge treatment}
Recent challenges with waste handling
incineration, pyrolysis will remove? Results from pollution campaign 
\citep{Morin2017flameWaste}
Energy production occurs from pyrolysis reactions through the gaseous and liquid fuels that are created as coproducts

Found in soils all over the world. The need for sorbent amendments is urgently needed. Remediation techniques. Emerging contaminants like PFAS

The perfluorinated carboxylic acids (PFCAs) are the focus of this thesis.'
PFAS in waste, leachate, sewage sludge  from waste-handling facilities \citep{Morin2017flameWaste}
Landfills, short-chain vs. long-chain PFASs: \citep{knutsen2019leachate}
Paper factory contaminates lake Tyrifjorden \citep{langberg2021paper}
Applicable for Lindum digested sludge char used in this research. Sludge generation in a WWTP \citep{Raheem2018}
Waste activated sludge (WAS) definition: "the residual semi-solid material which is inevitably left over from municipal or industrial wastewater or sewage treatment processes"
What is digestate:
Digestatet er en våt/fast restfraksjon som oppstår når man behandler organisk avfall anaerobt for biometan produksjon i et typisk biogassanlegg. Det kan inneholde mikro-/makroplast og tungmetaller, noe som kan begrense videre utnyttelse. Digestatet produseres i store mengder, noe som gjør transporten kostbar og bidrar negativt til anleggets økonomi.
\citep{Raheem2018} pre-treatment to primary settlement to biological treatment to secondary settlement = WAS
From \citep{Raheem2018}
\begin{itemize}
\item anaerobic digestion: AD transforms sludge organic solids to biogas (methane, ammonia, H$_2$S, CO$_2$). Biogas comprises of 60-70\% methane, and 30-40\% carbon dioxide, trace amounts of other gases. AD yields digestate as the final product containing high amounts of nutrients (P, K, N) which can be further utilized as fertilizer and/or compost. Process is anaerobic fermentation.

Will pyrolysis remove PFAS? Draw parallel to pyrolysis of PTFE (Teflon) that occurs when pan is heated to above 300 \textdegree C, can get polymer fume fever from toxic fumes (carbonyl fluoride and trifluoroacetyl fluoride) 

\begin{equation}
\label{eq:AD}
    \ce{C_cH_hO_oN_nS_s + yH_2O -> xCH_4 + nNH_3 + xH_2S + (c-x)CO_2}
\end{equation}

\item Incineration: "low-caloric surplus heat of exhaust gases released from power plant can be effectively employed to enhance the sludge drying process"
\begin{equation}
\label{Incineration}
    \ce{Biosolids / organics + O_2 (excess) -> CO_2 + H_2O + energy + ash}
\end{equation}
\item pyrolysis: generally characterized based on heating rate, temperature and gas residence time (slow pyrolysis to fast pyrolysis)
\item sludge gasification: converts dried WAS into combustible gases (syngas) under partial oxidation at elevated temperatures of 700-1000 \textdegree C
\end{itemize}

\subsection{The Charcoal Vision}
\citep{Laird2008}: "A win-win-win scenario for simultaneously producing bio-energy, permanently sequestering carbon, while improving soil and water quality". l, 

\subsection{The 4 per mille initiative}
The 4 per 1000 initiative United Nations climate Change conference of December 2015 (COP21)\footnote{\url{https://www.4p1000.org/}}. Annual growth rate of 0.4 \% in the soil carbon stocks, or 4 \textperthousand  per year in the first 30-40 cm of soil a means to reach the goals set forward by the Paris agreement, complement what is necessary efforts to reduce GHG emissions globally. Carbon sequestration. Goal to scale biochar for global agricultural and remediation markets. 

\subsection{Sustainable Development Goals}
Sustainable Development Goals (SDGs)

\subsection{International Biochar Initiative}
Biochar standards

\subsection{The European Green Deal}
Fresh air, clean water, healthy soil and biodiversity
14 July 2021 biochar is a means to achieve the goal to make Europe the first climate neutral continent by economic means: biochar is a potential commercial product\footnote{\url{https://ec.europa.eu/info/strategy/priorities-2019-2024/european-green-deal/delivering-european-green-deal_en}}.







