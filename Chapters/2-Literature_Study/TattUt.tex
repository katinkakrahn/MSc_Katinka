Polanyi Dubinin Manes (PDM): pore-filling theory, most correct isotherm for AC (and probably also biochar). Nanopore filling model. Theoretically the best model for AC, but almost no values to compare with. Expressed in terms of pore volume, sorbate density, gas constant, temperature, maximum solubility, free energy, dimensionless fitting exponent. 
Do not mix linear sorption to soil and Langmuir sorption for a sorbent - Langmuir assumptions won't work in the presence of soil anyway

\begin{equation} \label{eq:PDM}
    C_{BC} = V_0\rho_0 \left [ \frac{-RT\ln \frac{C_{aq}}{S}}{E}\right ]^b
\end{equation}


where $C_{BC}$ is concentration in the biochar (g kg\textsuperscript{-1} dw), $V_0$ is BC pore volume (cm\textsuperscript{3} kg\textsuperscript{-1}), $\rho_0$ is sorbate density (g cm\textsuperscript{-1}, $R$ is the gas constant (J mol\textsuperscript{-1}K\textsuperscript{-1}, $T$ is the temperature (K), S is the maximum solubility (mg L\textsuperscript{-1}), $E$ is the free energy of adsorption (J mol\textsuperscript{-1}), and $b$ is a dimensionless fitting exponent. 

Health effects of PFOA: \citep{Steenland2010} A review of health effects of PFOA comparing studies on rodents to humans. No convincing evidence of causal relationship except for heightened cholesterol levels. More on this... 
C8 Science Panel = three epidemiologists who carried out exposure and health studies in the communities residing near the DuPont factory, potentially affected by the releases of PFOA (or C8) emitted since the 1950s\footnote{\url{http://www.c8sciencepanel.org/}}
In a study of people living near the fluoropolymer manufacturer, DuPont Washington Works plant on the Ohio West Virginia border, sued the company for contaminating groundwater with PFOA C8 Science Panel  concluded that PFOA is probably linked to six outcomes : high cholesterol, kidney cancer, testicular cancer, ulcerative colitis, thyroid disease, and pregnancy-induced hypertension 
accumulate primarily in the serum, kidney ad liver \citep{EPA2014}