Polanyi Dubinin Manes (PDM): pore-filling theory, most correct isotherm for AC (and probably also biochar). Nanopore filling model. Theoretically the best model for AC, but almost no values to compare with. Expressed in terms of pore volume, sorbate density, gas constant, temperature, maximum solubility, free energy, dimensionless fitting exponent. 
Do not mix linear sorption to soil and Langmuir sorption for a sorbent - Langmuir assumptions won't work in the presence of soil anyway

\begin{equation} \label{eq:PDM}
    C_{BC} = V_0\rho_0 \left [ \frac{-RT\ln \frac{C_{aq}}{S}}{E}\right ]^b
\end{equation}


where $C_{BC}$ is concentration in the biochar (g kg\textsuperscript{-1} dw), $V_0$ is BC pore volume (cm\textsuperscript{3} kg\textsuperscript{-1}), $\rho_0$ is sorbate density (g cm\textsuperscript{-1}, $R$ is the gas constant (J mol\textsuperscript{-1}K\textsuperscript{-1}, $T$ is the temperature (K), S is the maximum solubility (mg L\textsuperscript{-1}), $E$ is the free energy of adsorption (J mol\textsuperscript{-1}), and $b$ is a dimensionless fitting exponent. 

What is new: sorption of PFAS to sludge biochars is not yet studied. sludge char is being sold as soil amendment but cannot sell for as much money as sludge char as sorbent! Sorbents are more expensive. The results from this study may provide important economic advantages for Scanship and Lindum that can sell sludge char as sorbents for organic pollutants.

