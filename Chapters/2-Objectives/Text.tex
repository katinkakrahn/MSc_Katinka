\chapter{Research objectives and hypotheses}\label{chap:Objectives}
%%%%%%%%%%%%%%%%%%%%%%%%%%%%%%%%%%%%%%%%%%%%%%%%%%%%%%%%%%%%%
\section{Objectives}\label{sec:objectives}
The aim of this thesis is to study whether biochar from various waste materials can be used as sorbents in the remediation of PFAS-contaminated soil and water. The overall goal of this study is to test the degree to which sewage sludge-based biochars (\acrshort{SS-BCs}) can be used as effective sorbents for PFAS in the remediation of contaminated water and soil. To achieve this, the sub-goals are to:

\begin{enumerate}[label=\Roman*]
    \item{Compare the relative abilities of SS-BCs and clean wood chip BCs to sorb PFCA}
    \item{Identify possible sorption mechanisms of PFCA for biochar from different feedstocks}
    \item{Study the effects of perfluorinated carbon chain-length on PFCA sorption to SS-BCs}
    \item{Study the attenuation effect of competing sorbates and the presence of soil, on sorption, and to evaluate the effectiveness of SS-BCs as sorbents in real-world conditions}
\end{enumerate}

To achieve these objectives, sorption isotherms were determined for biochar from three different feedstocks: raw sewage sludge, digested sewage sludge, and clean wood chips. Six perfluorinated carboxylic acids (PFCA) with increasing chain lengths from C5-C10 were selected for spiking the sorption isotherms, both individually and as mixtures of these compounds. To study sorption attenuation, isotherms with biochar only, and isotherms with biochar mixed with a sandy, low-\acrshort{TOC} soil were prepared.

\section{Hypotheses \label{sec:hypothesis}}
\begin{enumerate}[label=\roman*]
    \item Biochar from sewage sludge is not as effective a sorbent for PFAS as biochar from clean wood chips due to its lower carbon-content and porosity, though it can be used as a low-cost, lower quality-class sorbent.
    \item The dominant mechanism by which PFCAs sorb to sewage sludge biochars is electrostatic attraction due to high ash contents rich in Ca and Fe. This leads to the next hypothesis: 
    \item Following from hypothesis (ii), sorption to sewage sludge biochars is not chain-length dependent and is due to electrostatic interactions by the negatively charged PFCA functional groups, whereas sorption increases with chain length for clean wood chips due to predominating hydrophobic interactions with the more carbonaceous biochar matrix.
    \item Sorption of PFCA to biochar is attenuated by the presence of soil and other PFCAs. 
\end{enumerate} 



