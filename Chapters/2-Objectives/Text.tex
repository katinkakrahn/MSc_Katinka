\chapter{Research objectives and hypotheses}\label{chap:Objectives}
%%%%%%%%%%%%%%%%%%%%%%%%%%%%%%%%%%%%%%%%%%%%%%%%%%%%%%%%%%%%%
\section{Objectives}\label{sec:objectives}
The overall goal of this study was to test the degree to which sewage sludge-based biochars can be used as effective sorbents for PFAS in the remediation of contaminated water and soil. To achieve this, sub-goals were to:

\begin{enumerate}[label=\Roman*]
    \item{compare the relative abilities of sewage sludge biochars and clean wood chips to sorb PFAS}
    \item{identify possible sorption mechanisms of PFAS for different biochar feedstocks}
    \item{study the effects of perfluorinated carbon chain-length on sorption to sewage sludge biochar}
    \item{study the attenuation effect of competing sorbates and the presence of soil on sorption, and to evaluate the effectiveness of sewage sludge biochars as sorbents in real-world conditions}
\end{enumerate}

To achieve these objectives, sorption isotherms were prepared using three different biochar feedstocks: raw sewage sludge, digested sewage sludge, and clean wood chips. Six perfluorinated carboxylic acids (PFCA) with increasing chain lengths from C5-C10 were selected for spiking the sorption isotherms, both individually and as mixtures of these compounds. To study sorption attenuation, isotherms with biochar only, and isotherms with biochar mixed with a sandy, low-TOC soil, were prepared.

If successful, the recycling of sewage sludge will be a cost-effective method for limiting the spread of PFAS in the environment, and serve as an alternative to sub-optimal waste treatment forms. Sewage sludge also has the potential to sequester carbon.

\subsection{Hypotheses}\label{sec:hypotheis}
The following are the hypotheses this thesis will seek to test:  
\begin{enumerate}[label=\roman*]
    \item Biochar from clean wood chips is a stronger sorbent than sewage sludge biochars in PFAS-contaminated water and soil, a quality that is related to it having a higher carbon-content and porosity
    \item Sorption increases with perfluorinated carbon chain-length
    \item Sorption of PFAS to biochar is attenuated at increasing concentrations
    \item Sorption of PFAS to biochar is attenuated in the presence of soil and competing congeners. 
\end{enumerate}






