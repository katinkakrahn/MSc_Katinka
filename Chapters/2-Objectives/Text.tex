\chapter{Research aims and objectives}\label{chap:Objectives}
%%%%%%%%%%%%%%%%%%%%%%%%%%%%%%%%%%%%%%%%%%%%%%%%%%%%%%%%%%%%%
\section{Research purpose}\label{sec:purpose}
This study is one of the first to investigate the potential for sewage sludge biochars to work as sorbents for PFAS.

What is new: sorption of PFAS to sludge biochars is not yet studied. sludge char is being sold as soil amendment but cannot sell for as much money as sludge char as sorbent! Sorbents are more expensive. The results from this study may provide important economic advantages for Scanship and Lindum that can sell sludge char as sorbents for organic pollutants. The aim of the laboratory investigations was to characterize the PFCA sorption ability of biochars produced from waste materials to determine if such biochars can introduced to the market as sorbents. 

The perfluorinated carboxylic acids (PFCAs) are the focus of this thesis.
If the findings in this study can be used to may provide promising outlooks for limiting spread of pollutants to the environment and rather valorizing the waste to sequester carbon and remediate contaminated sites. 

%%%%%%%%%%%%%%%%%%%%%%%%%%%%%%%%%%%%%%%%%%%%%%%%%%%%%%%%%%%%%
\section{Objectives}\label{sec:objectives}
The overall goal of the study is to test if sewage sludge biochar can be used as effective sorbents for PFAS for remediation of contaminated water and soil. Specifically, the objectives of this thesis are:

\begin{enumerate}
    \item{Compare the relative abilities of sewage sludge biochars and clean wood chips to sorb perfluorinated carboxylic acids (PFCAs)}
    \item{Identify possible sorption mechanisms of PFCAs for the different biochar feedstocks}
    \item{Study the effects of perfluorinated carbon chain-length on sorption }
    \item{Study the attenuation effect of competing sorbates and the presence of soil on sorption to evaluate the effectiveness of sewage sludge biochars as sorbents applied to real-world application settings}
\end{enumerate}

\subsection{Hypothesis}\label{sec:hypotheis}
The main hypothesis of this thesis was:
\textit{Biochar from clean feedstock and waste based feedstock reduce PFAS concentration in water.}

Three sub-hypotheses were tested:
\begin{itemize}
    \item Biochar from clean wood chips is a more effective sorbent than sewage sludge biochars in PFAS contaminated water and soil due to higher carbon fraction.
    \item PFAS sorption increases with chain length.
    \item Sorption of PFAS to biochar becomes non-linear at increasing concentrations.
    \item Sorption of PFAS to biochar is attenuated by the presence of soil and competing congeners. 
\end{itemize}
%%%%%%%%%%%%%%%%%%%%%%%%%%%%%%%%%%%%%%%%%%%%%%%%%%%%%%%%%%%%%
\section{Scope} 
What research has already been done on this subject? How does the work of this thesis stand out?
%%%%%%%%%%%%%%%%%%%%%%%%%%%%%%%%%%%%%%%%%%%%%%%%%%%%%%%%%%%%%
\section{Approach}
A literature study is aimed at...







