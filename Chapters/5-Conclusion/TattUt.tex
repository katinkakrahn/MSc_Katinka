Thus, another hypothesis was generated from this thesis: the predominant sorption mechanism for PFAS is a low SA/PV ratio and a higher  
Determining iron speciation for sewage sludge biochars could have made for an interesting discussion because wastewater treatment plants often use iron and aluminum-based precipitating chemicals, so the biochar is expected to contain high amounts of Fe and Al, where how their speciation ends up after pyrolysis per today is unknown.

Hypotheses (i), (ii) and (iii) were rejected. Hypothesis (i) was debunked by finding that the sewage sludge biochars (SS-BCs) tested in this research were better sorbents for PFCAs than biochar from clean wood chips (CWC). Despite the fact that the largest surface area belonged to CWC biochar, the highest fraction of its pores were too small to accommodate PFAS. The higher fraction of mesopores (2-50 nm) than those found in clean wood chips (CWC) explains why the SS-BCs turned out to be such effective sorbents. Furthermore, better sorption to ULS biochar was explained by a higher carbon content (30\%) than that of DSL biochar (14\%). Although mechanisms responsible for the stronger sorption of ULS and DSL were proposed, these could not be fully verified due to a small sample size of only three biochars. In addition to stronger sorption by the SS-BCs than that of CWC, their Freundlich distribution coefficients were equivalent to, or higher than several $\log~K_F$-values for activated carbon (AC) reported in previous literature. $\log~K_F$, in units of $\mathrm{(\mu g/kg)/(\mu g/L)^{n_F}}$, were in the range 5.73-6.00 for ULS biochar, 5.12-5.61 for DSL biochar, and 5.06-5.22 for CWC biochar for PFOA, PFNA, and PFDA (\cref{tab:summary_stats_single}). It may be noted, however, that since narrow concentration intervals (average $\Delta~\log~C_w$ = 1.3) were obtained for most of the isotherms, extrapolation of $\log~K_F$ beyond the $\mu~g L^-1$ scale is associated with higher uncertainty.



%This study has compared the relative abilities of sewage sludge biochars and clean wood chips to sorb perfluorinated carboxylic acids (PFCA). It has also identified possible sorption mechanisms of PFCA for the different biochar feedstocks, competing sorbates, and the presence of soil on sorption. 

The SS-BCs sorbed an average of 0.8 mg/g PFCAs in the mixed systems spiked at the highest concentration (10 mg/L total PFCA concentration).

in determining the sorption affinity of PFAS and other organic contaminants to sewage sludge biochars should be evaluated for a larger sample size.

%words:
%great, outstanding, valuable, profound, startling, astonishing, remarkable, extraordinary, sensational

effectively reduce PFAS concentrations in water by 88 $\pm~$ 22\%

These are related to controlling the spread and effects of increased use of short-chain replacements \citep{knutsen2019leachate}. Short-chain PFAS have higher mobility and tend to slip through existing water treatment processes, thereby contaminating food and drinking water \citep{hale2020persistent,brendel2018short}.

Lindum estimates that production costs for one tonne SS-BC is \texteuro 100 (rough estimation provided by Lindum AS). The market price for clean wood biochar is \texteuro 600 - \texteuro 1,000. As SS-BC will be a low-cost product, one can expect a starting market price of, say \texteuro 500.  This equates to a total revenue of \texteuro 680k in Norway and \texteuro 66M in Europe. Based on if one expects that SS-BC replaces AC completely.

\section{Life cycle assessment (LCA) \label{sec:LCA}}
To assess the overall environmental impacts associated with pyrolysis of sewage sludge, research partners are currently working on a life cycle assessment (LCA). Factors such as GHG burden, energy inputs versus energy recovery, environmental impacts, and economic profitability are weighted in such an assessment \citep{huang2022comparative}. Generally, the highest energy demands are associated with moisture removal and heating of the pyrolysis chamber since burning of biomass without oxygen is endothermic \citep{mcnamara2016pyrolysis}. Due to energy recovery, \cite{huang2022comparative} reported that biochar and bio-oil pyrolysis is more sustainable than conventional sludge treatment methods, and has the potential to generate acceptable levels of profit \cite{huang2022comparative}. The feedstocks used in the present research have highly different caloric contents, where DSL has the lowest caloric content because it already has been through anaerobic digestion for bio-gas production. If DSL can be introduced as a commercial sorbent, this will improve the LCA even further. More research is needed to look at the potential for using low-grade bio-oil and syn-gas byproducts from pyrolysis for energy. These byproducts share a range of drawbacks related to successfully integrating them into a circular economy: they often contain a good deal of water, PAHs and varying concentrations of heavy metals. 

\footnote{\url{https://ec.europa.eu/info/strategy/priorities-2019-2024/european-green-deal/delivering-european-green-deal_en}}
\footnote{\url{https://www.4p1000.org/}}
\footnote{\url{https://biochar-international.org/about-ibi/}}
\footnote{\url{https://www.ngi.no/eng/Projects/VOW-Valorization-of-Organic-Waste}}
\footnote{\url{https://zeropm.eu}}
\footnote{\url{https://www.nmbu.no/en/services/centers/earthresque}}
\footnote{\url{https://www.ngi.no/eng/Projects/SLUDGEFFECT}}
\footnote{\url{https://perforce3-itn.eu}}

 total PFCA for ULS and DSL respectively, removing 86\% of PFOA for both DSL and ULS.  $\log~K_ds$ for PFOA were 4.13 and 3.17 for ULS and DSL biochars ($n_F$=0.39) respectively, . Since the biochars exposed to 10 mg/L the ULS and DSL biochars reduce wastewater PFAS concentration by respectively 96.4 and 74.7\%, resulting in equilibrium aqueous concentrations of 357 and 2,526 $\mathrm{\mu g/L}$. 
 
 (200 $\mathrm{\mu g/L}$
 
 The sorption capacity of the SS-BCs at this point would result in total aqueous PFOA concentrations of $\sim 1.4 mg/L PFCA C5-C10$,
 Based on currently available data, estimates could be made of the sorbent-capacity of filters using sewage sludge-based biochars.