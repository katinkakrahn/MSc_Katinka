\chapter{Implications, conclusion and recommendations for further work}\label{chap:Conclusion}
This study has compared the relative abilities of sewage sludge biochars and clean wood chips to sorb perfluorinated carboxylic acids (PFCA). It has also identified possible sorption mechanisms of PFCA for the different biochar feedstocks, competing sorbates, and the presence of soil on sorption. Overall, the data generated in this research both confirm the hypotheses defined in \cref{sec:hypotheis}, and open up for new research questions within the relatively novel research area of sorption of PFAS to sewage sludge biochars.

The results from the batch sorption tests show that biochars from Ullensaker sludge (ULS) and digested sludge Lindum (DSL) are stronger sorbents than clean wood chips (CWC) (\cref{fig:sorption_isotherms}). This was the most unexpected falsification of hypothesis (i). The Freundlich coefficients ($\log~K_F$) generated from the sorption isotherms are equivalent to, or higher than the, $\log~K_F$-values for activated carbon (AC) reported in previous literature. The results from this work suggest that stronger sorption of PFCAs to sewage sludge biochars is likely due to a higher fraction of mesopores (2-50 nm) than those found in clean wood chips (CWC). Furthermore, a hypothesis explaining why PFCAs sorb stronger to ULS and DSL than to CWC is that ULS and DSL have higher carbon-fractions in their pore wall matrices (lower $\log~SA/PV/C$ ratio). Although mechanisms responsible for the stronger sorption of ULS and DSL were proposed, these could not be fully verified due to a small sample size of only three biochars. 

The fact that sorption increases with increasing perfluorinated chain-length, as shown by an increase in $\log~K_F$ for each $\mathrm{CF_2}$ moiety (\cref{tab:summary_stats_single}, \cref{fig:chainlength}), verifies hypothesis (ii). Although poor correlations were sometimes acquired for the short-chain PFCAs (PFPeA, PFHxA and PFHpA), this information itself supports the hypothesis of a chain length dependency of PFAS sorption to biochar. This relationship suggests that hydrophobic interactions plays a predominant role in sorption and electrostatic interactions to a lesser extent. 

Hypothesis (iii) could be verified by Freundlich coefficients of non-linearity ($n_F$) $<$1 (\cref{tab:summary_stats_single}), and by observing a gradual decrease in uptake into the solid phase ($C_s$) with increasing concentrations (\cref{fig:nonlinear_OND}). However, since narrow concentration intervals (average $\Delta~\log~C_w$ = 1.3) were obtained for most of the isotherms, extrapolation of $\log~K_F$ beyond the $\mu~g L^-1$ scale is associated with higher uncertainty. Hypothesis (iv) was confirmed by yielding lower $\log~K_F$ and $\log~K_d$ for the batch tests with soil and PFCA cocktail (\cref{fig:C10_AF}). The order of which sorption was attenuated in the different batch test categories compared to the reference (single-compound biochar-water batch tests), suggests that attenuation by soil is less than by the presence of high concentrations of a mixture of PFAS. This suggests that the sorbents could be more effective in waste water treatment than in soil remediation, but especially in moderately contaminated wastewater. 

This thesis has proposed a set of mechanisms that are expected to explain the differences in sorption of PFCAs C5-C10 to raw sewage sludge, digested sludge, and clean wood chips. This study is one of the first to show that biochar made from two different sewage sludge substrates can be used to effectively reduce PFAS concentrations in water by 88 $\pm~$ 22\%. The strong sorption of PFCAs found to sewage sludge biochars is promising for their incorporation in a circular economy as sorbents for wastewater treatment, or as amendments to PFAS-contaminated soil. 

Future research could further investigate the ratios between surface area, pore volume, carbon, and minerals (mainly Ca and Fe) in determining the sorption affinity of PFAS and other organic contaminants to sewage sludge biochars. For more definitive conclusions than those generated in this thesis, multivariate regression analyses of these factors should be done with a larger biochar sample size. Further work could also focus on evaluating sewage sludge biochars as sorbents in field conditions. Attention could also be given to investigating the relationship between soil organic carbon and the biochar dosage required to reduce PFAS pore water concentrations down to environmental quality standards. 

More research:
capacity tests for filters
other PFAS, especially PFSAs
competition with PAH and other contaminants
effect of activation on sludge chars

\section{Potential for commercializing sludge chars as sorbents}
The sorption capacity of the sewage sludge biochars was tested in a controlled laboratory environment, with a low-TOC soil, and artificially spiked PFCAs. Therefore, the results from this research does not account for the heterogeneity in factors like different soil types and organic matter contents \citep{Sormo2021}, diversity of organic contaminants, and variations in PFAS concentration levels that may influence sorption strength. However, the results show that sewage sludge biochars bind PFCA strongly, especially for the long-chain compounds at concentrations that are many times higher than actual environmentally relevant concentrations. Separate investigations should also be conducted to determine the sorption capacity of biochars, work that is especially important to establish how long a carbon filter for wastewater treatment can be used. Care must be taken when applying a highly contaminated feedstock to make sure no heavy metals are leached into the environment and no additional PFAS is released from the sewage sludge. Results from a parallel project with this thesis will hopefully be able to provide a fuller understanding of a range of issues, including gaseous pollution measurements, biochar PFAS content, and particulate matter measurements. The hope is that PFAS is eliminated at pyrolysis temperatures.

\subsection{Life cycle assessment (LCA) \label{sec:LCA}}
To assess the overall environmental impacts associated with pyrolysis of sewage sludge, research partners are currently working on a life cycle assessment (LCA). Factors such as GHG burden, energy inputs versus energy recovery, environmental impacts, and economic profitability are weighted in such an assessment \citep{huang2022comparative}. Generally, the highest energy demands are associated with moisture removal and heating of the pyrolysis chamber since burning of biomass without oxygen is endothermic \citep{mcnamara2016pyrolysis}. Due to energy recovery, \cite{huang2022comparative} reported that biochar and bio-oil pyrolysis is more sustainable than conventional sludge treatment methods, and has the potential to generate acceptable levels of profit \cite{huang2022comparative}. The feedstocks used in the present research have highly different caloric contents, where DSL has the lowest caloric content because it already has been through anaerobic digestion for bio-gas production. If DSL can be introduced as a commercial sorbent, this will improve the LCA even further. More research is needed to look at the potential for using low-grade bio-oil and syn-gas byproducts from pyrolysis for energy. These byproducts share a range of drawbacks related to successfully integrating them into a circular economy: they often contain a good deal of water, PAHs and varying concentrations of heavy metals. ¨

\subsection{Back-of-the-envelope calculations} 
The voluntary carbon credit marked will be an increased incentive for the sewage sludge biochar (SS BC) manufacturers. 

Production of SS BC does not have to stop for the use as sorbents. 200 000 tonnes of sewage sludge is generated annually, and all this cannot be sold as sorbents. It is possible, however, that the remaining ss can still be pyrolyzed into BC and used 



 






