\chapter{Conclusion and recommendations for further work}\label{chap:Conclusion}
This study has compared the relative abilities of sewage sludge biochars and clean wood chips to sorb perfluorinated carboxylic acids (PFCA). It has also identified possible sorption mechanisms of PFCA for the different biochar feedstocks, and studied the attenuation effects of increasing perfluorinated carbon chain-length, competing sorbates, and the presence of soil on sorption. Overall, the data generated in this research both confirm the hypotheses defined in \cref{sec:hypotheis}, and open up for new research questions within the relatively novel research area of sorption of PFAS to sewage sludge biochars.

The results from the batch sorption tests show that Ullensaker sludge (ULS) and digested sludge Lindum (DSL) are stronger sorbents than clean wood chips (CWC) (\cref{fig:sorption_isotherms}). This was the most unexpected falsification of hypothesis (i). The Freundlich coefficients ($\log~K_F$) generated from the sorption isotherms are equivalent to, or higher than the, $\log~K_F$-values for activated carbon (AC) reported in previous literature. The results from this work suggest that stronger sorption of PFCAs to sewage sludge biochars is likely due to a higher fraction of mesopores (2-50 nm) than those found in clean wood chips (CWC). Furthermore, a hypothesis explaining why PFCAs sorb stronger to ULS and DSL than to CWC is that ULS and DSL have higher carbon-fractions in their pore wall matrices (lower $\log~SA/PV/C$ ratio). Although mechanisms responsible for the stronger sorption of ULS and DSL were proposed, these could not be fully verified due to a small sample size of only three biochars. 

The fact that sorption increases with increasing perfluorinated chain-length, as shown by an increase in $\log~K_F$ for each $\mathrm{CF_2}$ moiety (\cref{tab:summary_stats_single}, \cref{fig:chainlength}), verifies hypothesis (ii). Although poor correlations were sometimes acquired for the short-chain PFCAs (PFPeA, PFHxA and PFHpA), this information itself supports the hypothesis of a chain length dependency of PFAS sorption to biochar. Hypothesis (iii) could be verified by Freundlich coefficients of non-linearity ($n_F$) $<$1 (\cref{tab:summary_stats_single}), and by observing a gradual decrease in uptake into the solid phase ($C_s$) with increasing concentrations (\cref{fig:nonlinear_OND}). However, since narrow concentration intervals (average $\delta~\log~C_w$ = 1.3) were obtained for most of the isotherms, extrapolation of $\log~K_F$ beyond the $\mu~g L^-1$ scale is associated with higher uncertainty. Hypothesis (iv) was confirmed by yielding lower $\log~K_F$ and $\log~K_d$ for the batch tests with soil and PFCA cocktail (\cref{fig:C10_AF}). The order of which sorption was attenuated in the different batch test categories compared to the reference (single-compound biochar-water batch tests), suggests that attenuation by soil is less than by the presence of high concentrations of a mixture of PFAS. 

This thesis has proposed a set of mechanisms that are expected to explain the differences in sorption of PFCAs C5-C10 to raw sewage sludge, digested sludge, and clean wood chips. This study is one of the first to show that biochar made from two different sewage sludge substrates can be used to effectively reduce PFAS concentrations in water by 88 $\pm~$ 22\%. The strong sorption of PFCAs found to sewage sludge biochars is promising for their incorporation in a circular economy as sorbents for wastewater treatment, or as amendments to PFAS-contaminated soil. Future research could further investigate the ratios between surface area, pore volume, carbon, and minerals (mainly Ca and Fe) in determining the sorption affinity of PFAS and other organic contaminants to sewage sludge biochars. For more definitive conclusions than those generated in this thesis, multivariate regression analyses of these factors should be done with a larger biochar sample size. Further work could also focus on evaluating sewage sludge biochars as sorbents in field conditions. Attention could also be given to investigating the relationship between soil organic carbon and the biochar dosage required to reduce PFAS pore water concentrations down to environmental quality standards. 



 






