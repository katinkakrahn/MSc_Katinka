\chapter{Implications, conclusion and recommendations for further work}\label{chap:Conclusion} \chaptermark{Conclusion}
The data generated in this research has yielded promising results for the future use of sewage sludge-based biochar (SS-BC) as a sorbent. To begin the summary of the results obtained, the initial hypotheses that represented a point of departure for the research presented, are reiterated:

\begin{enumerate}[label=\roman*]
    \item Biochar from sewage sludge is not as effective a sorbent for PFAS as biochar produced from clean wood chips due to its lower carbon-content and porosity, though it can be used as a low-cost, lower quality-class sorbent.
    \item The dominant mechanism by which PFCAs sorb to sewage sludge biochars is electrostatic attraction due to mineral-rich components. 
    \item Following from hypothesis (ii), sorption to sewage sludge biochars is not chain-length dependent and is due to electrostatic interactions by the negatively charged PFCA functional group, whereas sorption increases with chain length for clean wood chips due to predominating hydrophobic interactions with the more carbonaceous matrix of this biochar.
    \item Sorption of PFCA to biochar is attenuated by the presence of soil and other PFCAs. 
\end{enumerate}

Hypotheses (i), (ii) and (iii) were rejected. Hypothesis (i) was falsified by discovering that the SS-BCs tested in this research were \textit{better} sorbents for PFCAs than biochar from clean wood chips (CWC). $\Log~K_F$, in units of $\mathrm{(\mu g/kg)/(\mu g/L)^{n_F}}$, were in the range 5.73-6.00 for ULS biochar, 5.12-5.61 for DSL biochar, and 5.06-5.22 for CWC biochar for PFOA, PFNA, and PFDA (\cref{tab:summary_stats_single}). Despite the fact that CWC biochar had the largest surface area, the majority of its pores were too small to accommodate PFAS. The higher fraction of mesopores (2-50 nm) in the sewage sludge biochars compared to those found in clean wood chips (CWC) probably explains why the SS-BCs turned out to be such effective sorbents. Furthermore, better sorption to ULS biochar was explained by a higher carbon content (30\%) than that of DSL biochar (14\%).

Hypothesis (ii) was rejected by the observation of higher $\log~K_d$'s (at $C_w = 1 \mathrm{\mu g/L}$) with increasing perfluorinated chain length for all biochar feedstocks (\cref{fig:chainlength}). Since no apparent correlations were found between Ca or Fe-content and $\log~K_d$, electrostatic interactions are likely not the primary sorption mechanism for the SS-BCs. Since strongest sorption was measured for the biochars with higher porosity and carbon content, hydrophobic interactions is likely the dominant sorption mechanism for PFCAs to the SS-BCs. Although mechanisms responsible for the stronger sorption of ULS and DSL were proposed, these could not be fully verified due to the limited sample size of only three biochars. 

The conclusions drawn from hypothesis (ii) are directly linked to hypothesis (iii) which was also rejected for the same reasons. The poor Freundlich sorption isotherm correlations acquired for the short-chain PFCAs (PFPeA, PFHxA and PFHpA) supports the chain length-dependency of PFCAs to ULS, DSL, and CWC biochars. This relationship suggests that sorption of PFCA to SS-BCs was mainly governed by hydrophobic interactions between the C-F chain and carbonaceous aromatic surfaces. Since mineral contents were high in the strong-sorbing SS-BCs, a larger sample size is needed to understand how these influence sorption strength and mechanisms. 

Finally, hypothesis (iv) proved to be correct. Attenuation factors (AFs) ranged from 3-10 for PFOA in the presence of soil, 6-140 for PFOA, PFNA, and PFDA in a mixture of PFCAs, and 8-138 for PFOA, PFNA, and PFDA in the presence of soil and other PFCAs (\cref{tab:attenuation}). Despite large variations in AFs, the presence of other PFCAs was responsible for a stronger degree of attenuation than the presence of soil. The presence of soil in the PFCA-mixture increased attenuation by an additional 27\%. This suggests that the sorbents could be more effective for filtration in waste water treatment than in soil remediation, and especially in moderately contaminated wastewater. 

%%%%%%%%%%%%%%%%%%%%%%%%%%%%%%%%%%%%%%%%%%%%%%%%%%%%%%%%%%%%%%%%%%%%%%%%%%%%%%%%%%%%%
%%%%%%%%%%%%%%%%%%%%%%%%%%%%%%%%%%%%%%%%%%%%%%%%%%%%%%%%%%%%%%%%%%%%%%%%%%%%%%%%%%%%%
\section{Application of sewage sludge biochar in the treatment of wastewater \label{sec:application}}
The results from this study demonstrate that the ULS and DSL biochars bind PFCA strongly at concentrations that are many times higher than most environmental concentrations. The SS-BCs tested have shown to sorb equally strongly, or stronger than, several activated carbons (AC) reported in previous literature \citep{Kupryianchyk2016b, hansen2010sorption, silvani2019can}. Therefore, some rough estimates have been made for how effective the ULS and DSL biochars are for removing PFAS from wastewater in Norway. 

Based on samples obtained by a Norwegian survey conducted in late 2020 (Gabriela Castro Varela, NTNU, unpublished data), Norwegian wastewater treatment plant influents contain between 0.1-1 $\mathrm{\mu g/L}$. The Freundlich distribution coefficients derived for the ULS and DSL biochars from the BC-soil-mix sorption isotherms can be used to estimate the sorption efficiency of the biochars for Norwegian wastewater. The BC-soil-mix isotherms were chosen because one can assume attenuation when they are used in treating wastewater. Removal efficiency of PFOA was used for simplicity. This estimate assumes no kinetic limitations during the wastewater treatment procedure, equilibrium conditions, and a water/BC ratio of 500. Linear sorption ($n_F = 1)$ can be assumed for the application of unused SS-BC to Norwegian wastewater because the wastewater concentration was two orders of magnitude below the lowest spiked concentration for the isotherms. Thus, $\log~K_F$ = $\log~K_d$ = 5.00 and 4.91 for ULS and DSL biochars respectively. To be conservative, these calculations were based on a PFOA wastewater concentration of $1 \mathrm{\mu g/L}$. These values result in the potential reduction of PFOA from wastewater by 99.5 and 99.4\%, that is, equilibrium aqueous concentrations of 0.005 and 0.006 $\mathrm{\mu g/L}$ by the ULS and DSL biochars. Clearly, these SS-BCs have the potential to be effective in removing PFAS from Norwegian wastewater. 

Equilibrium aqueous concentrations measured at higher concentrations say something about the sorption capacities of the ULS and DSL biochars, and can therefore give an indication of how long carbon filters used for wastewater treatment can be expected to reduce PFAS concentration to an acceptable level. As is widely accepted, and supported by the findings of this thesis, sorption to biochar is weaker (attenuated) at higher contaminant concentrations. Due to the kinetic aspects of the sorption process, the following assessment is only \textit{indicative} of filtration capacity as there are differences between a batch test with a spiked concentration, and the steady flow of a lower concentration. Using the same assumptions underlying the previous example, but now with an initial PFOA concentration of 1.9 mg/L in a total PFCA cocktail concentration of 10 mg/L, 85.5 and 77.7\% would be retained by ULS and DSL respectively, versus $>$99\% at 1 $\mathrm{\mu g/L}$. The sorption capacity of ULS and DSL biochars at this point would result in equilibrium aqueous PFOA concentrations of 0.28 and 0.44 mg/L---post-treatment levels that would be unacceptable. Hence, the filter would need to be exchanged before this point. Tests that account for wastewater residence time, flow rate, bed volumes, and sorbent dose, are factors that would need to be considered in order to design a water purification solution that uses SS-BCs, and is beyond the scope of the present study. 

\subsection{Considerations for commercializing sludge chars as sorbents}
The sorption strength of sewage sludge biochars was tested in a controlled laboratory environment, with low-TOC soil, and artificially spiked PFCAs. The results from this research do not, therefore, take into account the heterogeneity of sewage sludges, variations in the organic matter contents of soils that the sorbents will potentially be applied to, and the potential competition for sorption sites from a range of other possible contaminants in soil and water in need of remediation. All of the variables mentioned above will influence the sorption strength of SS-BCs. 

Since the results from this study showed poor sorption affinity to PFPeA, PFHxA, and PFHpA, consideration should also be given to ensure that short-chain PFASs do not slip through wastewater treatment systems that use SS-BC. For this reason, Lindum considers the application of SS-BCs more as a pre-treatment sorbent, to be followed by treatment using a stronger sorbent that is more effective in retaining short-chain PFAS. Also important to consider is the potential risk of residual contaminants in biochars produced from contaminated feedstocks. 

%The Danish limit for 2 ng/L PFOA, PFOA, PFNA, and PFHxS for safe drinking water has not been reached \citep{Danmark2021grenseverdier}, appendix 1 d. 2 ng/L is considered safe levels for drinking water. Attention could also be given to investigating the relationship between soil organic carbon and the biochar dosage required to reduce PFAS pore water concentrations down to the environmental quality standards in soils determined by Norwegian Environment Agency (2020) which are 0-9.1μg/L PFOA for class II/good. 
%\citep{PFOS2018grenser}: Norwegian Water Framework Directive: PFOS in surface water limit 36 $\mathrm{\mu g/L}$. 

\section{Revenue estimates for commercial production of sewage sludge biochar}
The following is a case example that gives a rough idea of the revenue potential of a scenario in which 100 \% of the activated carbon used for wastewater treatment is replaced with SS-BC. The Norwegian WWTP, VEAS, purchases 2 tonnes of AC annually for wastewater treatment. Market prices for AC range from 600 - 1,000 \texteuro/tonne. For VEAS, replacing AC with SS-BC would result in annual savings from 200 - 1,000 \texteuro, equivalent to a cost reduction of 17-50\%. This assumes that SS-BC has a starting price of 500 \texteuro/tonne. How many WWTPs are there in Norway?

A second case presents a prime example of a circular economy scenario. Wastewater from Ullensaker WWTP, the project partner that provided the ULS sludge analyzed in this thesis, contains 3.8 $\mathrm{\mu g/L}$ (Varela, unpublished). This is the highest level of PFAS concentration found among the six Norwegian WWTPs monitored. High concentrations of PFAS at the Ullensaker WWTP have been attributed to the use of aqueous film forming foam at a nearby firefighting training facility \citep{Hale2017fire}. In this scenario, Ullensaker WWTP adapts pyrolysis technology that can produce biochar from the high PFAS-level wastewater they are paid to remediate. The resulting ULS biochar is applied directly as a PFAS sorbent to wastewater. With an influent total PFAS concentration of 3.8 $\mathrm{\mu g/L}$ in Ullensaker wastewater, linear sorption can be assumed. Based on the BC-S-mix isotherm used in this study, $\log~K_ds$ for ULS biochar are 5.00, 5.22, and 5.62 for PFOA, PFNA, and PFDA. reduces water concentrations to $<$20 ng/L ($>$99.5\%). Applying this solution at the Ullensaker WWTP site would mean eliminating transportation-related costs. In addition, SS-BCs have low production costs that are estimated to be 100 \texteuro/tonne. If SS-BCs can be produced for 100\texteuro/tonne, and sold for 500 \texteuro/tonne, the revenue from pyrolyzing all sewage sludge in Norway would be 3,600,000 \texteuro/year = 36 million NOK annually. 

In summary, there appear to be several benefits associated with replacing AC with SS-BC: 1) PFAS present in wastewater can, for all intents and purposes, be eliminated by pyrolysis. Preliminary data from one of the partners in this study shows reductions from 90-95\% (Erlend S{\o}rmo unpublished), 2) biochar that previously contained PFAS can now be applied to sorb even more PFAS-contaminated sludge right at the production site, 3) reduced annual expenses for the purchase of currently available commercial sorbents, and 4) WWTPs can increase their profitability and at the same time contribute to carbon sequestration. A rough calculation of the carbon sequestration potential of SS-BCs has been made and is presented in the following section. 

\section{Carbon sequestration potential}
Since a commercial trade in carbon credits has opened up, the potential sale of SS-BCs at a global scale could be highly attractive. Assuming that SS-BC replaces the entire annual demands for AC in Norway and Europe, it is possible to make a rough estimate of the carbon sequestration potential of SS-BCs. Carbon dioxide equivalents (\acrshort{CO2eq}) can be calculated, where one $\mathrm{CO_2-eq}$ is the equivalent of one tonne of sequestered $\mathrm{CO_2}$. Annual demands for AC in Norway are approximately 1,700 tonnes/year, and 165,000 tonnes/year for Europe as a whole \citep{schmidt2019pyrogenic}. On average, sewage sludge biochars contain 20\% carbon (C), of which 70-80 \% is assumed to be stable over time \citep{schmidt2019pyrogenic}. This is equivalent to 1,000 and 100,000 annual CO\textsubscript{2}-eq for Norway and Europe respectively. Considering the current EU CO\textsubscript{2} trading price of 90 \texteuro/CO\textsubscript{2}-eq, this translates into annual carbon credit values of 90,000 and 8,800,000 \texteuro. 

For the production of the 1,700 tonnes of SS-BC needed annually to replace AC, assuming 30\% yield, a mere 7\% of the 200,000 tonnes w.w. (30,000 d.w.) of sewage sludge generated in Norway each year would be needed. This fraction could be much higher if soil remediation can also be done with SS-BC as well as the application of SS-BC as fertilizers in agriculture. However, much more research is needed before this point can be reached. Despite a wide range of both technical and legislative challenges, a great deal of research is being done to corroborate initial research that suggests that nearly all microplastics, heavy metals, and organic pollutants, including PFAS, are either combusted or immobilized at high pyrolysis temperatures (S\o rmo, unpublished). In a best-case-scenario, the 93\% of sewage sludge remaining after sorbent production could still be pyrolyzed into BC, and used, for example, in place of bio-based fertilizers in agriculture. \textit{Raw} biosolids contain microplastics, heavy metals, and various organic pollutants. These same masses can potentially safely be applied to agricultural fields \textit{after} pyrolysis. In addition, though further research is needed, pyrolyzed sludge could prove to increase soil fertility more effectively than non-pyrolyzed biosolids. Not only would this represent increased carbon sequestration, but it would serve as an additional incentive for SS-BC manufacturers to maximize their biochar yields from raw sewage sludge, thereby creating products that are attractive on the voluntary carbon credit market. 

It is tempting to further speculate on the revenues a company like Lindum AS could expect by processing all sewage sludge using pyrolysis. They currently generate revenues by 1) treating sewage sludge, and 2) using some of the fractions to produce bio-gas. Managing to utilize the final fraction, digestate, could further increase the company's revenues per tonne of processed raw sewage sludge. SS-BC production costs of 100 \texteuro/tonne will likely represent a tiny fraction of the total revenue that can be expected from each tonne of raw sewage sludge received. 

%%%%%%%%%%%%%%%%%%%%%%%%%%%%%%%%%%%%%%%%%%%%%%%%%%%%%%%%%%%%%%%%%%%%%%%%%%%%%%%%%%%%%
%%%%%%%%%%%%%%%%%%%%%%%%%%%%%%%%%%%%%%%%%%%%%%%%%%%%%%%%%%%%%%%%%%%%%%%%%%%%%%%%%%%%%

\section{Waste-based biochar for sustainable development \label{sec:SDGs}}
Within Norway and the EU, a number of research initiatives have been taken to study possible applications of waste-based biochar sorbents that could contribute to achieving the following   2030 United Nations Sustainable Development Goals (\acrshort{SDGs}) \citep{SDGs2015}:
\begin{itemize}
    \item \textbf{SDG 6}:  Clean water and sanitation
    \item \textbf{SDG 9}:  Industry, innovation and infrastructure
    \item \textbf{SDG 11}: Sustainable cities and communities
    \item \textbf{SDG 12}: Responsible consumption and production
    \item \textbf{SDG 13}: Climate action
    \item \textbf{SDG 14}: Life below water
    \item \textbf{SDG 15}: Life on land
\end{itemize}

and the European Green Deal of 2020. The European Green Deal of July 14, 2021 lists development of biochar as one of the most important economic means to achieve the goal of making Europe the first climate-neutral continent in the world. The 4 per mille initiative, Soils for Food Security and Climate, was launched by France during the UN climate change conference in December, 2015 (COP21). The 4 per 1000 mission is to increase soil carbon stocks in the first 30-40 cm of soil by 4 \textperthousand  annually as a means of complementing what is necessary efforts to compensate for global fossil-based GHG emissions. The goal is to scale biochar for global agricultural and remediation markets. If applied correctly, biochar can play an important role in sequestering carbon, and at the same time serve as a soil amendment and fertilizer in agriculture. However, this must be done carefully because application of biochar in soil has been shown to have a priming effect on soil-native C by improving microbial populations, thereby influencing the decomposition rate of organic matter \citep{Ahmad2014}. In most cases however, negative priming is seen, and biochar actually helps to increase natural organic matter content in soils \citep{chen2019competitive,weng2018accumulation}. The International Biochar Initiative (IBI) is a collaborative platform for science, industry, agriculture, government, and non-governmental organizations, such as the European Biochar Certificate (EBC) which creates awareness about the benefits of biochar, and develops biochar standards for safe and sustainable use. 

\section{Final recommendations}
This thesis has proposed a set of mechanisms that are expected to explain the differences in sorption of PFCAs to biochar derived from raw sewage sludge, digested sludge, and clean wood chips. It has indicated the significant potential of biochar in a circular economy, and its application as a sorbent for wastewater treatment, or as an amendment to PFAS-contaminated soil. This study is one of the first to show that biochar made from two different sewage sludge substrates can be used to reduce PFAS concentrations in water by 88 $\pm$ 22\% at a dosage of only 0.1 g. These results and their analysis also raise new research questions within the relatively recent research area of PFAS-sorption to sewage sludge biochars. Conducting multivariate regression analyses between PFAS sorption strenght and factors such as surface area, pore volume, carbon content, and mineral contents (mainly Ca and Fe) will be necessary to confirm the sorption mechanisms proposed in this work, and to draw more definitive conclusions than those presented here.  In addition to studying the effect of activation of sludge chars on sorption strength, additional research could also determine capacity tests for sewage sludge-based biochar filters, sorption strength to other PFAS groups, especially PFSA, competition with PAHs, and other contaminants that are commonly present in wastewater. Attention could also be given to investigating the relationship between soil organic carbon, and the biochar dosage required to reduce PFAS pore water concentrations to an acceptable level. 

Increased potential for carbon sequestration, and increased commercial valorization of sewage sludge, are highly promising for the incorporation, and ultimately, the replacement of existing fossil-derived sorbents with sewage-sludge based sorbents. Should further studies corroborate the findings presented in this study, the commercial and environmental benefits would be nothing less than sensational. 







