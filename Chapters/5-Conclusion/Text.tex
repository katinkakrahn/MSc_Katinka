\chapter{Conclusion}\label{chap:Conclusion}

Sorption increases from CWC < DSL < ULS, and with increasing perfluorinated chain-length. The Freundlich sorption coefficients (log KF) found in this study are equivalent to, or higher than, log KF  values for activated carbon reported in previous literature
Stronger sorption of PFCAs to sewage sludge biochars is likely due to a higher fraction of mesopores (2-50 nm, Fig. 5c)
A higher carbon-fraction in the pore wall matrix (lower log SA/PV/C ratio) of ULS is hypothesized to explain why PFCAs sorb stronger to ULS than to DSL 

The strong sorption of PFCAs found to sewage sludge biochars is promising for their incorporation in a circular economy, for example their use as fertilizers, sorbents in wastewater treatment plants, or as amendments to PFAS-contaminated soil
Future work should aim at further investigating the ratios between surface area, pore volume, carbon, and minerals (mainly Ca and Fe) in determining the sorption affinity of PFAS and other organic contaminants to sewage sludge biochars






