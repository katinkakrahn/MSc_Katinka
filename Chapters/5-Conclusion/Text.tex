\chapter{Conclusion and recommendations for further work}\label{chap:Conclusion}

Sorption increases from CWC $<$ DSL $<$ ULS, and with increasing perfluorinated chain-length. The Freundlich sorption coefficients (log KF) found in this study are equivalent to, or higher than, log KF  values for activated carbon reported in previous literature
Stronger sorption of PFCAs to sewage sludge biochars is likely due to a higher fraction of mesopores (2-50 nm)
A higher carbon-fraction in the pore wall matrix (lower log SA/PV/C ratio) of ULS is hypothesized to explain why PFCAs sorb stronger to ULS than to DSL 

The strong sorption of PFCAs found to sewage sludge biochars is promising for their incorporation in a circular economy, for example their use as fertilizers, sorbents in wastewater treatment plants, or as amendments to PFAS-contaminated soil
Future work should aim at further investigating the ratios between surface area, pore volume, carbon, and minerals (mainly Ca and Fe) in determining the sorption affinity of PFAS and other organic contaminants to sewage sludge biochars

This thesis has developed a mechanistic understanding of the sorption of six PFCAs to three biochars by spiking water with known concentrations of individual PFCAs and cocktails. Future work should look more into relating this work to field-scale applications, real-life relevance. Include your idea with finding an attenuation factor for soils of increasing \%TOC to adjust the dose biochar needed to reduce PFAS pore water concentration to EQS levels. 

Future work should aim at further investigating the ratios between surface area, pore volume, carbon, and minerals (mainly Ca and Fe) in determining the sorption affinity of PFAS and other organic contaminants to sewage sludge biochars

Multivariate regression with many chars to determine biochar properties that best contribute to sorption.

Determining iron speciation for sewage sludge biochars could have made for an interesting discussion because wastewater treatment plants often use iron and aluminum-based precipitating chemicals, so the biochar is expected to contain high amounts of Fe and Al, where how their speciation ends up after pyrolysis per today is unknown. 






